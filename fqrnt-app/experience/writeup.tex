\documentclass[11pt,letterpaper]{article}

\author{Jacob Thomas Errington}
\title{Experience}
\date{}

\usepackage[margin=2.0cm]{geometry}

\usepackage{hyperref}

\begin{document}

\maketitle

\section{Publications}

\subsection{Bioinformatics}

\begin{itemize}
  \item
    Baharian S, Barakatt M, Gignoux CR, Shringarpure S, Errington J, Blot WJ,
    et al. (2016) The Great Migration and African-American Genomic Diversity.
    PLoS Genetics 12(5): e1006059.
    \url{https://doi.org/10.1371/journal.pgen.1006059}
\end{itemize}

\section{Internships and research projects}

In the summer preceding the start of my bachelors, I joined the lab of Simon
Gravel\footnotemark{} as a casual research assistant.
%
\footnotetext{\url{http://simongravel.lab.mcgill.ca/}}%
%
I initially worked on data visualization software in Python. I later
implemented a new demographic model for tracts,\footnotemark{} to resolve an
issue with the existing model when faced with large populations due to an
averaging of ancestry proportions.
%
\footnotetext{\url{https://github.com/sgravel/tracts}}%
%
The results of this new model were included in the paper \emph{The Great
Migration and African-American Genomic Diversity} mentioned above.
In the context of an independent studies course, I worked with Simon Gravel on
a new method for unconstrained optimization of multivariate functions, based
his earlier work\footnotemark{} on constraint satisfaction.
%
\footnotetext{\url{https://arxiv.org/abs/0801.0222}}

In spring 2015, I worked with three friends on a mobile application for
matching part-time contractors with jobs in the restaurant and nightlife
industries. We incorporated a company, and were accepted to the first cohort of
the McGill X-1 Startup Accelerator\footnotemark{} alongside four other companies.
%
\footnotetext{\url{http://mcgillx1accelerator.com/}}

In early 2016, I started work at OOHLALA Mobile\footnotemark{} as a software
engineer while continuing to pursue my bachelors. Due to the small size of the
development team, I quickly found myself in a key position, as the lead
engineer of the new integrations framework. This framework exists to provide a
structured way to define new integrations between our customers' technical
infrastructures and our database.
%
\footnotetext{\url{https://oohlalamobile.com/}}

In spring 2016, I took a course on compiler design, taught by Dr. Laurie
Hendren, at McGill. The course centers around a project: to design and
implement a compiler for a subset of Go called GoLite. Ordinarily, this project
is completed in teams consisting of three members, but for an additional
challenge, our team consisted of only two members. Our compiler used a
generalized representation for the syntax tree, which gave us a type-safe way
to attach annotations to the nodes in the tree. These annotations played
different roles at different stages of compilation. At first, we used empty
annotations to represent programmatically generated trees used in test cases.
We used annotations to hold the locations in the source code of the different
nodes. Finally, we augmented the annotations with typing information.
%
Our compiler featured a partial x86\_64 code generator and a complete C++ code
generator. Our C++ code generator produced faster code than the reference
compiler written by the teaching assistant.
%
For outstanding achievement in the course, Professor Hendren emailed me to
award me an ``honourary A+'' grade in the course, as McGill does not have an A+
grade.

In fall 2016, I took the graduate-level course on program analysis and
transformations, which builds on the material in the compilers course.
This course also centers on a project, completed individually. I chose to
design an analysis and transformation framework for a toy imperative language.
The framework took the form of a Haskell library. I further generalized the
idea for the syntax tree representation that was used in the compiler project,
by borrowing ideas from a ICFP paper\footnotemark.
%
\footnotetext{
  Alexey Rodriguez, Stefan Holdermans, Andres L\"oh, Johan Jeuring.
  \textit{
    Generic programming with fixed points for mutually recursive datatypes.
  }
  ICFP 2009.
}

In summer 2017, I spent ten weeks in New York City as a software developer
intern for 1010data\footnotemark{}. I implemented a new strategy for logging
users into the platform which reduced the baseline login time by a factor of
twelve. Concretely, new user session processes are now created by forking an
existing user session process, in order to skip the code loading phase
associated with the traditional ``from scratch'' strategy.
%
\footnotetext{\url{http://1010data.com/}}%
%
Furthermore, I developed new infrastructure, in the form of a C extension, to
assist in interactively debugging K\footnotemark applications, such as user
sessions.
%
\footnotetext{\url{https://en.wikipedia.org/wiki/K_(programming_language)}}

\section{Other relevant information}

Two ``W'' grades appear on my transcript in the Winter 2016 term. That was
precisely the time I started work at OOHLALA Mobile, so I withdrew from
Artificial Intelligence and Discrete Structures 2 in late January in order to
balance my work schedule with my school schedule.

It is due to this part-time work during my studies that I am graduating one
semester later than I had originally anticipated.

\end{document}
