\documentclass[11pt,letterpaper]{article}

\usepackage[margin=2.0cm]{geometry}

\author{Jacob Thomas Errington}
\title{Motivation}
\date{}

\begin{document}

\maketitle

Software is ubiquitous in our world. More and more aspects of our lives are
governed by software products, and the potential for software to misbehave is a
concern that never ceases to grow.
For instance, the false alarms in the NORAD defense system in 1980 could have
resulted in major worldwide catastrophe.
So I frequently ask myself: how can we be \emph{sure} that software does
\emph{what it is supposed to do?}
This question is what drives my interest in programming languages research.

I first discovered functional programming as a teenager, as I acquainted myself
with the Haskell programming language.
It took a number of tries before the concepts finally clicked for me, as at
that time I had comparatively little programming experience and considerably
less mathematical maturity.
Since then, I enjoyed exploring the numerous language extensions that exist in
the GHC compiler.
Some of these give a semblance of dependent types to Haskell, which planted in
my mind a fascination with dependent types and correct-by-construction
programs.

In winter 2016, that fascination solidified itself as I took a course titled
``logic and computation'' by Professor Brigitte Pientka.
I gained considerable insight into the theory of programming languages thanks
to this course: we discussed the Curry-Howard correspondence for first-order
logic in quite some detail.
I have fond memories of visiting Prof.~Pientka in her office after class to
chat further about the different proofs we covered.
%
%I took a course on compiler design. It was this project that
%concretized my appreciation for strong type systems: we had many static
%guarantees about our compiler thanks to our syntax tree representation. In
%other words, many invariants about our code turned out to be expressible in the
%types. Consequently, these invariants were verified for us by GHC.

I find Prof.~Pientka's research very inspiring. In contrast with many
programming language researchers who ``throw stuff at the wall to see what
sticks,'' Prof.~Pientka arrives at novel programming language concepts by
building them from the ground up, ensuring that the theory is present to back
up these new ideas from the get-go.
For example, although it has been possible to define coinductive data types for
quite some time in advanced languages such as Coq and Agda, it was the work of
David Thibodeau and Andrew Cave under Prof.~Pientka that solidified an
understanding of coinductive definitions in terms of copattern matching, a
novel concept\footnotemark.
%
\footnotetext{citation for Indexed Codata Types at ICFP}%

This term, in fall 2017, I am working under the supervision Prof.~Pientka in
the context of an independent studies course to bridge this work on copatterns
and indexed codata types with earlier work on fair reactive
programming\footnotemark.
The goal of this project is to design a simply-typed core calculus supporting
both codata types and temporal modalities obtained as least and greatest fixed
points.
This work would naturally lend itself to be extended to indexed types or even
full dependent types. Prof.~Pientka and I believe that such an extension would
make for an excellent topic for a masters thesis.
%
\footnotetext{citation for FRP paper at POPL}%

My main research interest is in the design of programming languages.
Specifically, how can we create programming languages in which it is easy to
write code that is at once understandable and correct?
Second, I am interested in compilers and runtimes for functional languages.
Simon Peyton Jones showed in 1992 how lazy functional languages could be
efficiently implemented on stock hardware. However, the compilation story for
dependently-typed programming languages remains still to this day very
mysterious.

\end{document}
