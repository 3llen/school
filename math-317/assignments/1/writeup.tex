\documentclass[11pt,letterpaper]{article}

\author{Jacob Thomas Errington (260636023)}
\title{Assignment \#1\\Numerical analysis -- MATH 317}
\date{Monday, 2 October 2017}

\usepackage[geometry]{jakemath}

\begin{document}

\maketitle

\begin{enumerate}
  \item
    \begin{enumerate}
      \item
        \emph{How accurately do we need to know $\pi$ in order to compute
        $\sqrt{\pi}$ to four decimal places?}

        Generally, if we have a function $f : \R \to \R$ and we want to compute
        the error of $z = f(x)$ given the error $\Delta x$, we use a
        derivative.
        %
        \begin{equation*}
          \Delta z = \Delta x \cdot \deriv{x} f(x)
        \end{equation*}

        Taking $f(x) = \sqrt{x}$, we have
        %
        \begin{equation*}
          \Delta z
          = \Delta x \parens{\frac{1}{2} \cdot \frac{1}{\sqrt{x}}}
        \end{equation*}

        In this problem, we know that $\Delta z \leq 0.00001$, i.e. that we
        want no error in the first four decimal places.
        %
        We substitute and solve to find
        %
        \begin{equation*}
          \Delta x \leq 2 \sqrt{\pi} \cdot 0.00001 \approx 0.000035
        \end{equation*}

      \item
        \emph{Convert the binary number $11.101101$ to base $10$.}

        \begin{equation*}
          (11.101101)_2
          = 2^1 + 2^0 + 2^{-1} + 2^{-3} + 2^{-4} + 2^{-6}
          = \frac{237}{64}
          = 3.703125
        \end{equation*}
    \end{enumerate}

  \item
    \begin{enumerate}
      \item
        \emph{%
          Suppose we want to compute $z = e^x - e^{-x}$ in a small
          neighbourhood of $x$ near zero. What is the most accurate way to do
          this?%
        }

        If we just compute as written, we will have a catastophic cancellation:
        the result will be very close to zero, but the absolute errors on both
        summands will add to produce an error much larger than the result!

        Instead, we can replace the summing of absolute errors with a sum
        of relative errors by multiplying up and down by $e^x$ as follows.
        %
        \begin{equation*}
          z = \frac{e^{2x} - 1}{e^x}
        \end{equation*}

        Let $\Delta x$ be the error on $x$.
        Then the absolute error on $e^x$ is $e^x \Delta x$.
        So the relative error on $e^x$ is
        %
        \begin{equation*}
          \frac{e^x \Delta x)}{e^x} = \Delta x
        \end{equation*}
        %
        Next, the absolute error on $e^{-2x}$ is $2 e^{-2x} \Delta x$,
        and the relative error is hence $2 \Delta x$.

        The relative error on $z$ computed by the division is therefor
        $3 \Delta x$, as we add the relative errors of the divisor and
        dividend.
        Hence, for the absolute error on $z$, we have $\Delta z = 3 z \Delta
        x$. Since $z \approx 0$, this gives an acceptable error on $z$.

      \item
        \emph{%
          How about $z = \bracks{\sqrt{1 + x^2} - \sqrt{1 - x^2}}^{-1}$?%
        }

        We multiply by the conjugate to obtain
        %
        \begin{align*}
          \frac{1}{\sqrt{1 + x^2} - \sqrt{1 - x^2}}
          \cdot
          \frac{\sqrt{1 + x^2} + \sqrt{1 - x^2}}{
            \sqrt{1 + x^2} + \sqrt{1 - x^2}%
          }
          = \frac{\sqrt{1+x^2} + \sqrt{1 - x^2}}{2x^2}
        \end{align*}

        For the denominator, the relative error is $2 \frac{\Delta x}{x}$, i.e.
        twice the relative error of $x$.

        For the numerator, we compute the absolute error using a derivative.
        Let $y = \sqrt{1 + x^2} + \sqrt{1 - x^2}$.
        %
        \begin{align*}
          \Delta y
          &= \deriv[y]{x} \Delta x \\
          &=
          \parens{\frac{x}{\sqrt{1 + x^2}} - \frac{x}{\sqrt{1 - x^2}}}
          \Delta x
        \end{align*}

        We compute the relative error by dividing by $y$.
        %
        \begin{align*}
          \frac{\Delta y}{y}
          &= \frac{
            \parens{\frac{x}{\sqrt{1 + x^2}} - \frac{x}{\sqrt{1 - x^2}}}
          }{
            \sqrt{1 + x^2} + \sqrt{1 - x^2}
          }
          \Delta x
        \end{align*}

        Let's think about this relative error for a moment, keeping in mind
        that $x \approx 0$:
        the denominator is approximately $2$ and numerator is approximately
        zero.
        So the relative error is itself approximately zero here.

        Finally, we compute relative error on $z$ as the sum of the relative
        error of the divisor and of the dividend.
        %
        \begin{equation*}
          \frac{\Delta z}{z}
          =
          2 \Delta x
          +
          \frac{
            \parens{\frac{x}{\sqrt{1 + x^2}} - \frac{x}{\sqrt{1 - x^2}}}
          }{
            \sqrt{1 + x^2} + \sqrt{1 - x^2}
          }
          \Delta x
        \end{equation*}
        %
        but the second term is vanishing, so we can approximate to say
        %
        \begin{equation*}
          \frac{\Delta z}{z}
          \approx 2\Delta x
        \end{equation*}
        %
        so the absolute error on $z$ is $\Delta z = 2 z \Delta x$, which is
        acceptable since $z \approx 0$.
    \end{enumerate}
\end{enumerate}


\end{document}
