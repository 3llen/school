\documentclass[11pt,letterpaper]{article}

\author{Jacob Thomas Errington (260636023)}
\title{Assignment \#2\\Numerical analysis -- MATH 317}
\date{Monday, 23 October 2017}

\usepackage[geometry]{jakemath}
\usepackage{listings}

\lstset{
  basicstyle=\footnotesize\ttfamily,
}

\begin{document}

\maketitle

\section{Lagrange interpolation}

\begin{enumerate}
  \item
    We interpolate $\cot$ from the table and compute
    $\cot(0.0015) = 684.89532815$.

    We interpolate $\sin$ and $\cos$ from the table to compute
    \begin{equation*}
      \cot(0.0015)
      = \frac{\cos(0.0015)}{\sin(0.0015)}
      = \frac{0.99999871875}{0.0015000}
      = 666.666
    \end{equation*}

  \item
    We estimate the error due to division of $\cos$ by $\sin$ in our second
    calculation of $\cot$.

    In general if $z = f(x)$ then
    \begin{equation*}
      \Delta z = \Delta x \cdot \deriv{x} f(x)
    \end{equation*}

    In our case, $f = \cot$, so $\deriv{x} \cot x = \frac{-1}{\sin^2 x}$.
    The input $0.0015$ is correct up to four decimals, so
    $\Delta x \leq 0.00001$.
    Then we compute an upper bound on the error $\Delta z$ due to the division.
    \begin{equation*}
      \Delta z
      \leq 0.00001 \cdot \frac{1}{\sin^2 0.0015}
      = 4.4444
    \end{equation*}
    So the error arising from the division of $\cos$ by $\sin$ is quite large.

    This does not explain the difference in the results, however.
    We suspect that method b is more accurate for computing $\cot$.
    Our reasoning is that near $x = 0$, we know that $\sin x$ is approximated
    well by a straight line.
    Indeed, the Lagrange interpolation of $\sin$ using the given data does not
    oscillate between the data points, and produces a straight line.
    Similarly, $\cos$ appears to be well approximated by a straight line in the
    given dataset.
    On the other hand, our interpolation of $\cot$ is less reliable, since its
    behaviour as a function is more complex near $x = 0$; it deviates quite a
    lot from the truth.
\end{enumerate}

\section{Cubic splines}

See the code in appendix \ref{app:code} for the construction of the cubic
splines.

There is a problem in the code that we are unable to figure out.
Our Gaussian elimination matrix solver is correct, and our construction of the
cubic spline matrix is correct.
The constants that we set each equation equal to in the matrix appear to be
correct. However, the fitting parameters (the second derivatives) emitted by
our code differ from the true second derivatives of $\sin {\pi x}$.

Also, our third spline is bogus, and produces negative values.

\section{Lagrange interpolation}

Our data is the following.

\begin{center}
  \begin{tabular}{ccc}
    $j$ & $x_j$ & $f_j$ \\ \hline
    $1$ & $-1$ & $0$ \\
    $2$ & $-\frac{1}{2}$ & $\frac{\sqrt{2}}{2}$ \\
    $3$ & $\frac{1}{2}$ & $\frac{\sqrt{2}}{2}$ \\
    $4$ & $1$ & $0$
  \end{tabular}
\end{center}

We construct the functions $l_j(x)$.
%
\begin{align*}
  l_1(x)
  &= \frac{(x + 1)^3}{3/2} \\
  %
  l_2(x)
  &= \frac{(x + 1/2)^3}{3/4} \\
  %
  l_3(x)
  &= \frac{(x - 1/2)^3}{-3/4} \\
  %
  l_4(x)
  &= \frac{(x - 1)^3}{3/2}
\end{align*}
%
Then we define $L(x) = \sum_j f_j l_j(x)$.
Since $f_0 = 0$ and $f_4 = 0$, we only have two nonzero terms.

\newcommand{\uint}{\int_{-1}^1}
Next we compute $\int_{-1}^1 f(x) \; dx$
%
\begin{align*}
  \uint L(x) \; dx
  &= f(-1) \uint \frac{(x + 1)^3}{3/2} \; dx
  + f(-1/2) \uint \frac{(x + 1/2)^3}{3/4} \; dx
  + f(1/2) \uint \frac{(x - 1/2)^3}{-3/4} \; dx
  + f(1) \uint \frac{(x - 1^3}{3/2} \; dx
  \\
  &= f(-1) \frac{2}{3} \int_{-2}{0} u^3 \; du
  + f(-1/2) \frac{4}{3} \int_{-1/2}^{3/2} u^3 \; du
  + f(1/2) \frac{-4}{3} \int_{-3/2}^{1/2} u^3 \; du
  + f(1) \frac{2}{3} \int_{0}^{2} u^3 \; du
  \\
  &= f(-1) \frac{1}{6} u^4 \evaluated[-2]{0}
  + f(-1/2) \frac{1}{3} u^4 \evaluated[-1/2]{3/2}
  -  f(1/2) \frac{1}{3} u^4 \evaluated[-3/2]{1/2}
  + f(1) \frac{1}{6} u^4 \evaluated[0]{2}
  \\
  &= f(-1) \frac{16}{6}
  + f(-1/2) \frac{1}{3} \cdot \frac{81 - 1}{16}
  - f(1/2) \frac{1}{3} \cdot \frac{1 - 81}{16}
  + f(1) \frac{-16}{6} \\
  &= f(-1) \frac{16}{6}
  + f(-1/2) \frac{5}{3}
  + f(1/2) \frac{5}{3}
  - f(1) \frac{16}{6}
\end{align*}

\section{Numerical integration}

Note that from $0$ to $2$, the sine wave in $f$ makes one full revolution, so
its integral over that region is zero.
Therefore, to get the highest precision, we can replace the sine with a
straight line: take $x_0 = 0$ and $x_1 = 2$, and choose $c$ arbitrarily. This
gives $0 + 0 = 0$, which is in fact exact.

\appendix

\section{Code}
\label{app:code}

\lstinputlisting{a21.hs}

\end{document}
