\documentclass[11pt,letterpaper]{article}

\author{Jacob Thomas Errington (260636023)}
\title{Assignment \#2\\Numerical analysis -- MATH 317}
\date{Monday, 23 October 2017}

\usepackage[geometry]{jakemath}

\begin{document}

\maketitle

\section{Lagrange interpolation}

\begin{enumerate}
  \item
    We interpolate $\cot$ from the table and compute
    $\cot(0.0015) = 684.89532815$.

    We interpolate $\sin$ and $\cos$ from the table to compute
    \begin{equation*}
      \cot(0.0015)
      = \frac{\cos(0.0015)}{\sin(0.0015)}
      = \frac{0.99999871875}{0.0015000}
      = 666.666
    \end{equation*}

  \item
    We estimate the error due to division of $\cos$ by $\sin$ in our second
    calculation of $\cot$.

    In general if $z = f(x)$ then
    \begin{equation*}
      \Delta z = \Delta x \cdot \deriv{x} f(x)
    \end{equation*}

    In our case, $f = \cot$, so $\deriv{x} \cot x = \frac{-1}{\sin^2 x}$.
    The input $0.0015$ is correct up to four decimals, so
    $\Delta x \leq 0.00001$.
    Then we compute an upper bound on the error $\Delta z$ due to the division.
    \begin{equation*}
      \Delta z
      \leq 0.00001 \cdot \frac{1}{\sin^2 0.0015}
      = 4.4444
    \end{equation*}
    So the error arising from the division of $\cos$ by $\sin$ is quite large.

    This does not explain the difference in the results, however.
    We suspect that method b is more accurate for computing $\cot$.
    Our reasoning is that near $x = 0$, we know that $\sin x$ is approximated
    well by a straight line.
    Indeed, the Lagrange interpolation of $\sin$ using the given data does not
    oscillate between the data points, and produces a straight line.
    Similarly, $\cos$ appears to be well approximated by a straight line in the
    given dataset.
    On the other hand, our interpolation of $\cot$ is less reliable, since its
    behaviour as a function is more complex near $x = 0$; it deviates quite a
    lot from the truth.
\end{enumerate}

\end{document}
