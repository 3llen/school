\documentclass[11pt,letterpaper]{article}

\author{Jacob Thomas Errington}
\title{Assignment \#3\\Numerical analysis -- MATH 317}
\date{20 November 2017}

\usepackage[geometry]{jakemath}
\usepackage{listings}
\usepackage{pgfplots}
\usetikzlibrary{external}
\tikzexternalize[prefix=figures/]

\lstset{
  basicstyle=\footnotesize\ttfamily,
}

% \tikzset{
%   external/system call={%
%     lualatex \tikzexternalcheckshellescape -halt-on-error
%     -interaction=batchmode -jobname "\image" "\texsource"
%   }
% }

\begin{document}

\maketitle

\section{Constant angular velocity}

See appendix $\ref{app:q1-code}$ for the code.

The construction of the forward Euler method is simple so we won't discuss it
here.

We construct the centered differences scheme by taking the Taylor expansion of
$y(t + \Delta t)$ and $y(t - \Delta t)$ at $t$ and subtracting, which gives
%
\begin{equation*}
  \deriv{t} y(t_n) = \frac{y(t_n + \Delta t) - y(t_n - \Delta t)}{2 \Delta t}
\end{equation*}

From the problem, we know that $\deriv{t} y(t_n) = \omega x(t_n)$,
so we substitute and solve for $y(t_n + \Delta t)$.
%
\begin{equation*}
  y(t_n + \Delta t) = 2 \omega x(t_n) \Delta t + y(t_n - \Delta t)
\end{equation*}

A similar analysis gives us that
%
\begin{equation*}
  x(t_n + \Delta t) = -2 \omega y(t_n) \Delta t + x(t_n - \Delta t)
\end{equation*}

We can rewrite the equations into a nicer form.
%
\begin{align*}
  x_{n+1} &= - 2 \omega y_n \Delta t + x_{n-1} \\
  y_{n+1} &= 2 \omega x_n \Delta t + y_{n-1}
\end{align*}

Defining $f(t, x, y) = (-\omega y, \omega x)$ to be a vector function, we can
rewrite the family of equations above into the form of a single vector
equation.
%
\begin{equation*}
  \vec v_{n+1} = 2 \Delta t\, f(t_n, \vec v_n) + \vec v_{n-1}
\end{equation*}

The \texttt{centeredDifferenceStep} function in our code computes this equation
given the information on the right-hand side.

See figure \ref{fig:q1} for the results and discussion.

\tikzsetnextfilename{cd_ode1}
\begin{figure}[ht]
  \pgfplotstableread{output/cd_ode1.txt}{\odeonecd}
  \pgfplotstableread{output/fe_ode1.txt}{\odeonefe}
  \centering
  \begin{tikzpicture}[scale=1]
    \begin{axis}[
        minor tick num=1,
        xlabel={position (x)},
        ylabel={position (y)},
    ]
      \addplot [red] table [x={x}, y={y}] {\odeonefe};
      \addplot [blue] table [x={x}, y={y}] {\odeonecd};
    \end{axis}
  \end{tikzpicture}
  \caption{%
    The position of the particle is shown as a parametric plot in the time
    $t$.
    In red is the simulation using the forward Euler method. In blue is the
    simulation using the centered difference method.
    In forward Euler, truncation errors compound over time: physically
    speaking, this method injects energy into the system. This is evidenced
    by the particle moving further away from the center.
    In centered differences, we eliminate the first term in the Taylor
    expansions when we subtract the two equations that go into the
    construction. This boosts the accuracy of the method to $O(h^2)$ whereas
    forward Euler has accuray $O(h)$.%
  }
  \label{fig:q1}
\end{figure}

\section{Nonlinear pendulum}

See appendix $\ref{app:q2-code}$ for the code.

First we discuss the construction of the backwards difference.

We construct the first backwards difference by Taylor expanding $v(t - h)$
around $t$ to get
%
\begin{equation*}
  v(t - h) = v(t) - \deriv{t} v(t) h
\end{equation*}
%
Since $\deriv{t} v(t) = -g \sin \theta(t)$ is known, we get
%
\begin{equation}
  \label{eq:deriv-1}
  - g \sin \theta(t_n) = \frac{v_n - v_{n-1}}{h}
\end{equation}

Next we construct the second backwards difference by expanding $\theta(t - h)$
around $t$ to get
%
\begin{equation*}
  \theta(t - h) = \theta(t) - \deriv{t} \theta(t) h
\end{equation*}
%
and since $\deriv{t} \theta(t) = \frac{v(t)}{L}$ is known, we get
%
\begin{equation}
  \label{eq:deriv-2}
  v_n = \frac{L}{h}(\theta_n - \theta_{n-1})
\end{equation}

We substitute $v_n$ into \eqref{eq:deriv-1} to get
%
\begin{equation*}
  -g \sin \theta_n = \frac{\frac{L}{h}(\theta_n - \theta_{n-1}) - v_{n-1}}{h}
\end{equation*}

In this equation $\theta_{n-1}$ and $v_{n-1}$ are known and $\theta_n$ is the
only unknown. We solve for $\theta_n$ by root-finding using the Newton-Raphson
method, since the derivative is easy to compute.

In particular, we solve for the root(s) of the function
%
\begin{equation*}
  f(\theta)
  = \frac{\frac{L}{h}(\theta - \theta_{n-1}) - v_{n-1}}{h}
  + g \sin \theta
\end{equation*}
%
whose derivative is
%
\begin{equation*}
  f^\prime(\theta)
  = \frac{L}{h^2} \theta + g \cos \theta
\end{equation*}

We take $\theta_{n-1}$ as the initial guess in each case.

Once we obtain a value for $\theta_n$, we can substitute it back into
\eqref{eq:deriv-2} to compute $v_n$. This concludes the procedure for
performing one step of backwards Euler.

See figure \ref{fig:q2} for results and discussion.

\tikzsetnextfilename{ode2}
\begin{figure}[h]
  \pgfplotstableread{output/be_ode2.txt}{\odetwobe}
  \pgfplotstableread{output/fe_ode2.txt}{\odetwofe}
  \centering
  \begin{tikzpicture}[scale=1]
    \begin{axis}[
        minor tick num=1,
        xlabel={Time (s)},
        ylabel={Velocity (m/s)},
    ]
      \addplot [blue] table [x={t}, y={v}] {\odetwobe};
      \addplot [red] table [x={t}, y={v}] {\odetwofe};
    \end{axis}
  \end{tikzpicture}
  \caption{%
    The red curve shows the velocity calculated using the forward Euler method,
    and the blue curve shows the velocity calculated using the backward Euler
    method.
    Implicit schemes like backward Euler are numerically stable.
    In contrast, errors accumulate and compound over time in the forward Euler
    method, as evidenced by the growing amplitude of oscillation in the red
    curve.
  }
  \label{fig:q2}
\end{figure}

\section{Integrating ODEs}

See appendix \ref{app:q3-code} for the code.
See figure \ref{fig:q3} for the results.

\tikzsetnextfilename{ode3}
\begin{figure}
  \pgfplotstableread{output/ie_ode3.txt}{\odethreeie}
  \centering
  \begin{tikzpicture}[scale=1]
    \begin{axis}[
        minor tick num=1,
        ylabel={$z$-position},
        xlabel={$x$-position},
    ]
      \addplot [red] table [x={x}, y={z}] {\odethreeie};
    \end{axis}
  \end{tikzpicture}
  \caption{%
    The Lorenz attractor! We only plot the first $2000$ of the $20000$ data
    points, because otherwise compilation of this document takes upwards of ten
    minutes, and the resulting PDF contains so many paths that it's unviewable.
  }
\end{figure}

\section{Second-order ODE}

First we analytically solve the second-order homogeneous linear differential
equation with constant coefficients
%
\begin{equation*}
  y^{\prime\prime} + 2 y^\prime + 9 y = 0
\end{equation*}
%
by considering its characteristic equation
%
\begin{equation*}
  z^2 + 2z + 9 = 0
\end{equation*}
%
which has complex roots
%
\begin{equation*}
  z = -1 \pm 2 \sqrt 2 i
\end{equation*}

Hence the solution is of the form
%
\begin{equation*}
  y(t) = c_1 e^{-t} \cos{2\sqrt{2} t} + c_2 e^{-t} \sin{2\sqrt{2} t}
\end{equation*}

We compute the derivative and use the initial conditions to solve for $c_1$ and
$c_2$: we have two equations and two unknowns.
We find that
%
\begin{align*}
  c_1 &= 0 \\
  c_2 &= \frac{\sqrt{2}}{4}
\end{align*}
%
so the solution is
%
\begin{equation*}
  y(t) = \frac{\sqrt{2}}{4} e^{-t} \sin{2 \sqrt 2 t}
\end{equation*}

Next we want to solve this problem numerically.
We perform a change of variables to transform the problem into a system of
first-order differential equations.

Let $u_1 = y$ and $u_2 = y^\prime$, so
%
\begin{align*}
  u_2 &= u_1^\prime \\
  u_2^\prime + 2u_2 + 9 u_1 = 0
\end{align*}

Applying the backward Euler scheme gives
%
\begin{align*}
  u_1(t - h) &= u_1(t) - u_2(t) h \\
  u_2(t - h) &= u_2(t) - (- 9 u_1(t) - 2 u_2(t))h
\end{align*}
%
which we can rewrite in matrix form as
%
\begin{equation*}
  (I - h A) \vec u(t_n) = \vec u(t_{n-1})
\end{equation*}
%
where
\begin{equation*}
  A = \parens{
    \begin{array}{cc}
      0 & 1 \\
      -9 & -2
    \end{array}
  }
\end{equation*}

See figure \ref{fig:q4} for the results.

\tikzsetnextfilename{ode4}
\begin{figure}[ht]
  \pgfplotstableread{output/ode3-small.txt}{\odefoursmall}
  \pgfplotstableread{output/ode3-med.txt}{\odefourmed}
  \pgfplotstableread{output/ode3-large.txt}{\odefourlarge}
  \pgfplotstableread{output/ode3-analytical.txt}{\odefouranal}
  \centering
  \begin{tikzpicture}[scale=1]
    \begin{axis}[
        minor tick num=1,
        xlabel={$t$},
        ylabel={$y(t)$},
    ]
      \addplot [black] table [x={t}, y={u1}] {\odefouranal};
      \addplot [red] table [x={t}, y={u1}] {\odefoursmall};
      \addplot [blue] table [x={t}, y={u1}] {\odefourmed};
      \addplot [green] table [x={t}, y={u1}] {\odefourlarge};
    \end{axis}
  \end{tikzpicture}
  \caption{%
    The very oscillatory curve (black) is the analytic solution.
    The red curve computed with $h = 0.001$ is superimposed over the analytic
    solution.
    The blue curve is the second closest to the analytic solution,
    computed with $h = 0.05$.
    The green curve is the furthest from the analytic solution,
    computed with $h = 0.1$.
  }
\end{figure}

\pagebreak
\appendix

\section{Code}

\subsection{For question 1}
\label{app:q1-code}

\lstinputlisting[language=Haskell]{q1.hs}

\subsection{For question 2}
\label{app:q2-code}

\lstinputlisting[language=Haskell]{q2.hs}

\subsection{For question 3}
\label{app:q3-code}

\lstinputlisting[language=Haskell]{q3.hs}

\subsection{For question 4}
\label{app:q4-code}

\lstinputlisting[language=Haskell]{q4.hs}

\end{document}
