\documentclass[11pt,letterpaper]{article}

\author{Jacob Thomas Errington}
\title{Assignment \#3\\Numerical analysis -- MATH 317}
\date{20 November 2017}

\usepackage[geometry]{jakemath}
\usepackage{listings}

\begin{document}

\maketitle

\section{Constant angular velocity}

See appendix $\ref{app:code}$ for the code.

The construction of the forward Euler method is simple so we won't discuss it
here.

We construct the centered differences scheme by taking the Taylor expansion of
$y(t + \Delta t)$ and $y(t - \Delta t)$ at $t$ and subtracting, which gives
%
\begin{equation*}
  \deriv{t} y(t_n) = \frac{y(t_n + \Delta t) - y(t_n - \Delta t)}{2 \Delta t}
\end{equation*}

From the problem, we know that $\deriv{t} y(t_n) = \omega x(t_n)$,
so we substitute and solve for $y(t_n + \Delta t)$.
%
\begin{equation*}
  y(t_n + \Delta t) = 2 \omega x(t_n) \Delta t + y(t_n - \Delta t)
\end{equation*}

A similar analysis gives us that
%
\begin{equation*}
  x(t_n + \Delta t) = -2 \omega y(t_n) \Delta t + x(t_n - \Delta t)
\end{equation*}

We can rewrite the equations into a nicer form.
%
\begin{align*}
  x_{n+1} &= - 2 \omega y_n \Delta t + x_{n-1} \\
  y_{n+1} &= 2 \omega x_n \Delta t + y_{n-1}
\end{align*}

Defining $f(t, x, y) = (-\omega y, \omega x)$ to be a vector function, we can
rewrite the family of equations above into the form of a single vector
equation.
%
\begin{equation*}
  \vec v_{n+1} = 2 \Delta t\, f(t_n, \vec v_n) + \vec v_{n-1}
\end{equation*}

The \texttt{centeredDifferenceStep} function in our code computes this equation
given the information on the right-hand side.

\section{Nonlinear pendulum}

First we discuss the construction of the backwards difference.

We construct the first backwards difference by Taylor expanding $v(t - h)$
around $t$ to get
%
\begin{equation*}
  v(t - h) = v(t) - \deriv{t} v(t) h
\end{equation*}
%
Since $\deriv{t} v(t) = -g \sin \theta(t)$ is known, we get
%
\begin{equation}
  \label{eq:deriv-1}
  - g \sin \theta(t_n) = \frac{v_n - v_{n-1}}{h}
\end{equation}

Next we construct the second backwards difference by expanding $\theta(t - h)$
around $t$ to get
%
\begin{equation*}
  \theta(t - h) = \theta(t) - \deriv{t} \theta(t) h
\end{equation*}
%
and since $\deriv{t} \theta(t) = \frac{v(t)}{L}$ is known, we get
%
\begin{equation}
  \label{eq:deriv-2}
  v_n = \frac{L}{h}(\theta_n - \theta_{n-1})
\end{equation}

We substitute $v_n$ into \eqref{eq:deriv-1} to get
%
\begin{equation*}
  -g \sin \theta_n = \frac{\frac{L}{h}(\theta_n - \theta_{n-1}) - v_{n-1}}{h}
\end{equation*}

In this equation $\theta_{n-1}$ and $v_{n-1}$ are known and $\theta_n$ is the
only unknown. We solve for $\theta_n$ by root-finding using the Newton-Raphson
method, since the derivative is easy to compute.

In particular, we solve for the root(s) of the function
%
\begin{equation*}
  f(\theta)
  = \frac{\frac{L}{h}(\theta - \theta_{n-1}) - v_{n-1}}{h}
  + g \sin \theta
\end{equation*}
%
whose derivative is
%
\begin{equation*}
  f^\prime(\theta)
  = \frac{L}{h^2} \theta + g \cos \theta
\end{equation*}

We take $\theta_{n-1}$ as the initial guess in each case.

Once we obtain a value for $\theta_n$, we can substitute it back into
\eqref{eq:deriv-2} to compute $v_n$. This concludes the procedure for
performing one step of backwards Euler.

\section{Integrating ODEs}

\section{Second-order ODE}

\end{document}
