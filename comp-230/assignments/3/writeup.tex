\documentclass[11pt,letterpaper]{article}

\usepackage[geometry]{jakemath}

\author{Jacob Thomas Errington (260636023)}
\title{Assignment \#3\\Logic and computability -- COMP 230}
\date{24 October 2017}

\renewcommand{\implies}{\supset}

\begin{document}

\maketitle

\section{Propositional logic}

We consider the formula $\mathqd{B} \lor (\mathqd{A} \implies \mathqd{A})$.

\begin{enumerate}
    \item
      To see that this is a tautology, we draw the truth table for the formula.

      \begin{tabular}{cc|cc}
        \textqd{A} & \textqd{B} & $\mathqd{A} \implies \mathqd{A}$ & $\mathqd{B} \lor (\mathqd{A} \implies \mathqd{A})$ \\
        \hline
        $\top$ & $\top$ & $\top$ & $\top$ \\
        $\top$ & $\bot$ & $\top$ & $\top$ \\
        $\bot$ & $\top$ & $\top$ & $\top$ \\
        $\bot$ & $\bot$ & $\top$ & $\top$
      \end{tabular}

    \item
      We prove the statement using natural deduction.

      \begin{equation*}
        \infer[\lor\text{Intro}_r]{%
          \mathqd{B} \lor (\mathqd{A} \implies \mathqd{A})%
        }{
          \infer[\implies\text{Intro}_1]{
            \mathqd{A} \implies \mathqd{A}
          }{
            [\mathqd{A}]_1
          }
        }
      \end{equation*}

    \item
      We prove the statement using the axiomatic system.

      \begin{align}
        \mathqd{A} \implies \mathqd{A}
        &\quad
        \text{theorem from handout}
        \label{eq:a-a} \\
        %
        (\mathqd{A} \implies \mathqd{A})
        \implies (\mathqd{B} \lor (\mathqd{A} \implies \mathqd{A}))
        &\quad
        \text{%
          Ax9[$\mathqd{A} \implies \mathqd{A} / \mathqd{B}$,
          $\mathqd{B}/\mathqd{A}$]}
        \label{eq:a-a-b-or-a-a} \\
        %
        \mathqd{B} \lor (\mathqd{A} \implies \mathqd{A})
        &\quad
        \text{MP: \ref{eq:a-a}, \ref{eq:a-a-b-or-a-a}}
      \end{align}

    \item
      We prove the statement using the GEB system.
      Except that we don't.
\end{enumerate}

\section{Propositional logic}

\begin{description}
  \item[Exercise \#8]
    Check whether
    \begin{equation*}
      \mathqd{P}_1, \mathqd{P}_2
      \entails
      \neg(\mathqd{P}_1 \implies \mathqd{P}_2)
      \implies
      (\mathqd{P}_2 \implies \mathqd{P}_1)
    \end{equation*}

    We draw a truth table, but only considering the cases where the assumptions
    hold.

    \begin{center}
      \begin{tabular}{cc|cccc}
        $\mathqd{P}_1$ &
        $\mathqd{P}_2$ &
        $\mathqd{P}_1 \implies \mathqd{P}_2$ &
        $\neg(\mathqd{P}_1 \implies \mathqd{P}_2)$ &
        $\mathqd{P}_2 \implies \mathqd{P}_1$ &
        $\neg(\mathqd{P}_1 \implies \mathqd{P}_2)
        \implies
        (\mathqd{P}_2 \implies \mathqd{P}_1)$ \\
        \hline
        %
        $\top$ & $\top$ & $\top$ & $\bot$ & $\top$ & $\top$
      \end{tabular}
    \end{center}

  \item[Exercise \#10]
    We annotate a proof in the natural deduction calculus with the names of the
    inference rules used therein.

    \begin{equation*}
      \infer[\text{ex falso}]{
        A \lor \neg A
      }{
        \infer[\neg\text{Elim}]{\bot}{
          \infer{A \lor \neg A}{
            \infer{\neg A}{
              \infer{\bot}{
                \infer{A \lor \neg A}{
                  [A]
                }
                &
                [\neg(A \lor \neg A)]
              }
            }
          }
        }
      }
    \end{equation*}
\end{description}

\section{Soundness and completeness}

In the mathematical study of formal logic, a property that we seek for many
formal systems is that they be \emph{sound} and \emph{complete}.
These properties relate the syntax of the system with the semantics of the
system. For instance, consider the formula $A \implies A$.
Intuitively, this formula is \emph{true}: if we
assume something, then of course we can deduce that thing.
This is made precise by the notion of \emph{truth tables}.
Each connective, such as $\implies$ has a truth table that relates the truth
values (true or false) of its inputs to the truth value of the connective.
We can verify that $A \implies A$ is true by looking at the truth table for
$\implies$.
We say that $A \implies A$ follows from \emph{no assumptions} and write
$\cdot \entails A \implies A$ if, for any truth value of $A$,
it holds that $A \implies A$.

On the other hand, we can \emph{prove} the statement $A \implies A$ using an
appropriate \emph{proof system}. A proof system is a system of purely syntactic
rules for writing proofs.
We would like it that no matter what we write, as long as we follow the rules
of the system, that the statements that we end up proving are true. This
property is precisely \emph{soundness}.
A well-known proof system from the early 20\th{} century is \emph{natural
deduction}. This system is sound.

Conversely, if we know that a particular formula is true, we would also like to
be sure that there exists a proof for it. In other words, we don't want there
to be any ``unattainable'' truths, that we cannot deduce. This property is
\emph{completeness}.

\end{document}
