\documentclass[11pt,letterpaper]{article}

\usepackage[geometry]{jakemath}

\author{Jacob Thomas Errington (260636023)}
\title{Assignment \#3\\Logic and computability -- COMP 230}
\date{24 October 2017}

\renewcommand{\implies}{\supset}

\begin{document}

\maketitle

\section{Propositional logic}

We consider the formula $\mathqd{B} \lor (\mathqd{A} \implies \mathqd{A})$.

\begin{enumerate}
    \item
      To see that this is a tautology, we draw the truth table for the formula.

      \begin{tabular}{cc|cc}
        \textqd{A} & \textqd{B} & $\mathqd{A} \implies \mathqd{A}$ & $\mathqd{B} \lor (\mathqd{A} \implies \mathqd{A})$ \\
        \hline
        $\top$ & $\top$ & $\top$ & $\top$ \\
        $\top$ & $\bot$ & $\top$ & $\top$ \\
        $\bot$ & $\top$ & $\top$ & $\top$ \\
        $\bot$ & $\bot$ & $\top$ & $\top$
      \end{tabular}

    \item
      We prove the statement using natural deduction.

      \begin{equation*}
        \infer[\lor\text{Intro}_r]{%
          \mathqd{B} \lor (\mathqd{A} \implies \mathqd{A})%
        }{
          \infer[\implies\text{Intro}_1]{
            \mathqd{A} \implies \mathqd{A}
          }{
            [\mathqd{A}]_1
          }
        }
      \end{equation*}

    \item
      We prove the statement using the axiomatic system.

      \begin{align}
        \mathqd{A} \implies \mathqd{A}
        &\quad
        \text{theorem from handout}
        \label{eq:a-a} \\
        %
        (\mathqd{A} \implies \mathqd{A})
        \implies (\mathqd{B} \lor (\mathqd{A} \implies \mathqd{A}))
        &\quad
        \text{%
          Ax9[$\mathqd{A} \implies \mathqd{A} / \mathqd{B}$,
          $\mathqd{B}/\mathqd{A}$]}
        \label{eq:a-a-b-or-a-a} \\
        %
        \mathqd{B} \lor (\mathqd{A} \implies \mathqd{A})
        &\quad
        \text{MP: \ref{eq:a-a}, \ref{eq:a-a-b-or-a-a}}
      \end{align}

    \item
      We prove the statement using the GEB system.
      Except that we don't, because it's too tedious.
\end{enumerate}

\section{Propositional logic}

\begin{enumerate}
  \item
    We do exercises from handout 2.

    \begin{description}
      \item[Exercise \#8.]
        Check whether
        \begin{equation*}
          \mathqd{P}_1, \mathqd{P}_2
          \entails
          \neg(\mathqd{P}_1 \implies \mathqd{P}_2)
          \implies
          (\mathqd{P}_2 \implies \mathqd{P}_1)
        \end{equation*}

        We draw a truth table, but only considering the cases where the
        assumptions hold.

        \begin{center}
          \begin{tabular}{cc|cccc}
            $\mathqd{P}_1$ &
            $\mathqd{P}_2$ &
            $\mathqd{P}_1 \implies \mathqd{P}_2$ &
            $\neg(\mathqd{P}_1 \implies \mathqd{P}_2)$ &
            $\mathqd{P}_2 \implies \mathqd{P}_1$ &
            $\neg(\mathqd{P}_1 \implies \mathqd{P}_2)
            \implies
            (\mathqd{P}_2 \implies \mathqd{P}_1)$ \\
            \hline
            %
            $\top$ & $\top$ & $\top$ & $\bot$ & $\top$ & $\top$
          \end{tabular}
        \end{center}

      \item[Exercise \#10.]
        We annotate a proof in the natural deduction calculus with the names of
        the inference rules used therein.

        \begin{equation*}
          \infer[\text{RAA}_2]{
            A \lor \neg A
          }{
            \infer[\neg \text{E}]{
              \bot
            }{
              \infer[\lor \text{I}_r]{
                A \lor \neg A
              }{
                \infer[\neg \text{I}_1]{
                  \neg A
                }{
                  \infer[\neg \text{E}]{
                    \bot
                  }{
                    \infer[\lor \text{I}_l]{
                      A \lor \neg A
                    }{
                      [A]_1
                    }
                    &
                    [\neg (A \lor \neg A)]_2
                  }
                }
              }
              &
              [\neg(A \lor \neg A)]_2
            }
          }
        \end{equation*}

      \item[Exercise \#11.]
        We prove that the law of excluded middle proves
        \emph{reductio ad absurdum}.

        \begin{equation*}
          \infer[\lor \text{E}_1]{
            A
          }{
            A \lor \neg A
            &
            [A]_1
            &
            \infer[\text{\emph{ex falso}}]{
              A
            }{
              \infer[\implies \text{E}]{
                \bot
              }{
                \neg A \implies \bot
                &
                [\neg A]_1
              }
            }
          }
        \end{equation*}

        We encoded the notion that $\neg A \proves \bot$ as an implication,
        although the proof would look pretty much the same if we had instead
        said that we have a hypothetical derivation
        $\mathcal{D} : \neg A \proves \bot$.

      \item[Exercise \#13]
        We annotate an axiomatic proof with the substitutions used by each
        axiom schema.

        \newcommand{\ann}[1]{\quad\text{#1}}

        \begin{align*}
          A % \land \neg A \implies A
          &\ann{Ax6[$A/A, \neg A / B$]}
          \\
          A \land \neg A \implies \neg A
          &\ann{Ax7[ $A/A, \neg A / B$ ]}
          \\
          (\neg A \implies A \implies B)
          \implies
          A \land \neg A
          \implies \neg A \implies A \implies B
          &\ann{Ax2[$\neg A \implies A \implies B / B, A \land \neg A / A$]}
          \\
          A \land \neg A \implies \neg A \implies A \implies B
          &\ann{MP 3, 4}
          \\
          (A \land \neg A \implies \neg A \implies A \implies B)
          \implies
          (A \land \neg A \implies \neg A)
          \implies
          A \land \neg A
          \implies
          A
          \implies
          B
          &\ann{Ax4[$A \land \neg A / A, \neg A / B, A \implies B / C$]}
          \\
          (A \land \neg A \implies \neg A)
          \implies
          A \land \neg A
          \implies
          A \implies B
          &\ann{MP 5, 6}
          \\
          A \land \neg A \implies A \implies B
          &\ann{MP 2,7}
          \\
          (A \land \neg A \implies A \implies B)
          \implies
          (A \land \neg A \implies A)
          \implies
          A \land \neg A
          \implies B
          &\ann{Ax4[$A \land \neg A / A, A / B, B / C$]}
          \\
          (A \land \neg A \implies A)
          \implies
          A \land \neg A
          \implies
          B
          &\ann{MP 8, 9}
          \\
          A \land \neg A \implies B
          &\ann{MP 1, 10}
        \end{align*}
    \end{description}

  \item
    We do an exercise from handout 2a.

    \begin{description}
      \item[Exercise \#2.]
        \begin{description}
          \item[Infix form.] $(P_1 \implies P_2) \land \neg P_3$
          \item[Prefix form.] $\land \implies P_1 P_2 \neg P_3$
          \item[Postfix form.] $P_1 P_2 \implies P_3 \neg \land$
        \end{description}
    \end{description}
\end{enumerate}

\section{Soundness and completeness}

In the mathematical study of formal logic, a property that we seek for many
formal systems is that they be \emph{sound} and \emph{complete}.
These properties relate the syntax of the system with the semantics of the
system. For instance, consider the formula $A \implies A$.
Intuitively, this formula is \emph{true}: if we
assume something, then of course we can deduce that thing.
This is made precise by the notion of \emph{truth tables}.
Each connective, such as $\implies$ has a truth table that relates the truth
values (true or false) of its inputs to the truth value of the connective.
We can verify that $A \implies A$ is true by looking at the truth table for
$\implies$.
We say that $A \implies A$ follows from \emph{no assumptions} and write
$\cdot \entails A \implies A$ if, for any truth value of $A$,
it holds that $A \implies A$.

On the other hand, we can \emph{prove} the statement $A \implies A$ using an
appropriate \emph{proof system}. A proof system is a system of purely syntactic
rules for writing proofs.
We would like it that no matter what we write, as long as we follow the rules
of the system, that the statements that we end up proving are true. This
property is precisely \emph{soundness}.
A well-known proof system from the early 20\th{} century is \emph{natural
deduction}. This system is sound.

Conversely, if we know that a particular formula is true, we would also like to
be sure that there exists a proof for it. In other words, we don't want there
to be any ``unattainable'' truths, that we cannot deduce. This property is
\emph{completeness}.

\section{Predicate logic}

We do some exercises from handout 3.

\begin{description}
  \item[Exercise \#4.]
    We want to find two structures and interpretations (for which the domain is
    not numbers) so that $\forall x: x = 0$ is true, and two so that it is
    false.

    \begin{enumerate}
      \item
        If we pick a trivial universe of discourse $U = \{\text{Alice}\}$.
        Then the statement is true.
        The interpretation of the function symbols is fully determined by the
        fact that the universe is a singleton.

      \item
        If we pick a universe of discourse with more than one element, then the
        sentence becomes false.
    \end{enumerate}

  \item[Exercise \#5.]

\end{description}

\section{Free variables and substitution}

\begin{enumerate}
  \item
    A \emph{free variable} is a variable that does not appear under the scope
    of any \emph{binder}.
    Formally, we can define the set of free variables $V(\phi)$ in a formula
    $\phi$ inductively.
    \begin{itemize}
      \item
        If $\phi = x$ is a variable, then $V(\phi) = \setof{x}$.

      \item
        $V(\forall x: \phi) = V(\phi) \setminus \{x\}$.

      \item
        Connectives preserve or union the set of free variables.
    \end{itemize}

  \item
    A term may be substituted for a variable in a formula provided that
    \begin{itemize}
      \item the term contains no free variables that would become bound;
    \end{itemize}
\end{enumerate}

\section{Formalization in first-order logic}

We try to informally interpret two formal statements of TNT.

\begin{itemize}
  \item
    $\neg \forall c : \exists b : SS0 \cdot b = c$

    This essentially says ``odd numbers exist''.
    (It is not the case that for every number, we can find another number whose
    double gives us the first number.)

  \item
    $\forall c : \neg \exists b : 2 \cdot b = c$

    This essentially says ``division by two is impossible''.
    (For any number, you can't find a number whose double gives the first one.)
\end{itemize}

\section{A translation exercise}

We translate some English statements of number theory into the language of
first-order arithmetic.

\begin{itemize}
  \item
    ``All natural numbers are equal to 4.''

    $\forall n: x = 4$.

  \item
    ``Different natural numbers have different successors.''

    $\forall n: \forall m: \neg(n = m) \implies \neg (S n = S m)$
\end{itemize}

\section{Predicate logic}

The given deduction (purportedly) proves that ``if zero is equal to zero, then
every number is equal to zero.''

This is bogus. The given proof contains a scoping error; in particular, the
assumption $x = 0$ introduced by the implication-introduction is discharged
also by a for $\forall$-introduction.

\section{$\omega$-incomplete}

When considering a formal system, we call the set of all the theorems in the
system the \emph{theory} of the system. The theorems are exactly the strings
that can be produced by following the rules in the system. So we can think of
the theory as being the set of all valid sentences in the language of the
system.

These sentences have a lot of structure.
For example, we know that for any choice of $x$, we can show $0 + x = x$.
But when I say ``for any choice of $x$'' I'm making a metalogical claim!
Can we \emph{prove in the system} that this is indeed true?
We have as an axiom of our system that $\forall a: (a + 0) = a$, so if we can
prove commutativity, then we're done.
Alas proving commutativity requires the use of induction.
So if our system did not admit an induction rule, then we could not prove
$\forall x: 0 + x = x$, and our system would be $\omega$-incomplete.

\end{document}
