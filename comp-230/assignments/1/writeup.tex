\documentclass[11pt,letterpaper]{article}

\author{Jacob Thomas Errington (260636023)}
\title{Assignment \#1\\Logic and computability -- COMP 230}
\date{21 September 2017}

\usepackage[geometry]{jakemath}

\begin{document}

\maketitle

\section{A simple mathematical proof}

\begin{prop}
    For any natural number $a$, we have that $a^2$ is even if and only if $a$
    is even.
\end{prop}

\begin{proof}
    Take an arbitrary natural number $a$.

    First, suppose that $a^2$ is even.
    We want to show that there exists a natural number $b^\prime$ such that
    $2b^\prime = a$.
    By the definition of an even number, there exists a natural number $b$ such
    that $2b = a^2$. Dividing both sides by $a$, we deduce that
    %
    \begin{equation*}
        2 \frac{b}{a} = a
    \end{equation*}
    %
    and hence that $b^\prime = \frac{b}{a}$.
    By the definition of an even number, we conclude that $a$ is even.

    Next, suppose that $a$ is even.
    We want to show that there exists a natural number $b^\prime$ such that
    $2b^\prime = a^2$.
    By the definition of an even number, there exists a natural number $b$ such
    that $2b = a$.
    Multiplying both sides by $a$, we deduce that
    %
    \begin{equation*}
        2 b a = a^2
    \end{equation*}
    %
    so taking $b^\prime = b a$, we conclude that $a^2$ is even.
\end{proof}

\section{Recursive definitions}

\begin{description}
    \newcommand{\terms}{\mathbf{Term}}
    \newcommand{\true}{\mathtt{true}}
    \newcommand{\false}{\mathtt{false}}
    \newcommand{\Succ}{\operatorname{\mathtt{succ}}}
    \newcommand{\iszero}{\operatorname{\mathtt{iszero}}}
    \newcommand{\ifthenelse}[3]{%
        \mathtt{if}\; #1 \;\mathtt{then}\; #2 \;\mathtt{else}\; #3%
    }
    \newcommand{\bool}{\mathtt{bool}}
    \newcommand{\nat}{\mathtt{nat}}
    %
    \item[Abstract syntax trees.]
        Generally speaking, we can formulate a recursive definition for
        arbitrary trees, in which case a programming language's abstract syntax
        tree is just a special case.

        We can define the set of terms $\terms$ in our toy language as follows.
        Our language will have as values the natural numbers and booleans.

        The base clauses are as follows.
        $$
        \infer{\true \in \terms}{}
        \quad
        \infer{\false \in \terms}{}
        \quad
        \infer{0 \in \terms}{}
        $$

        The inductive clauses are as follows.
        $$
        \infer{\ifthenelse{t}{s_1}{s_2}}{t, s_1, s_2 \in \terms}
        \quad
        \infer{\Succ t \in \terms}{t \in \terms}
        \quad
        \infer{t + s \in \terms}{t, s \in \terms}
        \quad
        \infer{\iszero t \in \terms}{t \in \terms}
        $$

        The final clause is that nothing but the terms constructed in this way
        are in $\terms$.

    \item[Typing derivations.]
        Next, we can describe the typing rules for this programming language.

        The base clauses are as follows.
        %
        $$
        \infer{\true \hastype \bool}{}
        \quad
        \infer{\false \hastype \bool}{}
        \quad
        \infer{0 \hastype \nat}{}
        $$
gc
        The inductive clauses are as follows.
        %
        $$
        \infer{\Succ n \hastype \nat}{n \hastype \nat}
        \quad
        \infer{n + m \hastype \nat}{n, m \hastype \nat}
        \quad
        \infer{\ifthenelse{t}{s_1}{s_2} \hastype \alpha}{%
            t \hastype \bool
            &
            s_1, s_2 \hastype \alpha
        }
        \quad
        \infer{\iszero t \hastype \bool}{t \hastype \nat}
        $$

        This inductively defines the relation $\hastype$ (``has type''), which
        in this case is a subset of $\terms \times *$ where
        $* = \{\nat, \bool\}$ is a finite universe of types.

    \item[Operational semantics.]
        We can define the operation of an abstract machine for evaluating the
        expressions in this programming language. In particular we define a
        small-step semantics in the form of a relation
        $\evalto \subset \terms \times \terms$.

        Before talking about \emph{evaluation}, we need to establish what it
        means to be a \emph{value}, since that's when evaluation stops.gc
        An inductive definition would be necessary to define values as $\true$,
        $\false$, $0$, and $\Succ v$ if $v$ is a value.
        We skip this definition since it's uninteresting and take for granted
        that a set $V$ of values is defined.

        The base clauses implement the built-in functions
        $+$, $\iszero$, and the conditional.
        %
        $$
        \infer{0 + m \evalto m}{m \in V}
        \quad
        \infer{\Succ n + m \evalto \Succ (n + m)}{n, m \in V}
        \quad
        \infer{\iszero 0 \evalto \true}{}
        \quad
        \infer{\iszero (\Succ n) \evalto \false}{n \in V}
        $$
        %
        $$
        \infer{\ifthenelse{\true}{s_1}{s_2} \evalto s_1}{}
        \quad
        \infer{\ifthenelse{\false}{s_1}{s_2} \evalto s_2}{}
        $$
        %
        Note that these are all base clauses despite not being \emph{axioms}!
        The premises, if any, do not involve the relation $\evalto$ that we are
        defining.

        Finally, situations such as
        $$
        \Succ (\ifthenelse{\true}{0}{1})
        $$
        are dealt with by the inductive clauses.
        %
        $$
        \infer{\Succ t \to \Succ t^\prime}{t \to t^\prime}
        \quad
        \infer{\iszero t \to \iszero t^\prime}{t \to t^\prime}
        $$
        %
        $$
        \infer{
            \ifthenelse{t}{s_1}{s_2}
            \to
            \ifthenelse{t^\prime}{s_1}{s_2}
        }{
            t \to t^\prime
        }
        $$

        As a final clause, we can say that the relation
        $\to \subseteq \terms \times \terms$ is exactly given by these rules.
\end{description}

\section{Mathematical induction}

\begin{prop}
    The sum of the doubles of the first $n$ natural numbers is $n(n+1)$, i.e.
    %
    \begin{equation*}
        \sum_{i=1}^n 2i = n(n+1)
    \end{equation*}
\end{prop}

\begin{proof}
    By induction on $n$.
    %
    \begin{description}
        \item[Base case] $n=1$.

            Then $\sum_{i=1}^n 2i = 2 \cdot 1 = 1 (1 + 1)$ as required.

        \item[Step case.]

            Suppose that $\sum_{i=1}^k 2i = k(k+1)$.
            We wish to show that $\sum_{i=1}^{k+1} 2i = (k+1)((k+1)+1)$.
            Observe the following equalities.
            %
            \begin{align*}
                \sum_{i=1}^{k+1} 2i
                &= \sum_{i=1}^k 2i + 2(k+1) \\
                &= k(k+1) + 2(k + 1) \tag{by IH} \\
                &= (k+1)(k+2) \\
                &= (k+1)((k+1)+1)
            \end{align*}
    \end{description}

    Thus, we have shown by induction that $\sum_{i=1}^n 2i = n(n+1)$ for all
    $n$.
\end{proof}

\section{Mathematical induction}
%
\newcommand{\geqs}[2]{#1 \operatorname{\mathtt{geq}} #2}

\begin{enumerate}
    \item
        \begin{itemize}
            \item
                $\geqs{-}{-}$ is a valid geq-string.
            \item
                $\geqs{--}{--}$ is a valid geq-string.
            \item
                $\geqs{---}{--}$ is a valid geq-string.
            \item
                $\geqs{-}{--}$ is an invalid geq-string.
            \item
                $\geqs{-}{---}$ is an invalid geq-string.
            \item
                $\geqs{}{}$ is an invalid geq-string.
        \end{itemize}

    \item
        We prove the following property of geq-strings.

        \begin{prop}
            If $s = \geqs{x}{y}$ is a valid geq-string, then the number of hyphens
            in $x$ is greater than or equal to the number of hyphens in $y$.
        \end{prop}

        \begin{proof}
            By structural induction on $s$.

            \begin{description}
                \item[Base case] $s = \geqs{-}{-}$.
                    There is one hyphen on both sides, and one is greater than
                    or equal to one.

                \item[Step case 1.]
                    Then $s$ is obtained by inductive clause 1.
                    Thus we discover that $x = y$ and that $x = x^\prime -$.
                    We apply the induction hypothesis to the smaller string
                    $\geqs{x^\prime}{x^\prime}$ to deduce that the number of
                    hyphens in $x^\prime$ is greater than or equal to the
                    number of hyphens in $x^\prime$.
                    Adding one more hyphen to both sides preserves this
                    inequality.
                    This establishes that $s = \geqs{x^\prime -}{x^\prime -}$
                    has at least as many hyphens on the left as on the right.

                \item[Step case 2.]
                    Then $s$ is obtained by inductive clause 2.
                    Hence, it is of the form $\geqs{x^\prime -}{y}$.
                    We apply the inductive hypothesis to the smaller string
                    $\geqs{x^\prime}{y}$ to deduce that the number of hyphens
                    in $x^\prime$ is greater than or equal to the number of
                    hyphens in $y$.
                    Adding an additional hyphen on the left preserves this
                    ordering.
                    Hence $s = \geqs{x^\prime -}{y}$ has at least as many
                    hyphens on the left as on the right.
            \end{description}

            This establishes that for any well-formed geq-string, the number of
            hyphens on the left is greater than or equal to the number of
            hyphens on the right.
        \end{proof}

    \item (Reflexivity.)

        \begin{prop}
            For any natural number $a$, the geq-string formed with $a$ hyphens
            on the left and on the right is well-formed.
        \end{prop}

        \begin{proof}
            By induction on $a$.

            \begin{description}
                \item[Base case] $a = 1$.
                    Then we construct a well-formed geq-string with one hyphen
                    on both sides using the base clause: $\geqs{-}{-}$.

                \item[Step case.]
                    Suppose we have a geq-string $s = \geqs{x}{y}$ with $k$
                    hyphens on the left and on the right.
                    Then $x$ and $y$ are the same string, so we deduce $x = y$.
                    We apply inductive clause 1 to obtain the geq-string
                    $s^\prime = \geqs{x -}{x -}$ having $k+1$ hyphens on both
                    sides.
            \end{description}

            This establishes that for any choice of $a$, we can construct a
            geq-string having $a$ hyphens on both sides. Intuitively, this is
            done by applying inductive clause 1 $a-1$ times to the base clause.
        \end{proof}

    \item (Sort of completeness.)

        \begin{prop}
            For any natural numbers $a$ and $b$, if $a > b$, then the
            geq-string consisting of $a$ hyphens on the left and $b$ hyphens on
            the right is well-formed.
        \end{prop}

        \begin{proof}
            By induction on $b$.

            \begin{description}
                \item[Base case] $b = 1$.
                    We construct a well-formed geq-string for this case by
                    induction on $a$.

                    \begin{description}
                        \item[Base case] $a = 2$.
                            We apply inductive clause 2 one time to the base
                            clause to arrive at the geq-string $\geqs{--}{-}$.

                        \item[Step case.]
                            Suppose we have the well-formed geq-string
                            $\geqs{x}{-}$ with $a$ hyphens on the left and one
                            hyphen on the right.
                            We can obtain a well-formed geq-string with $a+1$
                            hyphens on the left by applying inductive clause 2.
                            In particular, we get $\geqs{x-}{-}$.
                    \end{description}

                \item[Step case.]
                    Suppose $a > b$.
                    By the induction hypothesis, we have a geq-string
                    $\geqs{x}{y}$ with $a$ hyphens on the left and $b$ hyphens
                    on the right.
                    Then by inductive clause 2 we obtain a
                    geq-string $\geqs{x-}{y-}$ with $a+1$ hyphens on the left
                    and $b+1$ hyphens on the right.
            \end{description}

            The idea of this proof is the following algorithm for constructing
            the appropriate geq-string:
            %
            \begin{itemize}
                \item
                    Start with the base clause $\geqs{-}{-}$.
                \item
                    Apply inductive clause 1 $a-b$ times so that the difference
                    between the numbers $a$ and $b$ is the same as the
                    difference between the counts of hyphens.
                \item
                    Apply inductive clause 2 $b-1$ times to grow the hyphen
                    counts to the correct numbers.
            \end{itemize}
        \end{proof}
\end{enumerate}

\section{Functions}

\begin{itemize}
    \item
        The relation $F_1$ is a total function from $A$ to $B$.
        Furthermore, it is a bijection.
    \item
        The relation $F_2$ is a total function from $A$ to $B$.
        It is neither injective nor surjective. It is the constant function
        $x \mapsto b$.
    \item
        The relation $F_3$'s domain is not $A$, so it cannot be a function from
        $A$ to $B$.
    \item
        The relation $F_4$ associates $b$ with two different values, so it is
        not a function.
    \item
        The relation $F_5$'s domain is a proper subset of $A$, so it is not a
        total function on $A$.
\end{itemize}

\section{Functions}

\begin{prop}
    The function $f : \N \to \N$ given by the mapping $x \mapsto x + 7$ is
    injective.
\end{prop}

\begin{proof}
    Suppose $f(a) = f(b)$. We want to show that $a = b$.
    Expanding the definition of $f$, we get $a + 7 = b + 7$.
    Subtracting both sides by seven, we conclude that $a = b$, as required.
\end{proof}

This function $f$ from the previous proposition is not surjective.
To see this, observe that this is no natural number $a$ such that $a + 7 = 3$.
Hence, the image of $f$ is a proper subset of $\N$.

\section{Diagonalization}

\begin{prop}
    The set of all subsets of natural numbers is not denumerable.
\end{prop}

\begin{proof}
    First, notice that a subset of $\N$ can be encoded as a function $\N \to 2$
    sending a natural to $1$ if it is in the subset and to $0$ else.
    Suppose that $\pset{\N}$ were countable.
    Let $f : \N \to (\N \to 2)$ be the bijection witnessing that $\pset{\N}$ is
    countable.

    We construct a subset $s : \N \to 2$ as follows.
    %
    \begin{equation*}
        n \mapsto 1 - f(n)(n)
    \end{equation*}
    %
    In other words, the number $n$ is included in $s$ if and only if it is
    \emph{not} included in the $n$\th subset identified by $f$.
    Consequently, this subset $s$ is distinct from each subset identified by
    $f$, meaning that it is not an element of $f$'s image.
    Hence $f$ is not surjective, which contradicts the existence of $f$ as a
    bijection.
    Therefore, the powerset of $\N$ is not denumerable.
\end{proof}

\section{Cardinal numbers}

Let $A$ be a countably infinite set.

\begin{prop}
    Suppose $a \notin A$.
    Then, the set $A \union \setof{a}$ is countably infinite.
\end{prop}

\begin{proof}
    Let $f : \N \to A$ be the bijection witnessing that $A$ is countable.
    We construct a bijection $g : \N \to A \union \setof{a}$ as follows.
    %
    \begin{align*}
        0 &\mapsto a \\
        n &\mapsto f(n-1)
    \end{align*}
    %
    It is easy to see that this function is a bijection.
\end{proof}

\begin{prop}
    Suppose $B$ is a countably infinite set, and that
    $A \intersn B = \emptyset$.
    Then the set $A \union B$ is countably infinite.
\end{prop}

\begin{proof}
    Let $f : \N \to A$ be the bijection witnessing that $A$ is countable.
    Let $g : \N \to B$ be the bijection witnessing that $B$ is countable.
    We construct a bijection $h : \N \to A \union B$ as follows.
    %
    \begin{equation*}
        n \mapsto \begin{cases}
            f(n/2) &\quad\text{if $n$ is even} \\
            g((n-1)/2 &\quad\text{if $n$ is odd}
        \end{cases}
    \end{equation*}
    %
    It is easy to see that this function is a bijection.
\end{proof}

\section{G\"odel, Escher, Bach}

\begin{enumerate}
    \item
        \emph{Metamathematics} is the study of mathematics itself.

    \item
        An \emph{object language} is a language discussed by a
        \emph{metalanguage}.
        For instance, using English to describe Mandarin would place English in
        the role of metalanguage and Mandarin in the role of object language.
        In a more formal setting, one could prove properties about the
        simply-typed lambda calculus in a proof assistant such as Coq.
        In that case, Coq is the metalanguage and the simply-typed lambda
        calculus is the object language.

    \item
        The G\"odel sentence $G$ is a \emph{number-theoretic encoding} of the
        statement ``$G$ has no proof''.
        G\"odel's theorem says that \emph{any axiomatic system of number theory
        is either incomplete or inconsistent}.
        If such a system were complete, then it would contain a proof of $G$,
        which would lead to a contradiction.
        Hence, any consistent system of number theory must be incomplete: the
        sentence $G$ would be true as it has no proof.
        Therefore, there are statements of number theory that are true but
        cannot be proven.

    \item
        \emph{Principia Mathematica} is a formalization of set theory published
        by Russell and Whitehead in the early 20\th century.
        It eliminates Russell's paradox, which concerns the consistency of
        mathematics in the presence of unrestricted set comprehension, by
        introducing a hierarchy of ``types'' to which sets belong according to
        the types of their contents.

    \item
        \begin{itemize}
            \item
                ``tridecalogism,'' meaning ``thirteen-letter word'' is
                \emph{autological}.
            \item
                ``finite'' is \emph{autological}.
            \item
                ``printed'' is \emph{autological}, provided that this document
                is read in printed form.
            \item
                ``rainbow'' is \emph{heterological}.
            \item
                ``bioluminescent'' is \emph{heterological}.
            \item
                ``vinegar'' is \emph{heterological}.
        \end{itemize}
\end{enumerate}

\end{document}
