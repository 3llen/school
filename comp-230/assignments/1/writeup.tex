\documentclass[11pt,letterpaper]{article}

\author{Jacob Thomas Errington (260636023)}
\title{Assignment \#1\\Logic and computability -- COMP 230}
\date{21 September 2017}

\usepackage[geometry]{jakemath}

\begin{document}

\maketitle

\section{A simple mathematical proof}

\begin{prop}
    For any natural number $a$, we have that $a^2$ is even if and only if $a$
    is even.
\end{prop}

\begin{proof}
    Take an arbitrary natural number $a$.

    First, suppose that $a^2$ is even.
    We want to show that there exists a natural number $b^\prime$ such that
    $2b^\prime = a$.
    By the definition of an even number, there exists a natural number $b$ such
    that $2b = a^2$. Dividing both sides by $a$, we deduce that
    %
    \begin{equation*}
        2 \frac{b}{a} = a
    \end{equation*}
    %
    and hence that $b^\prime = \frac{b}{a}$.
    By the definition of an even number, we conclude that $a$ is even.

    Next, suppose that $a$ is even.
    We want to show that there exists a natural number $b^\prime$ such that
    $2b^\prime = a^2$.
    By the definition of an even number, there exists a natural number $b$ such
    that $2b = a$.
    Multiplying both sides by $a$, we deduce that
    %
    \begin{equation*}
        2 b a = a^2
    \end{equation*}
    %
    so taking $b^\prime = b a$, we conclude that $a^2$ is even.
\end{proof}

\section{Recursive definitions}

\begin{description}
    \newcommand{\terms}{\mathbf{Term}}
    \newcommand{\true}{\mathtt{true}}
    \newcommand{\false}{\mathtt{false}}
    \newcommand{\Succ}{\operatorname{\mathtt{succ}}}
    \newcommand{\iszero}{\operatorname{\mathtt{iszero}}}
    \newcommand{\ifthenelse}[3]{%
        \mathtt{if}\; #1 \;\mathtt{then}\; #2 \;\mathtt{else}\; #3%
    }
    \newcommand{\bool}{\mathtt{bool}}
    \newcommand{\nat}{\mathtt{nat}}
    %
    \item[Abstract syntax trees.]
        Generally speaking, we can formulate a recursive definition for
        arbitrary trees, in which case a programming language's abstract syntax
        tree is just a special case.

        We can define the set of terms $\terms$ in our toy language as follows.
        Our language will have as values the natural numbers and booleans.

        The base clauses are as follows.
        $$
        \infer{\true \in \terms}{}
        \quad
        \infer{\false \in \terms}{}
        \quad
        \infer{0 \in \terms}{}
        $$

        The inductive clauses are as follows.
        $$
        \infer{\ifthenelse{t}{s_1}{s_2}}{t, s_1, s_2 \in \terms}
        \quad
        \infer{\Succ t \in \terms}{t \in \terms}
        \quad
        \infer{t + s \in \terms}{t, s \in \terms}
        \quad
        \infer{\iszero t \in \terms}{t \in \terms}
        $$

        The final clause is that nothing but the terms constructed in this way
        are in $\terms$.

    \item[Typing derivations.]
        Next, we can describe the typing rules for this programming language.

        The base clauses are as follows.
        %
        $$
        \infer{\true \hastype \bool}{}
        \quad
        \infer{\false \hastype \bool}{}
        \quad
        \infer{0 \hastype \nat}{}
        $$
gc
        The inductive clauses are as follows.
        %
        $$
        \infer{\Succ n \hastype \nat}{n \hastype \nat}
        \quad
        \infer{n + m \hastype \nat}{n, m \hastype \nat}
        \quad
        \infer{\ifthenelse{t}{s_1}{s_2} \hastype \alpha}{%
            t \hastype \bool
            &
            s_1, s_2 \hastype \alpha
        }
        \quad
        \infer{\iszero t \hastype \bool}{t \hastype \nat}
        $$

        This inductively defines the relation $\hastype$ (``has type''), which
        in this case is a subset of $\terms \times *$ where
        $* = \{\nat, \bool\}$ is a finite universe of types.

    \item[Operational semantics.]
        We can define the operation of an abstract machine for evaluating the
        expressions in this programming language. In particular we define a
        small-step semantics in the form of a relation
        $\evalto \subset \terms \times \terms$.

        Before talking about \emph{evaluation}, we need to establish what it
        means to be a \emph{value}, since that's when evaluation stops.gc
        An inductive definition would be necessary to define values as $\true$,
        $\false$, $0$, and $\Succ v$ if $v$ is a value.
        We skip this definition since it's uninteresting and take for granted
        that a set $V$ of values is defined.

        The base clauses implement the built-in functions
        $+$, $\iszero$, and the conditional.
        %
        $$
        \infer{0 + m \evalto m}{m \in V}
        \quad
        \infer{\Succ n + m \evalto \Succ (n + m)}{n, m \in V}
        \quad
        \infer{\iszero 0 \evalto \true}{}
        \quad
        \infer{\iszero (\Succ n) \evalto \false}{n \in V}
        $$
        %
        $$
        \infer{\ifthenelse{\true}{s_1}{s_2} \evalto s_1}{}
        \quad
        \infer{\ifthenelse{\false}{s_1}{s_2} \evalto s_2}{}
        $$
        %
        Note that these are all base clauses despite not being \emph{axioms}!
        The premises, if any, do not involve the relation $\evalto$ that we are
        defining.

        Finally, situations such as
        $$
        \Succ (\ifthenelse{\true}{0}{1})
        $$
        are dealt with by the inductive clauses.
        %
        $$
        \infer{\Succ t \to \Succ t^\prime}{t \to t^\prime}
        \quad
        \infer{\iszero t \to \iszero t^\prime}{t \to t^\prime}
        $$
        %
        $$
        \infer{
            \ifthenelse{t}{s_1}{s_2}
            \to
            \ifthenelse{t^\prime}{s_1}{s_2}
        }{
            t \to t^\prime
        }
        $$

        As a final clause, we can say that the relation
        $\to \subseteq \terms \times \terms$ is exactly given by these rules.
\end{description}

\section{Mathematical induction}

\begin{prop}
    The sum of the doubles of the first $n$ natural numbers is $n(n+1)$, i.e.
    %
    \begin{equation*}
        \sum_{i=1}^n 2i = n(n+1)
    \end{equation*}
\end{prop}

\begin{proof}
    By induction on $n$.
    %
    \begin{description}
        \item[Base case] $n=1$.

            Then $\sum_{i=1}^n 2i = 2 \cdot 1 = 1 (1 + 1)$ as required.

        \item[Step case.]

            Suppose that $\sum_{i=1}^k 2i = k(k+1)$.
            We wish to show that $\sum_{i=1}^{k+1} 2i = (k+1)((k+1)+1)$.
            Observe the following equalities.
            %
            \begin{align*}
                \sum_{i=1}^{k+1} 2i
                &= \sum_{i=1}^k 2i + 2(k+1) \\
                &= k(k+1) + 2(k + 1) \tag{by IH} \\
                &= (k+1)(k+2) \\
                &= (k+1)((k+1)+1)
            \end{align*}
    \end{description}

    Thus, we have shown by induction that $\sum_{i=1}^n 2i = n(n+1)$ for all
    $n$.
\end{proof}

\section{Mathematical induction}
%
\newcommand{\geqs}[2]{#1 \operatorname{\mathtt{geq}} #2}

\begin{enumerate}
    \item
        \begin{itemize}
            \item
                $\geqs{-}{-}$ is a valid geq-string.
            \item
                $\geqs{--}{--}$ is a valid geq-string.
            \item
                $\geqs{---}{--}$ is a valid geq-string.
            \item
                $\geqs{-}{--}$ is an invalid geq-string.
            \item
                $\geqs{-}{---}$ is an invalid geq-string.
            \item
                $\geqs{}{}$ is an invalid geq-string.
        \end{itemize}

    \item
        We prove the following property of geq-strings.

        \begin{prop}
            If $s = \geqs{x}{y}$ is a valid geq-string, then the number of hyphens
            in $x$ is greater than or equal to the number of hyphens in $y$.
        \end{prop}

        \begin{proof}
            By structural induction on $s$.

            \begin{description}
                \item[Base case] $s = \geqs{-}{-}$.
                    There is one hyphen on both sides, and one is greater than
                    or equal to one.

                \item[Step case 1.]
                    Then $s$ is obtained by inductive clause 1.
                    Thus we discover that $x = y$ and that $x = x^\prime -$.
                    We apply the induction hypothesis to the smaller string
                    $\geqs{x^\prime}{x^\prime}$ to deduce that the number of
                    hyphens in $x^\prime$ is greater than or equal to the
                    number of hyphens in $x^\prime$.
                    Adding one more hyphen to both sides preserves this
                    inequality.
                    This establishes that $s = \geqs{x^\prime -}{x^\prime -}$
                    has at least as many hyphens on the left as on the right.

                \item[Step case 2.]
                    Then $s$ is obtained by inductive clause 2.
                    Hence, it is of the form $\geqs{x^\prime -}{y}$.
                    We apply the inductive hypothesis to the smaller string
                    $\geqs{x^\prime}{y}$ to deduce that the number of hyphens
                    in $x^\prime$ is greater than or equal to the number of
                    hyphens in $y$.
                    Adding an additional hyphen on the left preserves this
                    ordering.
                    Hence $s = \geqs{x^\prime -}{y}$ has at least as many
                    hyphens on the left as on the right.
            \end{description}

            This establishes that for any well-formed geq-string, the number of
            hyphens on the left is greater than or equal to the number of
            hyphens on the right.
        \end{proof}

    \item (Reflexivity.)

        \begin{prop}
            For any natural number $a$, the geq-string formed with $a$ hyphens
            on the left and on the right is well-formed.
        \end{prop}

        \begin{proof}
            By induction on $a$.

            \begin{description}
                \item[Base case] $a = 1$.
                    Then we construct a well-formed geq-string with one hyphen
                    on both sides using the base clause: $\geqs{-}{-}$.

                \item[Step case.]
                    Suppose we have a geq-string $s = \geqs{x}{y}$ with $k$
                    hyphens on the left and on the right.
                    Then $x$ and $y$ are the same string, so we deduce $x = y$.
                    We apply inductive clause 1 to obtain the geq-string
                    $s^\prime = \geqs{x -}{x -}$ having $k+1$ hyphens on both
                    sides.
            \end{description}

            This establishes that for any choice of $a$, we can construct a
            geq-string having $a$ hyphens on both sides. Intuitively, this is
            done by applying inductive clause 1 $a-1$ times to the base clause.
        \end{proof}

    \item (Sort of completeness.)

        \begin{prop}
            For any natural numbers $a$ and $b$, if $a > b$, then the
            geq-string consisting of $a$ hyphens on the left and $b$ hyphens on
            the right is well-formed.
        \end{prop}

        \begin{proof}
            By induction on $b$.

            \begin{description}
                \item[Base case] $b = 1$.
                    We construct a well-formed geq-string for this case by
                    induction on $a$.

                    \begin{description}
                        \item[Base case] $a = 2$.
                            We apply inductive clause 2 one time to the base
                            clause to arrive at the geq-string $\geqs{--}{-}$.

                        \item[Step case.]
                            Suppose we have the well-formed geq-string
                            $\geqs{x}{-}$ with $a$ hyphens on the left and one
                            hyphen on the right.
                            We can obtain a well-formed geq-string with $a+1$
                            hyphens on the left by applying inductive clause 2.
                            In particular, we get $\geqs{x-}{-}$.
                    \end{description}

                \item[Step case.]
                    Suppose $a > b$.gc
                    We have $b > 0$ and we want to show that if $a > b$ then we
                    can obtain a geq-string with $a$ hyphens on the left and
                    $b$ hyphens on the right.

                    Suppose $a > b$. Since $b > 0$,gc
                    $s = \geqs{x}{y}$
                    with $a$ hyphens on the left and $b$ hyphens on the right.
                    Since $a > b$ and $b > 0$, the form of the string that we
                    have is in fact $s = \geqs{x^\prime -}{y^\prime -}$.
                    We apply the inductive hypothesis to the smaller string
                    $\geqs{x^\prime}{y^\prime}$
            \end{description}
        \end{proof}
\end{enumerate}

\end{document}
