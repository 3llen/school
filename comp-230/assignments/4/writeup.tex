\documentclass[11pt,letterpaper]{article}

\author{Jacob Thomas Errington (260636023)}
\title{Assignment \#4\\Logic and computability -- COMP 230}
\date{7 November 2017}

\usepackage[utf8]{inputenc}
\usepackage[geometry]{jakemath}

\begin{document}

\maketitle

\section{Typographical and arithmetical rules}

A typographical rule is an operation that we perform on a \emph{symbol} or
sequence thereof. For instance, a typographical rule might say \emph{if we have
any numeral, say $x$, that contains a nonzero digit, then we can obtain a
numeral of the form $x0$}.
This typographical rule doesn't say anything about the numerical \emph{value}
of the numeral: we call such a value a \emph{number}.
An arithmetical rule, in contrast, refers to a transformation of a number, and
therefore would not talk about ``adding a zero'' but would instead talk about
the \emph{interpretation} of such a typographical operation.
In this case, the interpretation of adding a zero on the end is multiplying by
ten.
So the corresponding arithmetical rule for the typographical rule we gave
earlier would be \emph{any nonzero number may be multiplied by ten}.

What Hofstadter means by ``typographical rules for manipulating numerals are
actually arithmetical rules for operating on numbers'' is that any
typographical rule can be \emph{interpreted} (in a natural way) as an
arithmetical rule operating on the \emph{number} associated with the numeral'.

\section{Gödel numbering}

\begin{enumerate}
  \item
    The Gödel number of the formula $0 = 0$ according to the GEB scheme is
    $666111666$.

  \item
    This question just involves looking up a translation, so I won't do it.

  \item
    Here is a rendition of
    $\forall a: 0 \neq S a$, i.e. ``zero is the successor of no number'',
    in TNT:
    \begin{equation*}
      626,262,636,223,362,666,111,123,262
    \end{equation*}
\end{enumerate}

\section{TNT numbers}

\begin{enumerate}
  \item
    The mapping from TNT sentences to natural numbers is a function.
    In particular it is injective, but not surjective.
    This is clear: $0$, for example, has no corresponding TNT sentence.
    However, if two TNT numbers are equal, then they must come from the same
    formula.

  \item
    The set of Gödel numbers of well-formed formulas in first-order arithmetic
    is decidable.
    Just as it is possible to parse a string in first-order arithmetic, it is
    possible parse a numeral corresponding to such a string.
    The difference is that the terminals to parse are now three-digit codons.

    Another way to view this is to transform the numeral back to FOA and parse
    that instead. If any step of the transformation fails, then the string is
    not well-formed. The steps would be to divide the numeral into groups of
    three, and map each resulting codon backwards using the table.

  \item
    The TNT numbers are recursively enumerable because they are in bijection
    with the theorems of first-order arithmetic, and the theorems of
    first-order arithmetic are recursively enumerable.

  \item
    The set of Gödel numbers of theorems of first-order arithmetic is not
    decidable.
    Again due to the bijectivity, if we could decide the set of the Gödel
    numbers, then we could decide the set of theorems, and we know that the set
    of theorems of FOA is undecidable, so this is a contradiction.
\end{enumerate}

\section{History}

\begin{enumerate}
  \item
    This question was asked in a previous assignment.

  \item
    Hilbert's Program, broadly speaking, was to ``prove all things''.
    Mathematics was to be established as a sound, complete, and decidable
    formal system.
    Concretely, any proof obtained in the system would be a true mathematical
    statement, and every true mathematical statement could be provable in the
    system.
    Furthermore, every statement of mathematics would be decidable if this
    program were successful.

    Gödel's first incompletenes theorem killed Hilbert's Program when it
    established that no system stronger than number theory could be both sound
    and complete.
\end{enumerate}

\end{document}
