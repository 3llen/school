\documentclass[11pt,letterpaper]{article}

\author{Jacob Thomas Errington (260636023)}
\title{Assignment \#2\\Logic and computability -- COMP 230}
\date{Thursday, 5 October 2017}

\usepackage[geometry]{jakemath}
\usepackage{hyperref}

\DeclareMathOperator{\isprime}{\mathqd{P}}
\DeclareMathOperator{\df}{\mathqd{DF}}
\DeclareMathOperator{\dnd}{\mathqd{DND}}
\newcommand{\hy}{\mathtt{-}}

\begin{document}

\maketitle

\section{Cardinal arithmetic}

\begin{prop}
  Suppose $(B_n)_{n\in\N}$ is a countable sequence of pairwise disjoint
  countable sets.
  Then, the union of the sequence is also countable.
\end{prop}

\begin{proof}
  We will construct an injection $g : B = \Union_{n\in\N} B_n \to \N$.
  Take arbitrary $a \in B$.
  Then it belongs to $B_i$ for exactly one $i \in \N$.
  Let $f_i : B_i \to \N$ be the bijection witnessing that $B_i$ is countable.
  Let $m = f_i(a)$.
  We define $g(a) = \pi_i^m$ where $\pi_i$ is the $i$\th{} prime number.
  (So $\pi_0 = 2$, $\pi_1$ = 3, and so on.)

  Now we claim that $g$ is injective.
  To see this, take arbitrary $g(a), g(b) \in g(B)$.
  Suppose $g(a) = g(b)$.
  We want to show that $a = b$.
  Note $g(a) = \pi_i^m$ for some $i$ and some $m$,
  and $g(b) = \pi_j^n$ for some $j$ and some $n$.
  \begin{description}
    \item[Case] $i \neq j$.
      Then we have reached a contradiction, by the definition of primality.
      Specifically, $\pi_i^x \neq \pi_j^y$ for any $i \neq j$ and $x, y > 0$.
      So this case is impossible.

    \item[Case] $i = j$.
      Then we have $\pi_i^m = \pi_i^n$.
      By the injectivity of the exponentiation, we deduce that $m = n$.
      Note that $m = f_i(a)$ and $m = n = f_i(b)$.
      So $f_i(a) = f_i(b)$.
      But $f_i$ is a bijection, so it is in particular injective.
      We deduce that $a = b$, as required.
  \end{description}

  Since there exists an injection from $B \to \N$,
  we have by definition that $|B| \leq |\N|$.
  However, $B$ must be infinite, as $B_0 \subset B$ and $B_0$ is countable,
  so $|\N| = |B_0| \leq |B| \leq |\N|$.
  We deduce that $|B| = |\N|$.
\end{proof}

\begin{rem}
  Suppose $A$ is a countably infinite set and $D \subseteq A$ is also
  countably infinite.
  Then we can't say much about the cardinality of $S = A \setminus D$.
  Of course, $|S| \leq |A|$; removing elements can't make the set bigger.
  However, take for example $A = \N$. If $D = \setof{\text{even numbers}}$,
  then $S$ is the odd numbers and is also infinite;
  but if $D = \setof{n \in \N | n > 1}$,
  then $S = \setof{0, 1}$ is finite.
\end{rem}

\section{Formal systems}

\newcommand{\vsys}{\mathbf{V}}
\newcommand{\fsys}{\mathbf{F}}
\newcommand{\tsys}{\mathbf{T}}

We define a first formal system $\vsys$, with a single axiom \textqd{P},
and a single inference rule $x \rewrite x1$.

We define a second formal system $\fsys$, whose axioms are the theorems of the
$\vsys$ system and whose inference rule is
%
\begin{equation*}
  \infer{(x \succ y)}{%
    x \hastype \fsys
    &
    y \hastype \fsys
  }
\end{equation*}

Finally, we define a third formal system $tsys$, with the following axiom
schema and inference rule.
%
\begin{equation*}
  \infer{(x \succ (y \succ x))}{%
    x \hastype \fsys
    &
    y \hastype \fsys
  }
  \quad
  \infer{(x \succ x)}{%
    x \hastype \tsys
  }
\end{equation*}

\begin{enumerate}
  \item
    Here are three theorems of the $\vsys$ system.
    \begin{itemize}
      \item
        $\mathqd{P}$
      \item
        $\mathqd{P1}$
      \item
        $\mathqd{P11}$
    \end{itemize}

  \item
    Here are three theorems of the $\fsys$ system.
    \begin{itemize}
      \item
        $\mathqd{P11111}$
      \item
        $\mathqd{P} \succ \mathqd{P1}$
      \item
        $\mathqd{P111} \succ (\mathqd{P11} \succ \mathqd{P1})$
    \end{itemize}

  \item
    Here are three theorems of the $\tsys$ system.
    \begin{itemize}
        \newcommand{\xxx}{%
          \mathqd{P111} \succ (\mathqd{P11} \succ \mathqd{P1})%
        }
        \newcommand{\yyy}{%
          (\mathqd{P} \succ \mathqd{P1})%
        }
        \newcommand{\tone}{(\xxx \succ (\yyy \succ \xxx))}
        \newcommand{\ttwo}{(\yyy \succ (\xxx \succ \yyy))}
      \item
        $\tone$
      \item
        $\ttwo$
      \item
        $(\tone \succ \ttwo)$.
    \end{itemize}

  \item
    It is hard to think of a meaningful interpretation for the $\tsys$ system.

    We can think of theorems of the $\vsys$ system as natural numbers this is
    sound and complete.
    Then the theorems of the $\fsys$ system can be interpreted as a binary
    trees of natural numbers.
    But the structure of the $\tsys$ system doesn't seem to represent anything
    particularly useful.
\end{enumerate}

\section{Formal systems}

\begin{enumerate}
    \newcommand{\ysys}{\mathbf{Y}}
  \item
    We define the formal system $\ysys$.
    \newcommand{\Lop}{\mathqd{L}}
    Its alphabet is $-$, $\Lop$.
    It has a single axiom $\Lop$.
    It has two inference rules.
    \begin{itemize}
      \item
        $\Lop y \rewrite \Lop y-$
      \item
        $x \Lop y \rewrite x-\Lop y$
    \end{itemize}

  \item
    Here are four theorems of this system.
    \begin{itemize}
      \item $\Lop$
      \item $\Lop---$
      \item $-\Lop---$
      \item $---\Lop---$
    \end{itemize}

  \item
    A decision procedure for this system counts the number of hyphens before
    the $\Lop$ and the number of hyphens after the $\Lop$. If the left number
    is less than or equal to the right number, then the string is a theorem.
\end{enumerate}

\section{Interpretations}

\begin{enumerate}
  \item
    Here are four theorems of the system.
    \begin{itemize}
      \item
        \texttt{T E T}
      \item
        \texttt{[T A T] E T}
      \item
        \texttt{[T A T] E [T O F]}
      \item
        \texttt{[T A T] E [T O [F A T]]}
    \end{itemize}

  \item
    The interpretations according to the proposed scheme are the following.
    \begin{itemize}
      \item
        $5 = 5$
      \item
        $(5 \min 5) = 5$
      \item
        $(5 \min 5) = (5 \max 3)$
      \item
        $(5 \min 5) = (5 \max (3 \min 5))$
    \end{itemize}

  \item
    Yes. This is a model.
    We won't give a formal proof.
    Intuitively to see this, we can think of each inference rule as replacing
    a five or three with an equivalent expression. So let's look at each rule.
    \begin{description}
      \item[Rule 1.] $5 = 5 \min 5$
      \item[Rule 2.] $5 = 5 \max 3$
      \item[Rule 3.] $5 = 3 \max 5$
      \item[Rule 4.] $5 = 5 \max 5$
      \item[Rule 5.] $3 = 3 \min 5$
      \item[Rule 6.] $3 = 5 \min 3$
    \end{description}

    Since these substitutions are legitimate, and indeed the axiom $5 = 5$ is
    true in the interpretation, we have that this interpretation is a model.

  \item
    You bet I can.
    We can interpret brackets as parens, \texttt{T} as \textqd{true},
    \texttt{F} as \textqd{false}, \texttt{O} as $\lor$, and \texttt{A} as
    $\land$.

    Indeed, the rules transform \textqd{true} into a tautologically true
    expression and \textqd{false} into a tautologically false expression.
\end{enumerate}

\section{Recursivity}

\begin{enumerate}
  \item
    \emph{%
      There exist formal systems whose negative space (set of non-theorems) is
      not the positive space (set of theorems) of any formal system.%
    }

    More precisely we can consider a set of theorems of a formal system $A$ and
    its complement $\neg A$, which is the set of all \emph{negations} of the
    theorems in $A$.
    By the nature of a formal system, we can recursively enumerate all its
    theorems, as a breadth-first search by applying inference rules starting
    from the axioms.
    However, the complementary set of theorems $\neg A$ may not be recursively
    enumerate.
    In fact, if the complementary set is recursively enumerable, then the set
    $A$ is decidable. To see this, consider an arbitrary well-formed string of
    the system.
    Recursively enumerate both sets; eventually the string \emph{must} show up
    in one of the enumerations, \emph{in finite time!}
    Some sets are undecidable (e.g. Turing machines that halt), so it follows
    that some formal systems must have complements that are not recursively
    enumerable.

  \item
    I would explain this to a non-technical person in a similar way via the
    above argument. Consider an infinite set $A$ of numbers that you can
    enumerate, but the enumeration doesn't have any special structure. The
    numbers that come out seem \emph{random}, in some sense. If you pick an
    arbitrary number $n$ that just so happens to be in $A$ and wait long
    enough, then it will show up in the listing. Otherwise it won't. However
    you can \emph{never be sure} that it won't show up later! However, if you
    \emph{also} can make a \emph{random listing} of the elements \emph{not in}
    $A$, then you can list both at the same time and of course the number will
    eventually crop up in one of the lists!
\end{enumerate}

\section{Formal primes}

What follows is a proof tree for $\isprime \hy\hy\hy\hy\hy$.

\begin{equation*}
  \infer{\isprime \hy\hy\hy\hy\hy}{
    \infer{\hy\hy\hy\hy\hy \df \hy\hy\hy\hy}{
      \infer{\hy\hy\hy\hy\hy \df \hy\hy\hy}{
        \infer{\hy\hy\hy\hy\hy \df \hy\hy}{
          \infer{\hy\hy \dnd \hy\hy\hy\hy\hy}{
            \infer{\hy\hy \dnd \hy\hy\hy}{
              \infer[\text{axiom}]{\hy\hy \dnd \hy}{}
            }
          }
        }
        &
        \infer{\hy\hy\hy \dnd \hy\hy\hy\hy\hy}{
          \infer[\text{axiom}]{\hy\hy\hy \dnd \hy\hy}{}
        }
      }
      &
      \infer{\hy\hy\hy\hy \dnd \hy\hy\hy\hy\hy}{
        \infer[\text{axiom}]{\hy\hy\hy\hy \dnd \hy}{}
      }
    }
  }
\end{equation*}

\section{Fancy nouns}

``The quick brown fox'' is a fancy noun because the simplest fancy noun is an
ornate noun. It's an ornate noun because it begins with an article, followed by
zero or more adjectives, followed by a noun.

\section{Propositional logic}

\begin{enumerate}
  \item
    Here is the truth table for
    $\mathqd{P}_1 \supset (\mathqd{P}_0 \supset \mathqd{P}_1)$.

    \begin{tabular}{c c | c c}
      $\mathqd{P}_0$ & $\mathqd{P}_1$ & $\mathqd{P}_0 \supset \mathqd{P}_1$ & $\mathqd{P}_1 \supset (\mathqd{P}_0 \supset \mathqd{P}_1)$ \\ \hline
      $\top$ & $\top$ & $\top$ & $\top$ \\
      $\top$ & $\bot$ & $\bot$ & $\top$ \\
      $\bot$ & $\top$ & $\top$ & $\top$ \\
      $\bot$ & $\bot$ & $\top$ & $\top$
    \end{tabular}

  \item
    We analyze some expressions.
    Note that I like the colour red.

    \begin{enumerate}
      \item
        ``The moon is made of green cheese or I don't like red''
        is \emph{false} because neither the moon is made of green cheese nor do
        I not like the colour red.

      \item
        ``If the moon is made of green cheese, then I like red''
        is \emph{true} because anything follows from a falsehood.

      \item
        ``If it is not the case that $2 + 2 = 4$, then the moon is made of
        green cheese.''
        To analyze this I consider the contrapositive
        ``If the moon is not made of green cheese, then $2 + 2 = 4$.''
        Since the moon is indeed not made of green cheese,
        and $2 + 2 = 4$, we can conclude that the hypothetical is true.
    \end{enumerate}
\end{enumerate}

\section{History}

Cantor was a German mathematician.
He invented set theory and established many fundamental results such as the
fact that the reals are more numerous than the naturals.
More generally he showed that there is an infinite hierarchy of infinite
cardinals.
This discovery was upsetting to some Christians at the time, because they
viewed it as a contradiction of the infinity of God.
As for Cantor's contemporaries, they viewed his work extremely negatively.
Many outright insulted him!
Broadly speaking, the mathematical community was very divided on the issue of
set theory.
Cantor's later years were quite tragic.
He was stricken with depression, due to the very harsh reception his work
received.
Towards the end of his life, he lived in poverty due to the First World War.

Source: \url{https://en.wikipedia.org/wiki/Georg_Cantor}

\end{document}
