\documentclass[11pt,letterpaper]{article}

\author{Jacob Thomas Errington (260636023)}
\title{Assignment \#2\\Logic and computability -- COMP 230}
\date{Thursday, 5 October 2017}

\usepackage[geometry]{jakemath}

\begin{document}

\maketitle

\section{Cardinal arithmetic}

\begin{prop}
  Suppose $(B_n)_{n\in\N}$ is a countable sequence of pairwise disjoint
  countable sets.
  Then, the union of the sequence is also countable.
\end{prop}

\begin{proof}
  We will construct an injection $g : B = \Union_{n\in\N} B_n \to \N$.
  Take arbitrary $a \in B$.
  Then it belongs to $B_i$ for exactly one $i \in \N$.
  Let $f_i : B_i \to \N$ be the bijection witnessing that $B_i$ is countable.
  Let $m = f_i(a)$.
  We define $g(a) = \pi_i^m$ where $\pi_i$ is the $i$\th{} prime number.
  (So $\pi_0 = 2$, $\pi_1$ = 3, and so on.)

  Now we claim that $g$ is injective.
  To see this, take arbitrary $g(a), g(b) \in g(B)$.
  Suppose $g(a) = g(b)$.
  We want to show that $a = b$.
  Note $g(a) = \pi_i^m$ for some $i$ and some $m$,
  and $g(b) = \pi_j^n$ for some $j$ and some $n$.
  \begin{description}
    \item[Case] $i \neq j$.
      Then we have reached a contradiction, by the definition of primality.
      Specifically, $\pi_i^x \neq \pi_j^y$ for any $i \neq j$ and $x, y > 0$.
      So this case is impossible.

    \item[Case] $i = j$.
      Then we have $\pi_i^m = \pi_i^n$.
      By the injectivity of the exponentiation, we deduce that $m = n$.
      Note that $m = f_i(a)$ and $m = n = f_i(b)$.
      So $f_i(a) = f_i(b)$.
      But $f_i$ is a bijection, so it is in particular injective.
      We deduce that $a = b$, as required.
  \end{description}

  Since there exists an injection from $B \to \N$,
  we have by definition that $|B| \leq |\N|$.
  However, $B$ must be infinite, as $B_0 \subset B$ and $B_0$ is countable,
  so $|\N| = |B_0| \leq |B| \leq |\N|$.
  We deduce that $|B| = |\N|$.
\end{proof}

\begin{rem}
  Suppose $A$ is a countably infinite set and $D \subseteq A$ is also
  countably infinite.
  Then we can't say much about the cardinality of $S = A \setminus D$.
  Of course, $|S| \leq |A|$; removing elements can't make the set bigger.
  However, take for example $A = \N$. If $D = \setof{\text{even numbers}}$,
  then $S$ is the odd numbers and is also infinite;
  but if $D = \setof{n \in \N | n > 1}$,
  then $S = \setof{0, 1}$ is finite.
\end{rem}

\section{Formal systems}

\newcommand{\vsys}{\mathbf{V}}
\newcommand{\fsys}{\mathbf{F}}
\newcommand{\tsys}{\mathbf{T}}

We define a first formal system $\vsys$, with a single axiom \textqd{P},
and a single inference rule $x \rewrite x1$.

We define a second formal system $\fsys$, whose axioms are the theorems of the
$\vsys$ system and whose inference rule is
%
\begin{equation*}
  \infer{(x \succ y)}{%
    x \hastype \fsys
    &
    y \hastype \fsys
  }
\end{equation*}

Finally, we define a third formal system $tsys$, with the following axiom
schema and inference rule.
%
\begin{equation*}
  \infer{(x \succ (y \succ x))}{%
    x \hastype \fsys
    &
    y \hastype \fsys
  }
  \quad
  \infer{(x \succ x)}{%
    x \hastype \tsys
  }
\end{equation*}

\begin{enumerate}
  \item
    Here are three theorems of the $\vsys$ system.
    \begin{itemize}
      \item
        $\mathqd{P}$
      \item
        $\mathqd{P1}$
      \item
        $\mathqd{P11}$
    \end{itemize}

  \item
    Here are three theorems of the $\fsys$ system.
    \begin{itemize}
      \item
        $\mathqd{P11111}$
      \item
        $\mathqd{P} \succ \mathqd{P1}$
      \item
        $\mathqd{P111} \succ (\mathqd{P11} \succ \mathqd{P1})$
    \end{itemize}

  \item
    Here are three theorems of the $\tsys$ system.
    \begin{itemize}
        \newcommand{\xxx}{%
          \mathqd{P111} \succ (\mathqd{P11} \succ \mathqd{P1})%
        }
        \newcommand{\yyy}{%
          (\mathqd{P} \succ \mathqd{P1})%
        }
        \newcommand{\tone}{(\xxx \succ (\yyy \succ \xxx))}
        \newcommand{\ttwo}{(\yyy \succ (\xxx \succ \yyy))}
      \item
        $\tone$
      \item
        $\ttwo$
      \item
        $(\tone \succ \ttwo)$.
    \end{itemize}

  \item
    It is hard to think of a meaningful interpretation for the $\tsys$ system.

    We can think of theorems of the $\vsys$ system as natural numbers this is
    sound and complete.
    Then the theorems of the $\fsys$ system can be interpreted as a binary
    trees of natural numbers.
    But the structure of the $\tsys$ system doesn't seem to represent anything
    particularly useful.
\end{enumerate}

\section{Formal systems}

\begin{enumerate}
    \newcommand{\ysys}{\mathbf{Y}}
  \item
    We define the formal system $\ysys$.
    \newcommand{\L}{\mathqd{L}}
    Its alphabet is $-$, $\L$.
    It has a single axiom $\L$.
    It has two inference rules.
    \begin{itemize}
      \item
        $\L y \rewrite \L y-$
      \item
        $x \L y \rewrite x-\L y$
    \end{itemize}

  \item
    Here are four theorems of this system.
    \begin{itemize}
      \item $\L$
      \item $\L---$
      \item $-\L---$
      \item $---\L---$
    \end{itemize}

  \item
    A decision procedure for this system counts the number of hyphens before
    the $\L$ and the number of hyphens after the $\L$. If the left number is
    less than or equal to the right number, then the string is a theorem.
\end{enumerate}

\section{Interpretations}

\begin{enumerate}
  \item
    Here are four theorems of the system.
    \begin{itemize}
      \item
    \end{itemize}
\end{enumerate}

\section{Recursivity}

\section{Formal primes}

\section{Fancy nouns}



\end{document}
