\documentclass[11pt,letterpaper]{article}

\usepackage[margin=2.0cm]{geometry}

\author{Jacob Thomas Errington}
\title{Personal Statement}
\date{}

\begin{document}

\maketitle

Software is ubiquitous in our world. More and more aspects of our lives are
governed by software products, and the potential for software to misbehave is a
concern that never ceases to grow.
For instance, the false alarms in the NORAD defense system in 1980 could have
resulted in major worldwide catastrophe.
So I frequently ask myself: how can we be \emph{sure} that software does
\emph{what it is supposed to do?}
This question is what drives my interest in programming languages research.

I first discovered functional programming as a teenager, as I acquainted myself
with the Haskell programming language.
Since then, I enjoyed exploring the numerous language extensions that exist in
the GHC compiler:
some give a semblance of dependent types to Haskell, which planted in
my mind a fascination for programs that are correct by construction.
Broadly speaking, a dependently-typed programming language enables the
programmer to express \emph{in types} arbitrary constraints on the
\emph{values} handled by the program at runtime.
Proofs that these constraints are satisfied are included by the programmer
in their code, and these proofs are checked by the compiler.

In winter 2016, that fascination solidified itself as I took a course titled
``logic and computation'' by Prof.~Brigitte Pientka.
I gained considerable insight into the theory of programming languages thanks
to this course: we discussed the Curry-Howard correspondence for first-order
logic, which informs the theoretical underpinnings of dependently-typed
programming.
%
In that same term, I took a course on compiler design. My teammate and I
implemented our compiler in Haskell, and used a simplified version of the ideas
in \textit{Generic programming with fixed points for mutually recursive
datatypes} by Alexey Rodriguez et al.
This afforded us a general way to include annotations on the syntax tree of the
program to compile: our types reflected precisely how the syntax tree became
further augmented with new annotations as subsequent phases of compilation were
performed.
%
For outstanding achievement in this course, Prof.~Laurie Hendren emailed me to
award me with an ``honorary A+'' grade, as McGill does not have an official A+
grade.

This term, I am working under the supervision of Prof.~Pientka in an
independent studies course to combine earlier work on codata types and on
functional reactive programming. The former gives rise to a type system capable
of reasoning about infinite structures such as streams or program traces,
whereas the latter, based on temporal logic, lets the programmer specify
properties of reactive applications such as operating systems or graphical user
interfaces.
%
For this research project, I will design a language combining elements of both
copattern matching (a novel way to describe the construction of codata) and
reactive programming. Prof.~Pientka and I believe that extending this language
to include indexed types (or perhaps even full dependent types) would make for
an appropriate masters thesis. Such an extension would in essence transform the
language into a proof assistant capable of \emph{proving} properties of
reactive systems. It would then become possible for instance to model, in the
type of a program, the number of clients in a concurrent server application,
and ensure that all their requests are eventually served, even as the number of
participants changes, simply by means of typechecking the program.

In sum, my interest in programming languages goes back a long way, to the
time that I began to learn Haskell. At first, that took the form of maintaining
a few Haskell libraries, but over time I developed a fascination for
dependently-typed programming and for the design of languages that can remove
the human element from the task of ensuring the correctness of a program.
%
I have a record of excelling in the courses that I take a special interest in,
which include very relevant courses in logic, in compiler design, and in formal
verfication.
%
I believe therefore that as a masters student, there are significant
contributions that I can make in this space.

\end{document}
