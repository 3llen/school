\documentclass[11pt,letterpaper]{article}

\usepackage[margin=1.9cm]{geometry}
\usepackage{tgtermes}
\usepackage{fancyhdr}

\pagestyle{fancy}
\fancyhf{}
\rhead{Errington, Jacob Thomas}
\lhead{MOTIVATION}
\cfoot{\thepage}

\begin{document}

Software is ubiquitous in our world. More and more aspects of our lives are
governed by software products, and the potential for software to misbehave is a
concern that never ceases to grow.
For instance, the false alarms in the NORAD defense system in 1980 could have
resulted in major worldwide catastrophe.
How can we be \emph{sure} that software does \emph{what it is supposed to do?}
This question is what drives my interest in programming languages research.

I wish to pursue graduate studies in programming languages research as it
combines the theoretical and the practical.
On the one hand, such research pushes the limits of logic: linear temporal
logic\footnotemark{} (LTL), for example, was introduced precisely for its
applications to formal verification.
Broadly speaking, LTL provides a formal language for reasoning about time.
%
\footnotetext{%
  Amir Pnueli, \emph{The temporal logic of programs}.
  Proceedings of the 18th Annual Symposium on Foundations of Computer Science
  (FOCS'77),
  pp. 46-57, ACM, 1977
}%
%
On the other hand, the languages that we design need to
be actually \emph{usable} by software engineers, who are often not logicians!
As we want to prove more complex properties about our software, the proofs
become accordinaly more difficult to write by hand.
That is why ancillary tooling such as editor support is also essential to the
more widespread adoption of newer, more advanced languages.

This area of research is therefore well suited to someone like me: I enjoy
both solving abstract mathematical problems and implementing concrete software
products.
I have achieved excellent grades in numerous theoretical courses on logic, set
theory, discrete mathematics, and computability theory while also accruing work
experience as a software engineer.

This term, I am working under the supervision of Prof.~Brigitte Pientka, who
has agreed to be my masters advisor in the coming term, on an independent
studies project. My work will build on two previous developments in
Prof.~Pientka's lab.
First, Thibodeau et~al.%
\footnote{%
  David Thibodeau, Andrew Cave, Brigitte Pientka.
  \emph{Indexed codata types},
  International Conference on Functional Programming (ICFP'16)
}
added indexed types to the earlier work of Abel et~al.%
\footnote{%
  Andreas Abel, Brigitte Pientka, David Thibodeau, Anton Setzer.
  \emph{Copatterns: programming infinite structures by observations},
  In proceedings of the 40th annual ACM SIGPLAN-SIGACT symposium on Principles
  of programming languages (POPL'13),
  pp. 27-38, ACM, 2013.
}
on \emph{copattern matching}.
This technique provides new insight into the programming of infinite
structures: rather than define an infinite value using \emph{constructors}, we
instead define it using \emph{observations}.
More informally, instead of defining how to \emph{produce} a value, we specify
how to \emph{consume} such a value.
The addition of indexed types by Thibodeau et~al. makes it possible to write
\emph{proofs} about these values.
For instance, it is possible with indexed types to reason about an infinite
stream of requests, where the lengths of these requests are statically known.
Having a sound foundation for infinite structures is essential in a world where
distributed computing is on the rise: the requests received by a server
application can be modelled by an infinite stream.
%
Second, Cave et~al. developed a sound proof system for LTL\footnotemark.
This system doubles as a programming language with first-class support for
reasoning about time.
Chiefly, the language ensures by its typing rules that programs are
\emph{causal}: the output at the current time step is independent of the
results of future computations.
%
\footnotetext{%
  Andrew Cave, Francisco Ferreira, Prakash Panangaden, and Brigitte Pientka.
  \emph{Fair reactive programming},
  In proceedings of
  41th ACM SIGPLAN-SIGACT Symposium on Principles of Programming Languages
  (POPL'14), pp. 361-372, ACM, 2014.%
}%
%
As for my work this term, I seek to combine these two developments.

Concretely, I will apply introduce copattern matching to the work of Cave et
al. and augment their type system with indexed types.
This will give rise to an indexed functional reactive programming language.
If we take the natural numbers as an index domain, then this language can be
used to express and prove stronger fairness properties than previously
possible. In the language of Cave et~al., it is possible to express the
type of a fair scheduler, which ensures that two input streams are served
infinitely often, whereas with the addition of indices, it becomes possible to
prove tight bounds on how long each stream will have to wait before being
served.

I wish to pursue my studies at McGill under the supervision of Prof.~Pientka.
She has unparallelled insights in the world of dependent types, with
publications in major conferences such as ICFP and POPL, and having designed
multiple languages similar to the one I intend to develop.
After all, she supervised the two projects my work will be based on; her advice
will be invaluable in the development of my own project.

% I find Prof.~Pientka's research very inspiring. In contrast with many
% programming language researchers who ``throw stuff at the wall to see what
% sticks,'' Prof.~Pientka arrives at novel programming language concepts by
% building them from the ground up, ensuring that the theory is present to back
% up these new ideas from the get-go.
% For example, although it has been possible to define coinductive data types for
% quite some time in advanced languages such as Coq and Agda, it was the work of
% David Thibodeau and Andrew Cave under Prof.~Pientka that solidified an
% understanding of coinductive definitions in terms of copattern matching, a
% novel concept\footnotemark.
% %
% \footnotetext{citation for Indexed Codata Types at ICFP}%
% 
% This term, in fall 2017, I am working under the supervision Prof.~Pientka in
% the context of an independent studies course to bridge this work on copatterns
% and indexed codata types with earlier work on fair reactive
% programming\footnotemark.
% The goal of this project is to design a simply-typed core calculus supporting
% both codata types and temporal modalities obtained as least and greatest fixed
% points.
% This work would naturally lend itself to be extended to indexed types or even
% full dependent types. Prof.~Pientka and I believe that such an extension would
% make for an excellent topic for a masters thesis.
% %
% \footnotetext{citation for FRP paper at POPL}%
% 
% My main research interest is in the design of programming languages.
% Specifically, how can we create programming languages in which it is easy to
% write code that is at once understandable and correct?
% Second, I am interested in compilers and runtimes for functional languages.
% Simon Peyton Jones showed in 1992 how lazy functional languages could be
% efficiently implemented on stock hardware. However, the compilation story for
% dependently-typed programming languages remains still to this day very
% mysterious.
% 
\end{document}
