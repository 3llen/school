\documentclass[12pt,letterpaper]{article}

\usepackage[margin=1.87cm]{geometry}
\usepackage{fancyhdr}
\usepackage[backend=biber]{biblatex}

\addbibresource{bib.bib}

\pagestyle{fancy}
\fancyhf{}
\rhead{Errington, Jacob Thomas}
\lhead{Outline of proposed research: \emph{Mechanize proofs now!}}
\cfoot{\thepage}
\setlength\headheight{14.5pt}

\begin{document}

Software systems are tightly integrated to our lives: software manages our
finances, drives our cars, and counts our votes.
Many applications such as these are \emph{safety-critical},
meaning that software malfunction is unacceptable due to the high associated
risks.
Merely testing the software on a large number of cases is insufficient when one
wants to be \emph{certain} that the software behaves according to its
specification.
In these cases, we should instead prefer to \emph{verify} software using formal
methods. Ensuring the reliability and trustworthiness of such software systems
is not only critical to our economy but also to essential to society at large.

The Beluga project\cite{beluga,inductive-beluga,beluga-2012} aims
to change the way we write trustworthy software, by
providing an advanced language and tools for directly representing, verifying,
and communicating formal guarantees.
Specifically, this is achieved by using \emph{dependent types}, which allow the
programmer to encode in the type of the program the specification that the
program is supposed to satisfy. Hence, the programmer can verify simply by
typechecking their code that their software satisfies its specification.
One downside to this approach to certified programming is that many proofs are
straightforward and frankly uninteresting. Writing out such proofs by hand is
not only tedious for an experienced programmer, but is also overwhelming for a
less experienced programmer.

My proposed research aims to resolve this downside, improving on the
state-of-the-art methodologies for the development of certified programs.
Specifically, I intend to design and implement a proof search engine that will
be capable of mechanically deriving many proofs. This engine will be integrated
with Beluga and become a part of its interactive program development mode. In
this mode, the programmer and the compiler are in a dialogue, in which the
compiler guides the programmer in writing out the program. The proof search
engine will be used throughout this dialogue to allow the programmer to skip
over the obvious parts of the program that can be inferred by the engine.
Thanks to this engine, the programmer can instead focus on proving the
trickier, more domain-specific properties.

% From a theoretical perspective, it impossible to derive a proof for an
% arbitrary true sentence in first-order logic.
% %Although Gödel showed that first-order logic is \emph{complete}, his proof was
% %non-constructive. Furthermore, no constructive proof can exist, as if one did,
% %then it could be used to implement a program to decide the halting problem.
% It is therefore unavoidable that my proof search engine will, for some true
% sentences as input, be unable to find a proof.
% A specific theoretical component of the proposed research is to provide a
% useful characterization of the kinds of theorems that my engine would be
% capable of proving.
% My hypothesis is that the engine will be capable of proving more kinds of
% theorems than what Schürmann's engine could, as we now have a better
% understanding of proof theory and coinduction.
% 
I will base my engine on the sequent calculus, which describes provability.
However, this calculus cannot be used directly as a proof search strategy since
it is highly non-deterministic.
To address this issue, I intend to use \emph{focusing}, which organizes the
search by first applying blocks of deterministic inferences and then generating
partial program fragments.
The programmer then resolves non-determinism by linking and composing these
fragments. This is exactly the dialogue mentioned earlier.

My goal is achievable:
Schürmann\cite{schurmann-phd} previously showed how inductive proofs
could be automated in a principled way.
However, his work had some shortcomings: at the time of his work, it was
unclear how to represent the generated proofs.
Concretely, if the programmer could prove automatically five of six
propositions, but not the sixth, then there was no way for the programmer to
write the sixth by hand while relying on the five proofs that were derived
automatically.
Furthermore, Schürmann's method only applied to induction, whereas I intend to
make my engine suitable for coinduction as well. This would allow for the
automatic proving of properties not only about finite data such as lists or
derivations, but also about infinite codata using as streams and program
traces.

The applicability of the proposed research is broad.
Although I concretely intend to develop this engine as an extension to Beluga,
the core algorithm will be independent of the Beluga system and will be
transferable to other systems.
I believe that my goal is achievable and well-suited to a masters: the research
involves both a significant theoretical component in which I will study the
kinds of theorems provable by my engine and a considerable practical component
in which I will build and test the engine on numerous theorems.
The style of interactive development that my engine will give rise to will
enable programmers to mechanize their proofs, so they can focus their energy on
their application and not on the proof details.

\pagebreak
\printbibliography

\end{document}
