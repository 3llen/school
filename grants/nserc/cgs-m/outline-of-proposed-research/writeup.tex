\documentclass[12pt,letterpaper]{article}

\usepackage[margin=1.87cm]{geometry}
\usepackage{fancyhdr}
\usepackage[backend=biber]{biblatex}

\addbibresource{bib.bib}

\pagestyle{fancy}
\fancyhf{}
\rhead{Errington, Jacob Thomas}
\lhead{Outline of proposed research: \emph{Mechanize proofs now!}}
\cfoot{\thepage}
\setlength\headheight{14.5pt}

\begin{document}

As software grows in ubiquity, understanding the properties of software
products becomes increasingly essential. For life-critical software, in which
the risks associated with software malfunction are unacceptably high, merely
testing the software on a large number of cases is insufficient. Instead, how
can we \emph{be sure} that software does \emph{what it is supposed to do?}
This question drives my interest in programming languages research.

One approach to certified programming is to use languages which by their very
nature provide correctness guarantees. Such languages leverage the Curry-Howard
correspondence: the \emph{type} of a program corresponds to a
\emph{proposition} and the program itself thus corresponds to a \emph{proof} of
that proposition.
% Hence, typechecking the program amounts to verifying the proof.
%
The upshot is that instead of proving a property about \emph{one program}, we
can instead prove properties about a whole \emph{language} and deduce as
corollaries the properties of the programs in that language.
%
In fact, many programming languages can be modelled using higher-order
abstract syntax\cite{hoas} as a signature in the Logical Framework LF\cite{lf}:
Pientka demonstrated\cite{bp-proof-pearl} the convenience of this encoding by
applying it in the Twelf\cite{twelf} system to investigate a problem from the
POPLMark challenge\cite{poplmark}.
Pientka, Cave, et al. later developed the
Beluga\cite{beluga,subst-beluga,inductive-beluga} system as a framework for
reasoning about programming languages.
In Beluga and similar proof frameworks, one models another language and then
proves properties about it by writing recursive programs.
The chief innovation of Beluga however is its first-class support for contexts
and substitutions, which significantly reduces the number of (uninteresting)
auxiliary proofs that must be written.

This proposal aims to improve on the state-of-the-art methodologies for the
development of (certified) programming languages.
My specific objective is to build a proof search engine capable of
automatically producing various meta-theoretical proofs.
By delegating work to the engine, a language researcher can instead focus on
the trickier proofs that cannot be proved automatically.
%I will implement this engine as an extension to Beluga, taking advantage of its
%first-class support for contexts and substitutions.
%We envision future Beluga users as needing only to write down the proposition
%in first-order logic to prove as a type in Beluga, and obtaining thanks to our
%engine a program of that type, i.e. a proof of the proposition.

Previously, Schürmann\cite{schurmann-phd} showed how inductive proofs
could be automated in a principled way.
However, his work had some shortcomings: at the time of his work, it was
unclear how to represent the generated proofs.
Consequently his engine did not produce reusable proof terms, nor could it take
in hand-written proofs as hints.
%Concretely, if a user of this system was able to prove 5 lemmas automatically,
%but unable to automatically prove a sixth, then there was no way for the user
%to write that proof by hand while relying on the lemmas automatically proven.
A specific practical component of the proposed research is to address
these concerns by using Beluga as the language for representing proofs: I will
implement my engine as a plug-in for Beluga, and I will extend Schürmann's
work to take advantage of Beluga's special features.

From a theoretical perspective, it impossible to derive a proof for an
arbitrary true sentence in first-order logic.
%Although Gödel showed that first-order logic is \emph{complete}, his proof was
%non-constructive. Furthermore, no constructive proof can exist, as if one did,
%then it could be used to implement a program to decide the halting problem.
It is therefore unavoidable that my proof search engine will, for some true
sentences as input, be unable to find a proof.
A specific theoretical component of the proposed research is to provide a
useful characterization of the kinds of theorems that my engine would be
capable of proving.
My hypothesis is that my engine will be capable of proving more kinds of
theorems than what Schürmann's engine could, as we now have a better
understanding of proofs theory.

I will base my engine on the sequent calculus, which describes provability.
However, this calculus cannot be used directly as a proof search strategy since
it is highly non-deterministic.
Baelde et al. ran into the same problem in the development of the Tac theorem
prover\cite{tac}. To address this issue, I intend to use the same strategy they
did, called \emph{focusing}, which organizes the rules of the sequent calculus
into two phases. This eliminates some of the non-determinism.

The applicability of the proposed research is broad.
Although I concretely intend to develop this engine as an extension to Beluga,
the core algorithm will be independent of the Beluga system and will be
transferable to other systems.
The engine will facilitate proving properties of languages, and as such
will facilitate the development of new languages.
In life-critical applications, code correctness is essential, as software
malfunction cannot be tolerated.
Ultimately, concepts from research languages trickle down into mainstream
languages, and these concepts will enable software engineers to write correct
code with confidence.

\pagebreak
\printbibliography

\end{document}
