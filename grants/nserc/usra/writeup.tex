\documentclass[11pt,letterpaper]{article}

\usepackage[margin=1.87cm]{geometry}
\usepackage{fancyhdr}

\pagestyle{fancy}
\fancyhf{}
\rhead{Errington, Jacob Thomas}
\lhead{Letter of intent: Mechanize proofs now!}
\cfoot{\thepage}
\setlength\headheight{14pt}

\begin{document}

Software is ubiquitous in our world. More and more aspects of our lives are
governed by software products, and the risks associated with software
malfunction are evermore expanding. How can we \emph{be sure} that software
does \emph{what it is supposed to do?}
This question drives my interest in programming languages, which are the
primary vehicle of software engineering.

My proposed supervisor, Prof.~Brigitte Pientka, is an expert on programming
languages, with publications in major conferences such as POPL and ICFP.
This semester, I have worked under Prof.~Pientka on an independent studies
project involving an application of linear temporal logic.
This work is fascinating to me, and has given me a lot of insight into the
world of type theory. Specifically, it has made me aware of some of the major
pain points in using proof assistants: a user of such an assistant is obligated
to fill in \emph{every} detail of a proof, even when these details would be
obvious to a human reader.

In the context of the USRA project, I would intend to resolve this by
implementing a proof search engine, to be added to the Beluga system developed
by Prof.~Pientka and collaborators.
This engine would be used to construct parts of proofs that are in fact obvious
but tedious to write out manually. The upshot is that the development of
sophisticated proofs about programming languages would be no longer cluttered
by as many smaller auxiliary lemmas, which ordinarily would be written by hand.
This project involves also a theoretical component: I want to find a useful
characterization of the kinds of theorems that my engine will be able to prove,
and I would like for the engine to be grounded in proof theory rather than
simply basic on heuristics.

I believe that working on this project in a full-time capacity in the winter
would give me an excellent feel for what graduate studies would resemble.
I would be encouraged by the USRA to pursue graduate studies; Prof.~Pientka and
I have discussed the possibility, and I would ideally begin my graduate studies
in the summer term.
I also believe that this winter research opportunity would help me to develop
useful research skills: Prof.~Pientka suggested that the work I did this fall
could be made into a paper, and we believe that the work on proof search could
also lead to a publication.
Should I receive the USRA, my primary focus would be the proof search project
and hopefully turn that into a publication, but a secondary focus would be to
polish my current work on linear temporal logic to ideally have two
publications pending before my masters would begin.

In sum, I believe that the USRA would be an excellent way to prepare me for
graduate studies. By enabling me to resolve a major problem in proof
assistants, I can make an impact in the programming languages research
community, and by enabling me to polish the project I have worked on this term
into a publication. Prof.~Pientka is the perfect fit for my background:
computational logic matches my desire to both build concrete software products
and solve abstract mathematical problems.

\end{document}
