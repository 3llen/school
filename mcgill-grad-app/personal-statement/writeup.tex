\documentclass[11pt,letterpaper]{article}

\usepackage[margin=2.0cm]{geometry}

\author{Jacob Thomas Errington}
\title{Motivation}
\date{}

\begin{document}

\maketitle

Software is ubiquitous in our world. More and more aspects of our lives are
governed by software products, and the potential for software to misbehave is a
concern that never ceases to grow.
For instance, the false alarms in the NORAD defense system in 1980 could have
resulted in major worldwide catastrophe.
So I frequently ask myself: how can we be \emph{sure} that software does
\emph{what it is supposed to do?}
This question is what drives my interest in programming languages research.

I first discovered functional programming as a teenager, as I acquainted myself
with the Haskell programming language.
Since then, I enjoyed exploring the numerous language extensions that exist in
the GHC compiler:
some give a semblance of dependent types to Haskell, which planted in
my mind a fascination for programs that are correct by construction.
Broadly speaking, a dependently-typed programming language enables the
programmer to express \emph{in types} arbitrary constraints on the
\emph{values} handled by the program at runtime.
Proofs that these constraints are satisfied are included by the programmer
in their code, and these proofs are checked by the compiler.

In winter 2016, that fascination solidified itself as I took a course titled
``logic and computation'' by Prof.~Brigitte Pientka.
I gained considerable insight into the theory of programming languages thanks
to this course: we discussed the Curry-Howard correspondence for first-order
logic, which informs the theoretical underpinnings of dependently-typed
programming.
%
In that same term, I took a course on compiler design. My teammate and I
implemented our compiler in Haskell, and used a simplified version of the ideas
in \textit{Generic programming with fixed points for mutually recursive
datatypes} by Alexey Rodriguez et al.
This afforded us a general way to include annotations on the syntax tree of the
program to compile: our types reflected precisely how the syntax tree became
further augmented with new annotations as subsequent phases of compilation were
performed.
%
For outstanding achievement in this course, Prof.~Laurie Hendren emailed me to
award me with an ``honorary A+'' grade, as McGill does not have an official A+
grade.

This term, I am working under the supervision of Prof.~Pientka in an
independent studies course to combine earlier work on codata types, which give
rise to a type system capable of reasoning about infinite structures such as
streams or program traces, and functional reactive programming, which 




and indexed codata types with earlier work on fair reactive
programming\footnotemark.
The goal of this project is to design a simply-typed core calculus supporting
both codata types and temporal modalities obtained as least and greatest fixed
points.
This work would naturally lend itself to be extended to indexed types or even
full dependent types. Prof.~Pientka and I believe that such an extension would
make for an excellent topic for a masters thesis.
%
\footnotetext{citation for FRP paper at POPL}%

My main research interest is in the design of programming languages.
Specifically, how can we create programming languages in which it is easy to
write code that is at once understandable and correct?
Second, I am interested in compilers and runtimes for functional languages.
Simon Peyton Jones showed in 1992 how lazy functional languages could be
efficiently implemented on stock hardware. However, the compilation story for
dependently-typed programming languages remains still to this day very
mysterious.

\end{document}
