\documentclass[8pt,letterpaper]{article}

\title{COMP 396 Application}
\author{Professor -- Brigitte Pientka\\Student -- Jacob Thomas Errington}
\date{14 September 2017}

\usepackage[margin=2cm]{geometry}
\usepackage{hyperref}

\begin{document}

\maketitle

\begin{description}
    \item[Supervisor's Name:] Brigitte Pientka
    \item[Supervisor's Email:] \texttt{bpientka@cs.mcgill.ca}
    \item[Supervisor's Phone:] $514-398-2583$
    \item[Supervisor's Website:] \url{http://www.cs.mcgill.ca/~bpientka/}
    \item[Supervisor's department:] Computer Science
    \item[Supervisor's department (if none of the above):]
    \item[Course number:] COMP 396 (Computer Science)
    \item[Term:] Fall 2017
    \item[Project start date:] Tuesday, September 5, 2017
    \item[Project end date:] Thursday, December 7, 2017
    \item[Project title:] Reactive Programming based on Copatterns
    \item[Project description (50-100 words suggested):]

        Reactive programming seeks to model systems which react and respond to
        input such as games, print and web servers, or user interfaces.
        Functional reactive programming (FRP) was introduced by Elliott and
        Hudak  to raise the level of abstraction for writing reactive programs,
        particularly emphasizing higher-order functions. Today FRP has several
        implementations, however many of them allow one to write programs where
        present output depends on future input and suffer space leaks. This
        project provides a prototype implementation for a reactive programming
        language based on linear temporal logic by exploiting the Curry-Howard
        isomorphism as described by Cave et. al. [2013] on the one hand and the
        idea of copattern matching to safely work with infinite sequence of
        events as proposed by Abel et. al. This will provide a programming
        environment that supports fair and safe reactive programming preventing
        space leaks and non-causal functions.

    \item[Prerequisite:] 1 term completed at McGill + CGPA of 3.0 or higher; or
        permission of instructor.
    \item[Grading scheme:]
        Milestone 25\%; Presentation in research group 25\%; Final report 50\%
    \item[Other project information:]
    \item[Project status:] This project is taken. The professor has no more
        '396' projects this term.
    \item[How students can apply / Next steps:] N/A; this project is filled.
    \item[Ethics, safety, and training:] Supervisors are responsible for the
        ethics and safety compliance of undergraduate students.
    \item[This project involves:] None of the above (NEITHER animal subjects,
        nor human subjects, nor biohazardous substances, nor radioactive
        materials, nor handling chemicals, nor using lasers)
\end{description}

\newpage

\begin{description}
    \item[Student's Name:] Jacob Thomas Errington
    \item[Student's McGill ID:] 260636023
    \item[Student's Email:] jacob.errington@mail.mcgill.ca
    \item[Student's Phone:] $514-503-3100$
    \item[Student's Program:] Major Math \& Computer Science
    \item[Student's Level (U0 / U1 / U2 / U3):] U3
\end{description}

I, Jacob Thomas Errington, certify that this course is with a different
supervisor and on a different topic than any previous '396' course I have
taken. I have not applied for another '396' course in this term:

~\\

I, Brigitte Pientka, give my permission for the student identified
above to register for this project under my supervision:

~\\

Date:

~\\
~\\

Unit chair/director/designate's name:

~\\

I, the unit chair/director/designate, certify that this project conforms to
departmental requirements for 396 courses:

~\\

Date:

\end{document}
