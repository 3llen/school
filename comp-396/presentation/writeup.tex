\documentclass[serif,mathserif]{beamer}

\title[Chrona]{Chrona}
\subtitle{%
  A copattern treatment of time
}
\author{Jacob Thomas Errington}
\date{30 November 2017}

\usepackage{jakemath}

\begin{document}

\frame{\titlepage}

\begin{frame}
  \frametitle{Motivation}
  \begin{itemize}
    \item Many systems are reactive e.g. OS, GUI
    \item Copatterns complete the data/codata duality
\end{frame}

\begin{frame}
  \frametitle{Timeline}

  \begin{description}
    \item[1977:]
      A.~Pnueli introduces LTL for verification
    \item[1987:]
      T.~Hagino develops codatatypes based on coinduction
    \item[1996:]
      R.~Davies applies LTL to binding-time analysis
    \item[1997:]
      P.~Hudak and C.~Elliott continuous-time FRP
    \item[2012:]
      A.~Jeffrey observes $\Gamma \proves_\text{LTL} \text{FRP}$
    \item[2013:]
      A.~Abel, B.~Pientka et al. introduce copatterns
    \item[2014:]
      A.~Cave, F.~Ferreira et al. develop a language with LTL as a type system
  \end{description}
\end{frame}

\end{document}
