
\documentclass{beamer}
\title{The future is now!}
\subtitle{A copattern treatment of linear temporal logic}
\author{Jacob Thomas Errington}
\institute{McGill University}
\date{30 November 2017}

\usepackage{jakemath}
\usepackage{listings}
\usepackage{graphicx}

\lstset{
  basicstyle=\ttfamily
}

\usefonttheme{serif}

\newcommand{\goal}{\mathbin{>}}
\renewcommand{\synth}{\mathbin{\searrow}}
\newcommand{\pip}{\mathbin{|}}
\newcommand{\pair}[2]{\mathtt{(}#1\mathtt{,}#2\mathtt{)}}
\newcommand{\obs}[1]{\mathtt{.#1}}
\newcommand{\fun}{\mathtt{fun}\;}
\newcommand{\fto}{\mathbin{\Rightarrow}}
\newcommand{\code}[1]{\;\text{\texttt{#1}}\;}
\renewcommand{\circle}{\bigcirc}
\newcommand{\tnext}{\text{\bullet}}
\newcommand{\hole}{\mathbin{?}}

\newcommand{\unk}{\text{?}}
\newcommand{\com}[1]{\code{(*}#1\code{*)}}

\begin{document}

\frame{\titlepage}

\begin{frame}
  \frametitle{Motivation}

  \begin{itemize}
    \item Want to program reactive systems.
      \pause
    \item Callbacks and state are horrifying.
      \pause
    \item FRP solves this \pause but at the cost of space leaks.
      \pause
    \item Existing leak-proof systems are cumbersome.
  \end{itemize}
\end{frame}

\begin{frame}
  \frametitle{Concrete problem}

  \only<1>{
    \centering
    POPL'14:
    LTL as a type system for FRP.
  }

  \only<2-4>{
    \begin{equation*}
      \infer{
        \alert<3>{\Theta;\Gamma} \proves \tnext M \hastype \circle A
      }{
        \alert<4>{\cdot;\Theta} \proves M \hastype A
      }
    \end{equation*}
  }

  \only<5-6>{
    \begin{equation*}
      \infer{
        \Theta;\Gamma \proves
        \code{let} \bullet x = M \code{in} N \hastype C
      }{
        \Theta;\Gamma \proves M \hastype \circle A
        &
        \alert<6>{\Theta, x \hastype A}; \Gamma \proves N \hastype C
      }
    \end{equation*}
  }

  \only<7->{
    \begin{equation*}
      \code{codata} \square A
      = \{ \code{.head} \hastype A;\; \code{.tail} \hastype \circle \square A \}
    \end{equation*}

    \visible<8->{
      \begin{align*}
        \code{map} & \hastype \square (A \to B) \to \square A \to \square B \\
        \visible<9->{
          \code{map}\; fs\; xs &= \{
            \code{.head} =
            \only<9-10>{\alert<10>{\hole}}
            \only<11->{fs\code{.head} \; xs\code{.head}}
            ;\; \code{.tail} =
            \only<-12>{\alert<12>{\hole}}
            \only<13->{z}
          \}
        } \\
        \visible<13->{
          z &=
          \only<13>{\alert{?}}
          \only<14->{
            \code{let} \bullet fs^\prime = fs\code{.tail} \code{in} \\
            &\hphantom{=\ \:}
            \code{let} \bullet xs^\prime = xs\code{.tail} \code{in} \\
            &\hphantom{=\ \: \;}
            \only<14-15>{\alert<15>{?}}
            \only<16->{
              \bullet
              \only<16>{\alert{?}}
              \only<17->{(\code{map} fs^\prime\; xs^\prime)}
            }
          }
        }
      \end{align*}
    }
  }
\end{frame}

\begin{frame}
  \centering
  {\Huge \alert{Key idea:}\\ the \texttt{.next} observation}
\end{frame}

\begin{frame}
  \centering
  {\Huge The \texttt{.next} observation \alert{changes time.}}
\end{frame}

\begin{frame}
  \frametitle{Eliminating the future: \texttt{.next}}

  \visible<1->{
    \begin{equation*}
      \boxed{
        \Gamma \proves^{\visible<2->{\alert<2>{n}}} M \hastype T
      }
    \end{equation*}
    ``$M$ has type $T$ in context $\Gamma$\only<1>{.}%
    \only<2->{\alert<2>{ at time $n$.}}''
  }

  \visible<3->{
    \begin{equation*}
      \infer{
        \Gamma \proves^{\alert<4>{n+1}} M\mathtt{.next} \hastype A
      }{
        \Gamma \proves^{\alert<4>{n}} M \hastype \circle A
      }
    \end{equation*}
  }

  % \visible<2->{
  %   \begin{equation*}
  %     \boxed{
  %       \Gamma
  %       \goal S
  %       \proves^{\visible<4->{\alert<4>{n}}} q
  %       \synth_{\visible<4->{\alert<4>{k}}} T
  %     }
  %   \end{equation*}
  %   %
  %   ``Check copattern $q$ \only<4->{\alert{at time $n$}}
  %   with goal $S$ synthesizing type $T$
  %   \only<4->{\alert{at (future) time $k$}}
  %   and bindings $\Gamma$.''
  % }
\end{frame}

\begin{frame}
  \frametitle{\sout{Be} \alert{observe} the change you wish to see in the world}

  \only<1>{
    \begin{equation*}
      \boxed{
        \Gamma
        \goal S
        \proves q
        \synth T
      }
    \end{equation*}

   ``Check copattern $q$ with goal $S$ synthesizing type $T$ and bindings
   $\Gamma$.''
  }

  \only<2>{
    \includegraphics[width=\textwidth]{doc-brown.jpg}
  }

  \only<3>{
    \begin{equation*}
      \boxed{
        \Gamma
        \goal S
        \proves^{\alert{n}} q
        \synth_{\alert{k}} T
      }
    \end{equation*}

   ``Check copattern $q$ \alert{at time $n$}
   with goal $S$ synthesizing type $T$
   \alert{at (future) time $k$}
   and bindings $\Gamma$.''
  }
\end{frame}

\begin{frame}
  \frametitle{Going to the future: now with 50\% less plutonium!}

  \begin{equation*}
    \infer{
      \Gamma
      \goal \nu X. R
      \proves^n \mathtt{.d}_i\; q
      \synth_k T
    }{
      \Gamma
      \goal T_i
      \proves^n q
      \synth_k T
    }
  \end{equation*}
  %
  \begin{center}
    where $R =
    \{ \mathtt{.d}_1 \hastype T_1; \ldots; \mathtt{.d}_n \hastype T_n \}$
  \end{center}

  \visible<2->{
    \begin{equation*}
      \infer{
        \Gamma
        \goal \circle A
        \proves^n \mathtt{.next}\; q
        \synth_k T
      }{
        \Gamma
        \goal A
        \proves^{\alert<3>{n+1}} q
        \synth_k T
      }
    \end{equation*}
  }

  \visible<4->{
    \begin{equation*}
      \infer{
        \Gamma \goal S
        \proves^n \cdot
        \synth_n S
      }{}
    \end{equation*}
  }
\end{frame}

\begin{frame}
  \frametitle{$k$ time-steps later: copatterns meet typing}

  \visible<2->{
    \begin{equation*}
      \infer{
        \Gamma \proves^{\alert<3>{n}} \code{fun} q_i \fto t_i \hastype S
      }{
        \Gamma^\prime \goal S \proves^{\alert<3>{n}} q_i \synth_{\alert<3>{k}} T_i
        &
        \Gamma, \Gamma^\prime \proves^{\alert<3>{k}} t \hastype T
      }
    \end{equation*}
  }
\end{frame}

\begin{frame}
  \centering
  \alert{\LARGE Is this an improvement?}

  \includegraphics[width=0.8\textwidth]{doc-brown-skeptical.jpg}
\end{frame}

\begin{frame}
  \frametitle{\code{map} revisited}

  \begin{align*}
    & \code{map} \hastype \square (A \to B) \to \square A \to \square B \\
    \visible<2->{
      & \code{map} fs\; xs \code{.head} =
      \only<2>{\alert<2>{\unk}}
      \only<3->{fs\mathtt{.head}\; xs\mathtt{.head}} \\
    }
    \visible<4->{
      & \code{map} fs\; xs\; \mathtt{.tail}\only<6->{\;\mathtt{.next}}\; =
      \only<4-5>{
        \alert<4->{\unk} \only<5->{\com{\circle \square B}}
      }
      \only<6-7>{
        \alert<6->{\unk} \only<7->{\com{\square B}}
      }
      \only<8->{
        \alert<10>{\code{map} fs\mathtt{.tail.next}\; xs\mathtt{.tail.next}} \\
      }
    }
    \visible<9->{
      & ~ \\
      & \code{map} \hastype
      \code{Str} (A \to B) \to \code{Str} A \to \code{Str} B \\
      %
      & \code{map} fs\; xs\; \mathtt{.head} =
      fs\mathtt{.head}\; xs\mathtt{.head} \\
      %
      & \code{map} fs\; xs\; \mathtt{.tail} =
        \alert<10>{\code{map} fs\mathtt{.tail}\; xs\mathtt{.tail}}
    }
  \end{align*}
\end{frame}

\begin{frame}
  \centering
  \alert{\Huge Yes, this is an improvement.}

  \includegraphics[width=0.9\textwidth]{doc-brown-wow.jpg}
  \pause

  But something is amiss...
\end{frame}

\begin{frame}
  \frametitle{Pulling a fast one at 88mph}

  \begin{center}
    \includegraphics[width=0.6\textwidth]{delorean-flames.png}
  \end{center}

  \pause

  \begin{itemize}[<+->]
    \item
      \alert{When} is the variable \texttt{map} available, in the recursive call?
    \item
      POPL'14 uses (co)iterators for recursion and remarks that
      ``these [iterators] are in some sense timeless terms (they will be used
      at multiple points in time).''
    \item
      Our solution: explicit notion of \emph{stability} inspired by
      Krishnaswami 2013.
  \end{itemize}
\end{frame}

\begin{frame}
  \frametitle{Spacetime is becoming unstable, Marty!}

  \only<1>{
    \begin{align*}
      \boxed{\Gamma \proves t \hastype A}\quad&\text{stable typing judgment}\\
      \boxed{\Gamma \proves^n t \hastype A}\quad&\text{timed typing judgment}
    \end{align*}
  }

  \only<2-4>{
    \begin{equation*}
      \infer[\text{whenever}]{
        \Gamma \proves^{\alert<3>{n}} \mathtt{unstable}\; t \hastype A
      }{
        \Gamma \proves t \hastype A
      }
      \quad
      \quad
      \infer[\text{stabilize}_{\alert<3>{n}}]{
        \Gamma \proves \mathtt{stable}\; t \hastype A
      }{
        \Gamma \proves^{\alert<3>{n}} t \hastype A
      }
    \end{equation*}
    %
    \begin{center}
      \visible<4->{
        Similar to $\forall$ introduction and elimination.
      }
    \end{center}
  }

  \only<5->{
    \visible<5->{
      \begin{equation*}
          \text{Context } \Gamma \quad::=\;
          \cdot \galt \Gamma, x \hastype A \galt \Gamma, x \hastype A^n
      \end{equation*}
    }
    \visible<6->{
      \begin{equation*}
        \infer{
          \Gamma \proves^n \texttt{rec}\;f.\; t \hastype T
        }{
          \Gamma, \alert<7>{f \hastype T} \proves^n t \hastype T
        }
      \end{equation*}
    }
  }
\end{frame}

\begin{frame}
  \frametitle{\texttt{map} revisited, again}

  \only<1-2>{
    \begin{align*}
      & \mathtt{map} \equiv \mathtt{rec}\; \mathtt{map}. \\
      & \mathtt{fun}\; fs\; xs\; \mathtt{.head}
      = fs\mathtt{.head}\; xs\mathtt{.head} \\
      %
      &\orcase\; fs\; xs\; \mathtt{.tail}\; \mathtt{.next}
      = \\
      &\hphantom{\mathtt{fun}}\;
      \visible<2>{
        (\mathtt{unstable}\; \mathtt{map})\;
        fs\mathtt{.tail.next}\;
        xs\mathtt{.tail.next}
      }
    \end{align*}
  }
  \only<3->{
    \begin{align*}
      & \mathtt{map}\; fs\; xs\; \mathtt{.head} =
      fs\mathtt{.head}\; xs\mathtt{.head} \\
      %
      & \mathtt{map}\; fs\; xs\; \mathtt{.tail}\; \mathtt{.next} = \\
      & \hphantom{\mathtt{map}}
        (\alert<4>{\mathtt{unstable}}\; \mathtt{map})\;
        fs\mathtt{.tail.next}\; xs\mathtt{.tail.next}
    \end{align*}
  }
\end{frame}

\begin{frame}
  \centering
  \alert{\Huge The future is now!} \\
  Convenient leak-free functional reactive programming is here!

  \includegraphics[width=0.75\textwidth]{doc-brown-happy.jpg}

  \pause 
  \begin{itemize}
    \item
      Flesh out the operational semantics, and syntactic termination criteria.

    \item
      Study the possibility of inference for \texttt{unstable} and
      \texttt{.next}.

    \item
      Complete the prototype implementation.
  \end{itemize}
\end{frame}

  % \only<4>{
  % }

  % \visible<2->{
  %   \begin{equation*}
  %     \infer{
  %       \Gamma^\text{next}
  %       \goal \circle A
  %       \proves^n \code{.next} q
  %       \synth_k T
  %     }{
  %       \Gamma
  %       \goal A
  %       \proves^{\alert{n+1}} q
  %       \synth_k T
  %     }
  %   \end{equation*}
  % }
  % \visible<3->{
  %   \begin{equation*}
  %     \infer{
  %       \Gamma
  %       \goal S
  %       \proves^n \cdot
  %       \synth_{\alert{n}} T
  %     }{}
  %   \end{equation*}
  % }

% \begin{frame}
%   \frametitle{Goal}
% 
%   \alert{
%     Create a (surface-level) functional reactive programming language with
%     support for copattern matching.
%   }
% \end{frame}
% 
% \begin{frame}
%   \frametitle{Motivation}
% 
%   \only<1>{
%     Many systems are reactive:
%     \begin{itemize}
%       \item operating systems,
%       \item graphical user interfaces,
%       \item games,
%       \item servers,
%       \item etc.
%     \end{itemize}
%   }
%   \only<2>{
%     The traditional event-driven model has many challenges:
%     \begin{itemize}
%       \item concurrency,
%         % how can we run multiple things "at once"?
%         % Java: complex concurrency model
%         % JavaScript: cooperative multitasking
%       \item imperative programming,
%         % components of a reactive system have internal state that they
%         % may modify, which can result in complicated implicit data
%         % dependencies
%       \item callbacks.
%         % requires writing code in CPS, which is a whole-program
%         % transformation unless you have call/cc, is error-prone (e.g.
%         % calling the same continuation twice)
%     \end{itemize}
%   }
%   \only<3>{
%     Functional reactive programming resolves these problems by employing:
%     \begin{itemize}
%       \item pure functional programming,
%         % which avoids the implicit dependencies that arise in stateful
%         % imperative programming
%       \item a direct coding style.
%         % as opposed to the CPS style
%     \end{itemize}
%   }
% 
%   \only<4-5>{
%     Codata types are not as well-understood as data types:
%     \begin{itemize}
%       \item
%         defining codata using constructors leads to problems
%         (coinduction is broken in Coq: it breaks type preservation);
%       \item
%         data and codata cannot be freely mixed in Agda.
%         % we can encode "infinitely often" but not "eventually forever"
%     \end{itemize}
%     \visible<5>{
%       Copatterns try to resolve these issues.
%     }
%   }
% 
%   \only<6>{
%     Copatterns and observations are dual to patterns and constructors:
%     \begin{itemize}
%       \item
%         patterns \emph{analyze} data whereas copatterns \emph{synthesize}
%         codata.
%     \end{itemize}
%   }
%   % \alert{Let's combine copatterns with reactive programming.}
% \end{frame}
% 
% \begin{frame}[fragile]
%   \frametitle{Recall: copattern matching}
% 
%   \lstinputlisting{copattern-example-1.chr}
% 
%   \pause
%   \begin{rem}
%     Not all branches have the same type!
%     \begin{itemize}
%       \item
%         \texttt{fs.head xs.head} has type \texttt{B};
%       \item
%         \texttt{map fs.tail xs.tail} has type \texttt{Stream B}.
%     \end{itemize}
%   \end{rem}
% \end{frame}
% 
% \begin{frame}
%   \frametitle{Copatterns: syntax and typing}
% 
%   \visible<1->{
%     \begin{align*}
%       &\text{Copattern} \; &q &::=
%       \cdot \pip \obs{d} \, q \pip p\, q \\
%       %
%       &\text{Pattern} \; &p &::=
%       x \pip \mathtt{c}\, p \pip \pair{p_1}{p_2} \\
%       %
%       &\text{Term} \; &t &::= \cdots \pip \fun \overrightarrow{q \fto t}
%       %
%     \end{align*}
%   }
% 
%   \visible<2->{
%   }
% \end{frame}

% \begin{frame}
%   \frametitle{Copattern typing: example}
% 
%   \begin{equation*}
%     \infer{
%       \unk
%       \goal \code{S (A -> B) -> S A -> S B}
%       \proves \code{fs xs .head}
%       \synth \unk
%     }{
%       \only<1>{\unk}
%       \visible<2->{
%         \infer{
%           \unk, \code{fs} \hastype \code{S (A -> B)}
%           \goal \code{S A -> S B}
%           \proves \code{xs .head}
%           \synth \unk
%         }{
%           \only<2>{\unk}
%           \visible<3->{
%             \infer{
%               \unk,
%               \code{fs} \hastype \code{S (A -> B)},
%               \code{xs} \hastype \code{S A}
%               \goal \code{S B}
%               \proves \code{.head}
%               \synth \unk
%             }{
%               \only<3>{\unk}
%               \visible<4->{
%                 \infer{
%                   \unk,
%                   \code{fs} \hastype \code{S (A -> B)},
%                   \code{xs} \hastype \code{S A}
%                   \goal \code{B}
%                   \proves \cdot
%                   \synth \code{B}
%                 }{}
%               }
%             }
%           }
%         }
%       }
%     }
%   \end{equation*}
% \end{frame}

% \begin{frame}
%   \frametitle{Recall: LTL}
% 
%   %\begin{itemize}[<+->]
%   %  \item A.~Pnueli (1977) introduces LTL for formal verification.
%   %  \item R.~Davies (1996) uses LTL for binding-time analysis (staged
%   %    computation).
%   %  \item P.~Hudak and C.~Elliott (1997) invent FRP, with continuous time.
%   %  \item A.~Jeffrey (2012) observes that LTL can be used as a type system FRP.
%   %  \item A.~Cave et al. (2014) use LTL for FRP.
%   %\end{itemize}
% 
%   \only<1>{
%     \begin{itemize}
%       \item
%         Based on the \emph{lax modality} $\circle$.
% 
%       \item
%         More familiar modalities $\square$ (always) and $\lozenge$ (eventually)
%         can be \emph{derived} using (co)induction:
%         \begin{align*}
%           \square A &\equiv \nu X.\; A \times \circle X \\
%           \lozenge A &\equiv \mu X.\; A + \circle X
%         \end{align*}
%     \end{itemize}
%   }
%   \only<2>{
%     \begin{itemize}
%       \item A.~Cave et al. (2014) use LTL for FRP with two ``time zones''.
% 
%         RULES
% 
%       \item R.~Davies (1996) uses LTL for staged computation with arbitrarily
%         many time zones.
% 
%         RULES
%     \end{itemize}
%   }
% \end{frame}
% 
% \begin{frame}
%   \frametitle{Timeline}
% 
%   \begin{description}
%     \item[1977:]
%       A.~Pnueli introduces LTL for verification
%     \item[1987:]
%       T.~Hagino develops codatatypes based on coinduction
%     \item[1996:]
%       R.~Davies applies LTL to binding-time analysis
%     \item[1997:]
%       P.~Hudak and C.~Elliott invent FRP, with continuous time
%     \item[2012:]
%       A.~Jeffrey observes $\Gamma \proves_\text{LTL} \text{FRP}$
%     \item[2013:]
%       A.~Abel, B.~Pientka et al. introduce copatterns
%     \item[2014:]
%       A.~Cave, F.~Ferreira et al. develop a language with LTL as a type system
%   \end{description}
% \end{frame}

\end{document}
