\documentclass[11pt,letterpaper]{article}

\usepackage[geometry]{jakemath}

\author{Jacob Thomas Errington (260636023)}
\title{Assignment \#2\\Language-based security -- COMP 523}
\date{Thursday, 5 October 2017}

\DeclareMathOperator{\LetOp}{\mathqd{let}}
\DeclareMathOperator{\InOp}{\mathqd{in}}
\newcommand{\Let}[2]{\LetOp #1 \InOp #2}

\DeclareMathOperator{\DelayOp}{\mathqd{delay}}
\newcommand{\delay}{\DelayOp}

\DeclareMathOperator{\ForceOp}{\mathqd{force}}
\newcommand{\force}{\ForceOp}

\DeclareMathOperator{\SuspOp}{\mathqd{susp}}
\newcommand{\susp}{\SuspOp}

\newcommand{\iszero}{\operatorname{\mathtt{iszero}}}
\renewcommand{\succ}{\operatorname{\mathtt{succ}}}
\newcommand{\pred}{\operatorname{\mathtt{pred}}}
\newcommand{\nat}{\mathtt{nat}}
\newcommand{\case}[1]{\operatorname{\mathtt{case}}#1\operatorname{\mathtt{of}}}

\begin{document}

\maketitle

\section{Lazy evaluation}

\begin{enumerate}
  \item
    $\force \equiv \lambda x. \Let{\delay x^\prime = x}{x^\prime}$

  \item
    We want to show operational equivalence between an expression $e$ and
    $\force(\delay e)$.

    \begin{prop}
      For any expression $e$,
      $\force(\delay e) \evalbig v$
      if and only if
      $e \evalbig v$.
    \end{prop}

    \begin{proof}
      Suppose $\force(\delay e) \evalbig v$.
      First we expand the definition of $\force$.
      %
      \begin{equation*}
        (\lambda x. \Let{\delay x^\prime = x}{x^\prime}) (\delay e) \evalbig v
      \end{equation*}

      By a lemma\footnotemark{} seen in class, we can proceed with a $\beta$
      reduction.
      \footnotetext{%
        The lemma relates the big-step and small-step semantics.
        \emph{%
          If $t^\prime \evalbig v$ and $t \evalto t^\prime$,
          then $t \evalbig v$.
        }
      }
      %
      \begin{equation*}
        \mathcal{D} : \Let{\delay x^\prime = \delay e}{x^\prime} \evalbig v
      \end{equation*}

      By inversion of $\mathcal{D}$,
      we obtain
      %
      \begin{align*}
        \mathcal{D}_1 &: \delay e \evalbig \delay e^\prime \\
        \mathcal{D}_2 &: [e^\prime/x^\prime]x^\prime \evalbig v
      \end{align*}

      By inversion of $\mathcal{D}_1$, we deduce that $e = e^\prime$.
      Then by substitution we deduce that $e \evalbig v$.

      Next suppose $\mathcal{D} : e \evalbig v$.
      We construct the following derivation.
      \begin{equation*}
        \infer{\Let{\delay x = \delay e}{x} \evalbig v}{
          \infer{\delay e \evalbig \delay e}{}
          &
          \infer[\text{ass.}]{e \evalbig v}{}
        }
      \end{equation*}
      %
      Finally we would use a variant of the lemma from class to get to the
      unevaluated form $\force(\delay e)$.
    \end{proof}

  \item
    We extend substitutions for the $\delay$ introduction and elimination
    forms.

    \begin{align*}
      [e^\prime/x](\delay e) &= \delay [e^\prime/x]e \\
      [e^\prime/x](\Let{\delay x^\prime = e_1}{e_2})
        &= \Let{\delay x^\prime = [e^\prime/x]e_1}{[e^\prime/x]e_2}
    \end{align*}
    where in the $\Let{\cdot}{\cdot}$ case we ensure the substitution is
    capture-avoiding.

  \item
    We show type preservation for the operational semantics.

    \begin{prop}[Type preservation.]
      If $\Gamma \proves t \hastype T$ and $t \evalbig v$,
      then $\Gamma \proves v \hastype T$.
    \end{prop}

    \begin{proof}
      By structural induction on the evaluation $\mathcal{D} : t \evalbig v$.

      \begin{description}
        \item[Case]
          We have $\mathcal{D} : \delay e \evalbig \delay e$,
          i.e. term to evaluate is already a value.
          The terms are the same, so by type uniqueness they have the same
          type.

        \item[Case]
          We have that $\mathcal{D}$ is
          \begin{equation*}
            \infer{\Let{\delay x = e_1}{e_2} \evalbig v}{
              \deduce[\mathcal{D}_1]{e_1 \evalbig \delay e^\prime}{}
              &
              \deduce[\mathcal{D}^\prime]{[e^\prime/x]e_2 \evalbig v}{}
            }
          \end{equation*}
          %
          and we know that
          $\Gamma \proves \Let{\delay x = e_1}{e_2} \hastype T$
          by assumption.

          By inversion of the typing rule for the $\Let{\cdot}{\cdot}$ form, we
          deduce
          %
          \begin{align*}
            \mathcal{F}_1 &: \Gamma \proves e_1 \hastype \susp S \\
            \mathcal{F}_2 &: \Gamma, x \hastype S \proves e_2 \hastype T
          \end{align*}

          By the induction hypothesis on $\mathcal{F}_1$ and $\mathcal{D}_1$,
          we deduce
          $\Gamma \proves \delay e^\prime \hastype \susp S$.

          By inverting the typing rule for $\delay$ introduction, we deduce
          $\mathcal{F} : \Gamma \proves e^\prime \hastype S$.

          By the substitution lemma on $\mathcal{F}$ and $\mathcal{F}_2$,
          we deduce
          $\mathcal{F}^\prime : \Gamma \proves [e^\prime/x]e_2 \hastype T$.

          By the induction hypothesis on
          $\mathcal{F}^\prime$ and $\mathcal{D}^\prime$,
          we deduce $\mathcal{E} : \Gamma \proves v \hastype T$.
      \end{description}
    \end{proof}

  \item
    Type preservation breaks down if we pick the rule
    \begin{equation*}
      \infer{\Gamma \proves \Let{\delay x = e_1}{e_2} \hastype T}{%
        \Gamma \proves e_1 \hastype \susp S
        &
        \Gamma \proves [e_1/x]e_2 \hastype T
      }
    \end{equation*}

    The proof would be the same up to the point where we invoke the
    substitution lemma, since that's the first time we touch the typing of
    $e_2$.
    The lemma says
    \emph{%
      if $\Gamma \proves N \hastype T$
      and $\Gamma, x \hastype T, \Gamma^\prime \proves M \hastype S$,
      then $\Gamma, \Gamma^\prime \proves [N/x]M \hastype S$.
    }
    So it wouldn't be possible to use the lemma.

    Concretely, we would have
    $F_2 : \Gamma \proves [e_1/x]e_2 \hastype T$,
    but in order to complete the proof, we would need
    to have $\Gamma \proves [e^\prime/x]e_2 \hastype T$
    in order to use the induction hypothesis the second time.

  \item
    We extend the definition of values and prove value soundness.

    Since we don't evaluate $\delay e$, we simply consider delayed computations
    to be values. This agrees with the idea that values should evaluate to
    themselves, since $\delay e \evalbig \delay e$.

    \begin{prop}
      If $e \evalbig v$, then $v$ is a value.
    \end{prop}

    \begin{proof}
      By structural induction on $\mathcal{D} : e \evalbig v$.

      \begin{description}
        \item[Case]
          $\mathcal{D} : \delay e \evalbig \delay e$ is trivial.

        \item[Case] we have that $\mathcal{D}$ is
          \begin{equation*}
            \infer{\Let{\delay x = e_1}{e_2} \evalbig v}{
              \deduce[\mathcal{D}_1]{e_1 \evalbig \delay e^\prime}{}
              &
              \deduce[\mathcal{D}^\prime]{[e^\prime/x]e_2 \evalbig v}{}
            }
          \end{equation*}

          We appeal to the induction hypothesis on $\mathcal{D}^\prime$ and
          immediately conclude $v$ is a value.
      \end{description}
    \end{proof}

  \item
    Rather than eliminate $\delay$ using a let form, we consider a builtin
    $\force$ eliminator.
    It would be evaluated like this:
    \begin{equation*}
      \infer{\force e \evalbig v}{
        e \evalbig \delay e^\prime
        &
        e^\prime \evalbig v
      }
    \end{equation*}

    Aesthetically this rule looks pretty nice because it reminds me of a
    transitivity property if you squint a little.

    Proving type preservation would be simpler.
    \begin{description}
      \item[Case]
        we have
        \begin{equation*}
          \infer{\force e \evalbig v}{
            \deduce[\mathcal{D}_1]{e \evalbig \delay e^\prime}{}
            &
            \deduce[\mathcal{D}_2]{e^\prime \evalbig v}{}
          }
        \end{equation*}
        and
        $\mathcal{F} : \Gamma \proves \force e \hastype T$.

        By inverting the $\force$ typing rule
        we deduce $\mathcal{F}^\prime : \Gamma \proves e \hastype \susp T$.

        By the induction hypothesis on $\mathcal{D}_1$ and
        $\mathcal{F}^\prime$,
        we deduce
        $\mathcal{E}^\prime : \Gamma \proves \delay e^\prime \hastype \susp T$.

        By inverting the $\delay$ type rule
        we deduce $\mathcal{E} : \Gamma \proves e^\prime \hastype T$.

        By the induction hypothesis on $\mathcal{D}_2$ and $\mathcal{E}$,
        we deduce $\Gamma \proves v \hastype T$.
    \end{description}
    No substitution lemma required!

    No downsides jump out at me.
\end{enumerate}

\section{Case statement}

\begin{enumerate}
  \item
    With a case statement for naturals, certain earlier builtins can now be
    replaced by functions.
    %
    \begin{enumerate}
      \item
        $%
        \iszero \equiv \lambda n : \nat.
        \case{n} \succ k \goesto \true
        \orcase \mathtt{z} \goesto \false
        $

      \item
        $%
        \pred \equiv \lambda n : \nat.
        \case{n} \succ k \goesto k
        \orcase \mathtt{z} \goesto \mathtt{z}
        $
    \end{enumerate}

  \item
    We extend substitutions for case-expressions.
    (I am assuming that the duplicate $x$ in the question is a typo,
    since the substitution should be capture-avoiding.)
    %
    \begin{equation*}
      [s/x](\case{t} \mathtt{z} \goesto t_1 \orcase \succ k \goesto t_2)
      =
      \case{[s/x]t}
      \mathtt{z} \goesto [s/x]t_1
      \orcase \succ k \goesto [s/x]t_2
    \end{equation*}
    %
    where $x \neq k$.

  \item
    We define a typing rule for case-expressions.
    %
    \begin{equation*}
      \infer{%
        \Gamma \proves
        \case{t} \mathtt{z} \goesto t_1
        \orcase \succ k \goesto t_2
        \hastype T
      }{%
        \Gamma \proves t \hastype \nat
        &
        \Gamma \proves t_1 \hastype T
        &
        \Gamma, k \hastype \nat \proves t_2 \hastype T%
      }
    \end{equation*}

  \item
    We extend the substitution lemma for case-expressions.
    %
    \begin{prop}
      If $\Gamma, x \hastype T, \Gamma^\prime \proves s \hastype S$
      and $\Gamma \proves t \hastype T$,
      then $\Gamma, \Gamma^\prime \proves [t/x]s \hastype S$.
    \end{prop}
    %
    \begin{proof}
      By structural induction on the typing derivation
      $\mathcal{D} : \Gamma, x \hastype T, \Gamma^\prime \proves s \hastype S$.
      %
      \begin{description}
        \item[Case] we have that $\mathcal{D}$ is of the form
          %
          \begin{equation*}
            \infer{%
              \Gamma, x \hastype T, \Gamma^\prime
              \proves \mathtt{z} \hastype \nat%
            }{}
          \end{equation*}

          By the definition of substitution,
          $[t/x]\mathtt{z} \equiv \mathtt{z}$,
          so we deduce
          $\Gamma, \Gamma^\prime \proves [t/x]\mathtt{z} \hastype \nat$
          by weakening on the typing rule for $\mathtt{z}$.

        \item[Case] we have that $\mathcal{D}$ is of the form
          %
          \begin{equation*}
            \infer{%
              \Gamma, x \hastype T, \Gamma^\prime
              \proves \succ k \hastype \nat%
            }{%
              \deduce[\mathcal{D}^\prime]{%
                \Gamma, x \hastype T, \Gamma^\prime
                \proves k \hastype \nat%
              }{}%
            }
          \end{equation*}

          By assumption, we have
          $\mathcal{E} : \Gamma \proves t \hastype \nat$.

          By the induction hypothesis on $\mathcal{D}^\prime$ and
          $\mathcal{E}$,
          we deduce
          $\mathcal{E}^\prime :
          \Gamma, \Gamma^\prime \proves [t/x]k \hastype \nat$.

          By the typing rule for $\succ$,
          we deduce
          $\mathcal{F} :
          \Gamma, \Gamma^\prime \proves \succ [t/x]k \hastype \nat$.

          By inversion on the substitution definition,
          we deduce
          $\Gamma, \Gamma^\prime \proves [t/x] \succ k \hastype \nat$.

        \item[Case] we have that $\mathcal{D}$ is of the form

          \begin{equation*}
            \infer{%
              \Gamma, x\hastype T, \Gamma^\prime
              \proves
              \case{n} \mathtt{z} \goesto t_1
              \orcase \succ k \goesto t_2
              \hastype S%
            }{%
              \deduce[\mathcal{D}_1]{%
                \Gamma, x\hastype T, \Gamma^\prime
                \proves
                n \hastype \nat%
              }{}
              &
              \deduce[\mathcal{D}_2]{%
                \Gamma, x\hastype T, \Gamma^\prime
                \proves
                t_1 \hastype S%
              }{}
              &
              \deduce[\mathcal{D}_3]{%
                \Gamma, x\hastype T, \Gamma^\prime, k\hastype \nat
                \proves
                t_2 \hastype S%
              }{}
            }
          \end{equation*}
          %
          and we have by assumption that
          %
          $\mathcal{D}^\prime : \Gamma \proves t \hastype T$.

          By the induction hypothesis on $\mathcal{D}_1$ and
          $\mathcal{D}^\prime$,
          we deduce
          $\mathcal{D}_1^\prime : \Gamma, \Gamma^\prime
          \proves [t/x]n \hastype \nat$.

          By the induction hypothesis on $\mathcal{D}_2$ and
          $\mathcal{D}^\prime$,
          we deduce
          $\mathcal{D}_2^\prime : \Gamma, \Gamma^\prime
          \proves [t/x]t_1 \hastype S$.

          By the induction hypothesis on $\mathcal{D}_3$ and
          $\mathcal{D}^\prime$,
          we deduce
          $\mathcal{D}_3^\prime : \Gamma, \Gamma^\prime, k \hastype \nat
          \proves [t/x]t_2 \hastype \nat$.

          By the typing rule for case-expressions,
          we construct the derivation $\mathcal{E}$.
          %
          \begin{equation*}
            \infer{
              \Gamma, \Gamma^\prime
              \proves
              \case{[t/x]n} \mathtt{z} \goesto [t/x]t_1
              \orcase \succ k \goesto [t/x]t_2
              \hastype S%
            }{%
              \deduce[\mathcal{D}_1^\prime]{%
                \Gamma, \Gamma^\prime
                \proves
                [t/x]n \hastype \nat%
              }{}
              &
              \deduce[\mathcal{D}_2^\prime]{%
                \Gamma, \Gamma^\prime
                \proves
                [t/x]t_1 \hastype S%
              }{}
              &
              \deduce[\mathcal{D}_3^\prime]{%
                \Gamma, \Gamma^\prime, k\hastype \nat
                \proves
                [t/x]t_2 \hastype S%
              }{}
            }
          \end{equation*}

          By inversion of the substitution definition,
          we deduce
          $\mathcal{E} :
          \Gamma, \Gamma^\prime
          \proves
          [t/x](\case{n} \mathtt{z} \goesto t_1 \orcase \succ k \goesto t_2)
          \hastype S$, as required.
      \end{description}
    \end{proof}

  \item
    We extend type preservation for case-expressions.
    %
    \begin{prop}
      If $\Gamma \proves t \hastype T$ and $t \evalbig v$,
      then $\Gamma \proves v \hastype T$.
    \end{prop}

    \begin{proof}
      By structural induction on the evaluation $\mathcal{D} : t \evalbig v$.
      \begin{description}
        \item[Case] we have that $\mathcal{D}$ is of the form
          %
          \begin{equation*}
            \infer{%
              \case{t} \mathtt{z} \goesto t_1
              \orcase \succ k \goesto t_2%
              \evalbig
              v
            }{%
              \deduce[\mathcal{D}_0]{t \evalbig \mathtt{z}}{}
              &
              \deduce[\mathcal{D}_1]{t_1 \evalbig v}{}%
            }
          \end{equation*}
          %
          and that
          $\mathcal{E} :
          \Gamma \proves
          \case{t} \mathtt{z} \goesto t_1
          \orcase \succ k \goesto t_2
          \hastype T$
          by assumption.

          By inverting the typing rule for case-expressions in $\mathcal{E}$,
          we deduce
          \begin{align*}
            \mathcal{E}_0 &: \Gamma \proves t \hastype \nat \\
            \mathcal{E}_1 &: \Gamma \proves t_1 \hastype T \\
            \mathcal{E}_2 &: \Gamma, k\hastype \nat \proves t_2 \hastype T
          \end{align*}

          By the induction hypothesis on $\mathcal{D}_1$ and $\mathcal{E}_1$,
          we deduce
          $\mathcal{E}^\prime : \Gamma \proves v \hastype T$, as required.

        \item[Case] we have that $\mathcal{D}$ is of the form
          %
          \begin{equation*}
            \infer{%
              \case{t} \mathtt{z} \goesto t_1
              \orcase \succ k \goesto t_2%
              \evalbig
              v
            }{%
              \deduce[\mathcal{D}_0]{t \evalbig \succ v^\prime}{}
              &
              \deduce[\mathcal{D}_1]{[v^\prime/k]t_2 \evalbig v}{}%
            }
          \end{equation*}
          %
          and that
          $\mathcal{E} :
          \Gamma \proves
          \case{t} \mathtt{z} \goesto t_1
          \orcase \succ k \goesto t_2
          \hastype T$
          by assumption.

          By inverting the typing rule for case-expressions in $\mathcal{E}$,
          we deduce
          \begin{align*}
            \mathcal{E}_0 &: \Gamma \proves t \hastype \nat \\
            \mathcal{E}_1 &: \Gamma \proves t_1 \hastype T \\
            \mathcal{E}_2 &: \Gamma, k\hastype \nat \proves t_2 \hastype T
          \end{align*}

          By the induction hypothesis on $\mathcal{D}_0$ and $\mathcal{E}_0$,
          we deduce
          $\mathcal{E}_0^\prime : \Gamma \proves \succ v^\prime \hastype \nat$.

          By inverting the typing rule for $\succ$ in $\mathcal{E}_0^\prime$,
          we deduce
          $\mathcal{E}_0^{\prime\prime} :
          \Gamma \proves v^\prime \hastype \nat$.

          By the substitution lemma on $\mathcal{E}_0^{\prime\prime}$ and
          $\mathcal{E}_2$,
          we deduce
          $\mathcal{E}_2^\prime :
          \Gamma \proves [v^\prime/k]t_2 \hastype T$.

          By the induction hypothesis on
          $\mathcal{D}_1$ and $\mathcal{E}_2^\prime$,
          we deduce $\Gamma \proves v \hastype T$, as required.
      \end{description}
    \end{proof}

  \item
    We propose the following corresponding (i.e. strict) small-step semantics.
    %
    \begin{equation*}
      \infer{%
        \case{t} \mathtt{z} \goesto t_1 \orcase \succ k \goesto t_2
        \evalto
        \case{t^\prime} \mathtt{z} \goesto t_1 \orcase \succ k \goesto t_2
      }{t \evalto t^\prime} \\
    \end{equation*}
    %
    \begin{equation*}
      \case{\mathtt{z}} \mathtt{z} \goesto t_1 \orcase \succ k \goesto t_2
      \evalto
      t_1
    \end{equation*}
    %
    \begin{equation*}
      \case{\succ v} \mathtt{z} \goesto t_1 \orcase \succ k \goesto t_2
      \evalto
      [v/k]t_2
    \end{equation*}

  \item
    We show that progress holds for these operational semantics.
    %
    \begin{prop}
      If $\Gamma \proves t \hastype T$, then
      either $t$ is a value or there exists $t^\prime$ such that $t \evalto
      t^\prime$.
    \end{prop}
    %
    \begin{proof}
      By structural induction on the typing derivation
      $\mathcal{D} : \Gamma \proves t \hastype T$.

      \begin{description}
        \item[Case] we have that $\mathcal{D}$ is of the form
          %
          \begin{equation*}
            \infer{%
              \Gamma
              \proves
              \case{t} \mathtt{z} \goesto t_1
              \orcase \succ k \goesto t_2
              \hastype
              T
            }{
              \deduce[\mathcal{D}_1]{\Gamma \proves t \hastype \nat}{}
              &
              \deduce[\mathcal{D}_2]{\Gamma \proves t_1 \hastype T}{}
              &
              \deduce[\mathcal{D}_3]{
                \Gamma, k\hastype \nat \proves t_2 \hastype T
              }{}
            }
          \end{equation*}

          By the induction hypothesis on $\mathcal{D}_1$,
          we deduce that either $t$ is a value or there exists $t^\prime$ such
          that $t \evalto t^\prime$.
          We analyze the two subcases.
          %
          \begin{description}
            \item[Subcase] $t$ is a value.
              Then we proceed by case analysis on $t$,
              since we know $\Gamma \proves t \hastype \nat$.
              \begin{description}
                \item[Case] $t = \mathtt{z}$.

                  Then we can step using the $\mathtt{z}$ evaluation rule for
                  $\case{\cdot}$.

                \item[Case] $t = \succ k$.

                  Then we can step using the $\succ$ evaluation rule for
                  $\case{\cdot}$.
              \end{description}

            \item[Subcase]
              there exists $t^\prime$ such that $t \evalto t^\prime$.

              Then we can step using the congruence rule for the subject of the
              case expression.
          \end{description}
      \end{description}

      This establishes that we have progress for case-expressions.
    \end{proof}
\end{enumerate}

\end{document}
