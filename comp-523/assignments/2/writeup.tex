\documentclass[11pt,letterpaper]{article}

\usepackage[geometry]{jakemath}

\author{Jacob Thomas Errington (260636023)}
\title{Assignment \#2\\Language-based security -- COMP 523}
\date{Thursday, 5 October 2017}

\DeclareMathOperator{\LetOp}{\mathqd{let}}
\DeclareMathOperator{\InOp}{\mathqd{in}}
\newcommand{\Let}[2]{\LetOp #1 \InOp #2}

\DeclareMathOperator{\DelayOp}{\mathqd{delay}}
\newcommand{\delay}{\DelayOp}

\DeclareMathOperator{\ForceOp}{\mathqd{force}}
\newcommand{\force}{\ForceOp}

\DeclareMathOperator{\SuspOp}{\mathqd{susp}}
\newcommand{\susp}{\SuspOp}

\begin{document}

\maketitle

\section{Lazy evaluation}

\begin{enumerate}
  \item
    $\force \equiv \lambda x. \Let{\delay x^\prime = x}{x^\prime}$

  \item
    We want to show operational equivalence between an expression $e$ and
    $\force(\delay e)$.

    \begin{prop}
      For any expression $e$,
      $\force(\delay e) \evalbig v$
      if and only if
      $e \evalbig v$.
    \end{prop}

    \begin{proof}
      Suppose $\force(\delay e) \evalbig v$.
      First we expand the definition of $\force$.
      %
      \begin{equation*}
        (\lambda x. \Let{\delay x^\prime = x}{x^\prime}) (\delay e) \evalbig v
      \end{equation*}

      By a lemma\footnotemark{} seen in class, we can proceed with a $\beta$
      reduction.
      \footnotetext{%
        The lemma relates the big-step and small-step semantics.
        \emph{%
          If $t^\prime \evalbig v$ and $t \evalto t^\prime$,
          then $t \evalbig v$.
        }
      }
      %
      \begin{equation*}
        \mathcal{D} : \Let{\delay x^\prime = \delay e}{x^\prime} \evalbig v
      \end{equation*}

      By inversion of $\mathcal{D}$,
      we obtain
      %
      \begin{align*}
        \mathcal{D}_1 &: \delay e \evalbig \delay e^\prime \\
        \mathcal{D}_2 &: [e^\prime/x^\prime]x^\prime \evalbig v
      \end{align*}

      By inversion of $\mathcal{D}_1$, we deduce that $e = e^\prime$.
      Then by substitution we deduce that $e \evalbig v$.

      Next suppose $\mathcal{D} : e \evalbig v$.
      We construct the following derivation.
      \begin{equation*}
        \infer{\Let{\delay x = \delay e}{x} \evalbig v}{
          \infer{\delay e \evalbig \delay e}{}
          &
          \infer[\text{ass.}]{e \evalbig v}{}
        }
      \end{equation*}
      %
      Finally we would use a variant of the lemma from class to get to the
      unevaluated form $\force(\delay e)$.
    \end{proof}

  \item
    We extend substitutions for the $\delay$ introduction and elimination
    forms.

    \begin{align*}
      [e^\prime/x](\delay e) &= \delay [e^\prime/x]e \\
      [e^\prime/x](\Let{\delay x^\prime = e_1}{e_2})
        &= \Let{\delay x^\prime = [e^\prime/x]e_1}{[e^\prime/x]e_2}
    \end{align*}
    where in the $\Let{\cdot}{\cdot}$ case we ensure the substitution is
    capture-avoiding.

  \item
    We show type preservation for the operational semantics.

    \begin{prop}[Type preservation.]
      If $\Gamma \proves t \hastype T$ and $t \evalbig v$,
      then $\Gamma \proves v \hastype T$.
    \end{prop}

    \begin{proof}
      By structural induction on the evaluation $\mathcal{D} : t \evalbig v$.

      \begin{description}
        \item[Case]
          We have $\mathcal{D} : \delay e \evalbig \delay e$,
          i.e. term to evaluate is already a value.
          The terms are the same, so by type uniqueness they have the same
          type.

        \item[Case]
          We have that $\mathcal{D}$ is
          \begin{equation*}
            \infer{\Let{\delay x = e_1}{e_2} \evalbig v}{
              \deduce[\mathcal{D}_1]{e_1 \evalbig \delay e^\prime}{}
              &
              \deduce[\mathcal{D}^\prime]{[e^\prime/x]e_2 \evalbig v}{}
            }
          \end{equation*}
          %
          and we know that
          $\Gamma \proves \Let{\delay x = e_1}{e_2} \hastype T$
          by assumption.

          By inversion of the typing rule for the $\Let{\cdot}{\cdot}$ form, we
          deduce
          %
          \begin{align*}
            \mathcal{F}_1 &: \Gamma \proves e_1 \hastype \susp S \\
            \mathcal{F}_2 &: \Gamma, x \hastype S \proves e_2 \hastype T
          \end{align*}

          By the induction hypothesis on $\mathcal{F}_1$ and $\mathcal{D}_1$,
          we deduce
          $\Gamma \proves \delay e^\prime \hastype \susp S$.

          By inverting the typing rule for $\delay$ introduction, we deduce
          $\mathcal{F} : \Gamma \proves e^\prime \hastype S$.

          By the substitution lemma on $\mathcal{F}$ and $\mathcal{F}_2$,
          we deduce
          $\mathcal{F}^\prime : \Gamma \proves [e^\prime/x]e_2 \hastype T$.

          By the induction hypothesis on
          $\mathcal{F}^\prime$ and $\mathcal{D}^\prime$,
          we deduce $\mathcal{E} : \Gamma \proves v \hastype T$.
      \end{description}
    \end{proof}

  \item
    Type preservation breaks down if we pick the rule
    \begin{equation*}
      \infer{\Gamma \proves \Let{\delay x = e_1}{e_2} \hastype T}{%
        \Gamma \proves e_1 \hastype \susp S
        &
        \Gamma \proves [e_1/x]e_2 \hastype T
      }
    \end{equation*}

    The proof would be the same up to the point where we invoke the
    substitution lemma, since that's the first time we touch the typing of
    $e_2$.
    The lemma says
    \emph{%
      if $\Gamma \proves N \hastype T$
      and $\Gamma, x \hastype T \proves M \hastype S$,
      then $\Gamma \proves [N/x]M \hastype S$.
    }
    So it wouldn't be possible to use the lemma.

    Concretely, we would have
    $F_2 : \Gamma \proves [e_1/x]e_2 \hastype T$,
    but in order to complete the proof, we would need
    to have $\Gamma \proves [e^\prime/x]e_2 \hastype T$
    in order to use the induction hypothesis the second time.

  \item
    We extend the definition of values and prove value soundness.

    Since we don't evaluate $\delay e$, we simply consider delayed computations
    to be values. This agrees with the idea that values should evaluate to
    themselves, since $\delay e \evalbig \delay e$.

    \begin{prop}
      If $e \evalbig v$, then $v$ is a value.
    \end{prop}

    \begin{proof}
      By structural induction on $\mathcal{D} : e \evalbig v$.

      \begin{description}
        \item[Case]
          $\mathcal{D} : \delay e \evalbig \delay e$ is trivial.

        \item[Case] we have that $\mathcal{D}$ is
          \begin{equation*}
            \infer{\Let{\delay x = e_1}{e_2} \evalbig v}{
              \deduce[\mathcal{D}_1]{e_1 \evalbig \delay e^\prime}{}
              &
              \deduce[\mathcal{D}^\prime]{[e^\prime/x]e_2 \evalbig v}{}
            }
          \end{equation*}

          We appeal to the induction hypothesis on $\mathcal{D}^\prime$ and
          immediately conclude $v$ is a value.
      \end{description}
    \end{proof}

  \item
    Rather than eliminate $\delay$ using a let form, we consider a builtin
    $\force$ eliminator.
    It would be evaluated like this:
    \begin{equation*}
      \infer{\force e \evalbig v}{
        e \evalbig \delay e^\prime
        &
        e^\prime \evalbig v
      }
    \end{equation*}

    Aesthetically this rule looks pretty nice because it reminds me of a
    transitivity property if you squint a little.

    Proving type preservation would be simpler.
    \begin{description}
      \item[Case]
        we have
        \begin{equation*}
          \infer{\force e \evalbig v}{
            \deduce[\mathcal{D}_1]{e \evalbig \delay e^\prime}{}
            &
            \deduce[\mathcal{D}_2]{e^\prime \evalbig v}{}
          }
        \end{equation*}
        and
        $\mathcal{F} : \Gamma \proves \force e \hastype T$.

        By inverting the $\force$ typing rule
        we deduce $\mathcal{F}^\prime : \Gamma \proves e \hastype \susp T$.

        By the induction hypothesis on $\mathcal{D}_1$ and
        $\mathcal{F}^\prime$,
        we deduce
        $\mathcal{E}^\prime : \Gamma \proves \delay e^\prime \hastype \susp T$.

        By inverting the $\delay$ type rule
        we deduce $\mathcal{E} : \Gamma \proves e^\prime \hastype T$.

        By the induction hypothesis on $\mathcal{D}_2$ and $\mathcal{E}$,
        we deduce $\Gamma \proves v \hastype T$.
    \end{description}
    No substitution lemma required!

    No downsides jump out at me.
\end{enumerate}

\end{document}
