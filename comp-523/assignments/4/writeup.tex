\documentclass[11pt,letterpaper]{article}

\author{Jacob Thomas Errington (260636023)}
\title{Assignment \#4\\Language-based security -- COMP 523}
\date{Thursday, 16 November 2017}

\usepackage[geometry]{jakemath}

\begin{document}

\maketitle

\section{Hereditary substitutions}

\begin{enumerate}
  \item
    We first extend the type approximations and type erasure for
    $\Sigma$-types.
    %
    \begin{equation*}
      \text{Type Approximation}\quad
      \alpha,\beta
      ::= \alpha
      | \alpha \to \beta
      | \alpha \times \beta
    \end{equation*}
    %
    Specifically, we approximate a dependent pair as a plain pair.

    Next, the erasure proceeds in the same way as for function spaces: we just
    forget the dependency and erase the subtrees recursively, replacing the
    approximate simple type.
    %
    \begin{equation*}
      (\Sigma x \hastype A. B)^- = A^- \times B^-
    \end{equation*}

  \item
    Next, we extend the definition of hereditary substitutions to dependent
    pairs.

    We dispense with the easy case:
    %
    \begin{equation*}
      [M/x]^n_\alpha (N_1, N_2) = ([M/x]^n_\alpha N_1, [M/x]^n_\alpha N_2)
    \end{equation*}
    %
    Here there isn't anything special to do: since hereditary substitution
    preserves normal forms, the result of the normal substitution to the two
    subterms yields normal terms. Forming the pair of normal terms gives a
    normal term, so we're fine.

    Next we have the trickier cases, which can give rise to redeces.

    First, we may need to substitute under a $\fst$, in which case we proceed
    as follows.
    %
    \begin{equation*}
      [M/x]^r_\alpha (\fst R) = M_1 \hastype \beta_1
    \end{equation*}
    %
    if $[M/x]^r_\alpha R = (M_1, M_2) \hastype \beta_1 \times \beta_2$.

    In this case, if we substitute under a $\fst$ and discover a (typed) pair,
    then we have produced a redex. To resolve this we take the first component
    of the pair and adjust the result type accordingly.

    Second, we may need to substitute under a $\snd$, in which case we proceed
    as follows.
    %
    \begin{equation*}
      [M/x]^r_\alpha (\snd R) = M_2 \hastype \beta_2
    \end{equation*}
    %
    if $[M/x]^r_\alpha R = (M_1, M_2) \hastype \beta_1 \times \beta_2$.

    This case is resolved in the same way.

  \item
    Next we wish to prove termination of hereditary substitution.

    \begin{lem}
      If $[M/x]^r_\alpha = M^\prime \hastype \alpha^\prime$,
      then $\alpha^\prime \leq \alpha$.
    \end{lem}

    \begin{proof}
      We proceed by induction on the definition of the substitution,
      considering the cases that produce typed results.
      %
      \begin{description}
        \item[Case]
          $[M/x]^r_\alpha x = M : \alpha$

          Here, $\alpha^\prime = \alpha$, so we're done.

        \item[Case]
          $[M/x]^r_\alpha (R N) = M^{\prime\prime} : \beta$
          where
          $[M/x]^r_\alpha R = \lambda y. M^\prime \hastype \alpha_1 \to \beta$
          and
          $[M/x]_\alpha^n N = N^\prime$
          and
          $[N^\prime/x]_{\alpha_1}^n = M^{\prime\prime}$.

          Note that $\alpha_1 \to \beta$ is no greater than $\alpha$ by the
          induction hypothesis, so $\beta < \alpha$.

        \item[Case]
          $[M/x]^r_\alpha (\fst R) = M_1 \hastype \beta_1$
          where
          $[M/x]^r_\alpha R = (M_1, M_2) \hastype \beta_1 \times \beta_2$.

          By the induction hypothesis $\beta_1 \times \beta_2 \leq \alpha$,
          so $\beta_1 < \alpha$.

        \item[Case] for $\snd$ follows the same argument.
      \end{description}
    \end{proof}

    \begin{prop}
      $[M/x]^*_\alpha (\cdot)$ terminates, either by returning a result or
      failing after a finite number of steps.
    \end{prop}

    \begin{proof}
      By lexicographic induction on the form of $\alpha$ and the form of the
      term.

      \begin{description}
        \item[Case]
          $[M/x]^n_\alpha (\lambda y. N)$.

          Then by the induction hypothesis $[M/x]^n_\alpha (N)$ terminates.
          (Here the term gets smaller but the type stays the same.)
          If it fails, then we fail.
          Else, we have $N^\prime = [M/x]^n_\alpha (N)$, so we form the lambda
          and return it.

        \item[Case]
          $[M/x]^n_\alpha (M_1, M_2)$.

          Then by the induction hypothesis
          $[M/x]^n_\alpha M_1$ and $[M/x]^n_\alpha M_2$ terminate.
          If either fails, then we fail.
          Else we have $M_1^\prime = [M/x]^n_\alpha M_1$
          and $M_2^\prime = [M/x]^n_\alpha M_2$,
          so we return $(M_1^\prime, M_2^\prime)$.

        \item[Case]
          When we switch from the normal substitution to the neutral
          substitution, the use of the induction hypothesis is justified
          because the neutral substitution only switches back to the normal
          substitution on a smaller type or a smaller term.
          We will point out these uses of the induction hypothesis with ``(*)''
          when they arise below.

        \item[Case]
          $[M/x]_\alpha^r y$.

          If $y = x$, then we return $M \hastype \alpha$ (a typed normal term).
          Else we return $y$ (an untyped neutral term).

        \item[Case]
          $[M/x]_\alpha^r (R N)$.

          By the induction hypothesis (*), $[M/x]_\alpha^n N$ terminates since
          $N$ is a smaller term.
          If it fails, then we fail.
          Else we have $N^\prime = [M/x]_\alpha^n N$.

          By the induction hypothesis, $[M/x]_\alpha^r R$ terminates since the
          term is smaller.
          If it fails, then we fail.

          Since this is a neutral substitution it can return a typed normal
          term or another neutral term.
          %
          \begin{description}
            \item[Subcase]
              We have $R^\prime = [M/x]_\alpha^r R$ is a neutral term.

              We return $R^\prime N^\prime$.

            \item[Subcase]
              We have
              $M_1 \hastype \alpha_1 \to \beta = [M/x]_\alpha^r R$
              is a typed normal term.

              By a canonical forms lemma,
              $M_1 = \lambda y. M^\prime \hastype \alpha_1 \to \beta$.

              Then by the induction hypothesis (*),
              $[N^\prime/y]_{\alpha_1}^n M^\prime$
              terminates.
              If it fails, then we fail.
              Else we have
              $M^{\prime\prime} = [N^\prime/y]_{\alpha_1}^n M^\prime$
              and we return
              $M^{\prime\prime} \hastype \beta$.

            \item[Otherwise]
              the term must be ill-typed so we fail.
          \end{description}

        \item[Case]
          $[M/x]_\alpha^r (\fst R)$.

          % if $(M_1, M_2) = [M/x]_\alpha^r R$.
          %
          By the induction hypothesis $[M/x]_\alpha^r R$ terminates since the
          term is smaller.
          If it fails, then we fail.

          Else since it's a neutral substitution, it can produce a typed normal
          term or an untyped neutral term.
          %
          \begin{description}
            \item[Subcase]
              We have
              $M^\prime \hastype \beta_1 \times \beta_2 = [M/x]_\alpha^r R$
              is a typed normal term.

              Then by a canonical forms lemma we deduce that
              $M^\prime = (M_1, M_2)$ so that $M_1 \hastype \beta_1$.

              We return $M_1 \hastype \beta_1$.

            \item[Subcase]
              We have
              $R^\prime = [M/x]_\alpha^r R$
              is a neutral term.

              Then we return $\fst R^\prime$ which is an untyped neutral term.

            \item[Otherwise]
              the term must be ill-typed so we fail.
          \end{description}

        \item[Case]
          for $\snd$ follows the same argument as for $\fst$.
      \end{description}
    \end{proof}
\end{enumerate}

\end{document}
