\documentclass[11pt]{article}

\author{Jacob Thomas Errington (260636023)}
\title{Assignment \#5\\Language-based security -- COMP 523}
\date{30 November 2017}

\usepackage[geometry,questions]{jakemath}

\begin{document}

\maketitle

\question{Evaluating copattern matching with stacks}

\begin{enumerate}
  \item
    We extend the definition of stacks to also allow storing intermediate
    expressions when we evaluate pairs and finite data.

    \begin{align*}
      \text{Stack}\; E\quad
      \operatorname{::=} \cdots
      \alt (c_i\; \bullet) E
      \alt (\bullet, s) E
      \alt (v, \bullet) E
    \end{align*}

  \item
    We extend the rules of stepping configurations to also step pairs and
    finite data.

    \begin{align*}
      (t, s); E &\evalto t; (\bullet, s) E \\
      v; (\bullet, s) E &\evalto (v, s); E \\
      (v, s); E &\evalto s; (v, \bullet) E \\
      v_2; (v_1, \bullet) E &\evalto (v_1, v_2) ; E \\
      c_i\; t; E &\evalto t; (c_i\; \bullet) E \\
      v; (c_i\; \bullet) E &\evalto c_i\; v ; E
    \end{align*}

  \item
    We extend the typing rules for stacks to handle the new intermediate
    states.

    \begin{align*}
      \infer{
        T_1 \proves (\bullet, s) E \hastype S
      }{
        \proves s \hastype T_2
        &
        T_1 \times T_2 \proves E \hastype S
      }
      &\quad
      \infer{
        T_2 \proves (t, \bullet) E \hastype S
      }{
        \proves t \hastype T_1
        &
        T_1 \times T_2 \proves E \hastype S
      } \\
      %
      \infer{
        [\mu X.\; D/X]T_i \proves (c_i\; \bullet) E \hastype S
      }{
        \mu X.\; D \proves E \hastype S
      }
      &\quad
      ~
    \end{align*}
    %
    where $D = \langle c_1\hastype T_1,\ldots,c_n\hastype T_n \rangle$.

    (Our typing rules are slightly more general than they need to be since we write
    $t \hastype T_1$ instead of $v \hastype T_1$.)

  \item
    We extend the proof of type preservation for the new rules.

    \begin{prop}[Type preservation]
      If $\mathcal{D} \deduc \proves t; E \hastype S$
      and $t;E \evalto t^\prime ; E^\prime$,
      then $\proves t^\prime ; e^\prime \hastype S$.
    \end{prop}

    \begin{proof}
      By case analysis of the stepping relation.

      \begin{description}
        \item[Case]
          $(t, s); E \evalto t; (\bullet, s) E$.

          Therefore $\mathcal{D}$ is of the form
          %
          \begin{equation*}
            \infer{%
              \proves (t, s) ; E \hastype S
            }{%
              \deduce[\mathcal{D}^\prime]{\proves (t, s) \hastype T}{}
              &
              \deduce[\mathcal{F}]{T \proves E \hastype S}{}
            }
          \end{equation*}

          We invert $\mathcal{D}^\prime$ to
          deduce $\mathcal{D}_1 \deduc \proves t \hastype T_1$
          and $\mathcal{D}_2 \deduc \proves s \hastype T_2$,
          and we discover that $T = T_1 \times T_2$.

          Then we construct a derivation
          %
          \begin{equation*}
            \infer{%
              \proves t ; (\bullet, s) E \hastype S
            }{%
              \deduce[\mathcal{D}_1]{\proves t \hastype T_1}{}
              &
              \infer{%
                T_1 \proves (\bullet, s) E \hastype S
              }{%
                \deduce[\mathcal{D}_2]{\proves s \hastype T_2}{}
                &
                \deduce[\mathcal{F}]{T_1 \times T_2 \proves E \hastype S}{}
              }
            }
          \end{equation*}
          %
          as required.

        \item[Case]
          $v; (\bullet, s) \evalto (v, s) ; E$.

          Therefore $\mathcal{D}$ is of the form
          %
          \begin{equation*}
            \infer{%
              v ; (\bullet, s) E \hastype S
            }{%
              \deduce[\mathcal{D}^\prime]{\proves v \hastype T}{}
              &
              \deduce[\mathcal{F}]{T \proves (\bullet, s) E \hastype S}{}
            }
          \end{equation*}

          We invert $\mathcal{F}$ to deduce
          %
          \begin{equation*}
            \infer{%
              T_1 \proves (\bullet, s) E \hastype S
            }{%
              \deduce[\mathcal{F}_1]{\proves s \hastype T_2}{}
              &
              \deduce[\mathcal{F}_2]{T_1 \times T_2 \proves E \hastype S}{}
            }
          \end{equation*}
          %
          and we discover $T = T_1$.

          Then we construct a derivation
          %
          \begin{equation*}
            \infer{%
              \proves (v, s) ; E \hastype S
            }{%
              \infer{%
                \proves (v, s) \hastype T_1 \times T_2
              }{%
                \deduce[\mathcal{D}^\prime]{\proves v \hastype T_1}{}
                &
                \deduce[\mathcal{F}_1]{\proves s \hastype T_2}{}
              }
              &
              \deduce[\mathcal{F}_2]{T_1 \times T_2 \proves E \hastype S}{}
            }
          \end{equation*}
          %
          as required.

        \item[Case]
          $(v, s) ; E \evalto s ; (v, \bullet) E$.

          Therefore $\mathcal{D}$ is of the form
          %
          \begin{equation*}
            \infer{%
              \proves (v, s) ; E \hastype S
            }{%
              \deduce[\mathcal{D}^\prime]{(v,s) \hastype T}{}
              &
              \deduce[\mathcal{F}]{T \proves E \hastype S}{}
            }
          \end{equation*}

          We invert $\mathcal{D}^\prime$ to get
          %
          \begin{equation*}
            \infer{%
              \proves (v, s) \hastype T_1 \times T_2
            }{%
              \deduce[\mathcal{D}_1]{\proves v \hastype T_1}{}
              &
              \deduce[\mathcal{D}_2]{\proves s \hastype T_2}{}
            }
          \end{equation*}
          %
          and we discover $T = T_1 \times T_2$.
          Therefore $\mathcal{F} \deduc T_1 \times T_2 \proves E \hastype S$.

          We construct a derivation.
          %
          \begin{equation*}
            \infer{%
              \proves s ; (v, \bullet) E \hastype S
            }{%
              \deduce[\mathcal{D}_2]{\proves s \hastype T_2}{}
              &
              \infer{%
                T_2 \proves (v, \bullet) E \hastype S
              }{%
                \deduce[\mathcal{D}_1]{\proves v \hastype T_1}{}
                &
                \deduce[\mathcal{F}]{T_1 \times T_2 \proves E \hastype S}{}
              }
            }
          \end{equation*}
          %
          as required.

        \item[Case]
          $v_2; (v_1, \bullet) E \evalto (v_1, v_2) ; E$.

          Therefore $\mathcal{D}$ is of the form
          %
          \begin{equation*}
            \infer{
              \proves v_2; (v_1, \bullet) E \hastype S
            }{
              \deduce[\mathcal{D}^\prime]{\proves v_2 \hastype T}{}
              &
              \deduce[\mathcal{F}]{T \proves (v_1, \bullet) \hastype S}{}
            }
          \end{equation*}

          We invert $\mathcal{F}$ to deduce
          %
          \begin{equation*}
            \infer{
              T_2 \proves (v_1, \bullet) E \hastype S
            }{
              \deduce[\mathcal{F}_1]{\proves v_1 \hastype T_1}{}
              &
              \deduce[\mathcal{F}_2]{T_1 \times T_2 \proves E \hastype S}{}
            }
          \end{equation*}
          %
          and we discover that $T = T_2$.

          We construct a derivation
          %
          \begin{equation*}
            \infer{
              \proves (v_1, v_2) ; E \hastype S
            }{
              \infer{
                \proves (v_1, v_2) \hastype T_1 \times T_2
              }{
                \deduce[\mathcal{F}_1]{\proves v_1 \hastype T_1}{}
                &
                \deduce[\mathcal{D}^\prime]{\proves v_2 \hastype T_2}{}
              }
              &
              \deduce[\mathcal{F}_2]{T_1 \times T_2 \proves E \hastype S}{}
            }
          \end{equation*}
          %
          as required.

        \item[Case]
          $c_i\; t; E \evalto t; (c_i\;\bullet) E$.

          Therefore $\mathcal{D}$ is of the form
          %
          \begin{equation*}
            \infer{%
              \proves c_i\; t ; E \hastype S
            }{%
              \deduce[\mathcal{D}^\prime]{c_i\; t \hastype T}{}
              &
              \deduce[\mathcal{F}]{T \proves E \hastype S}{}
            }
          \end{equation*}

          We invert $\mathcal{D}^\prime$ to deduce
          %
          \begin{equation*}
            \infer{%
              \proves c_i \; t \hastype \mu X.\; D
            }{%
              \deduce[\mathcal{D}^\dprime]{%
                \proves t \hastype [\mu X.\; D/X]T_i
              }{}
            }
          \end{equation*}
          %
          and we discover that $T = \mu X.\; D$
          where
          $D = \langle c_1 \hastype T_1, \ldots, c_n \hastype T_n \rangle$.

          We construct a derivation
          %
          \begin{equation*}
            \infer{%
              \proves t ; (c_i\;\bullet) E \hastype S
            }{%
              \deduce[\mathcal{D}^\dprime]{%
                \proves t \hastype [\mu X.\; D / X]T_i%
              }{}
              &
              \infer{%
                [\mu X.\; D/X]T_i \proves (c_i\;\bullet)E\hastype S
              }{
                \deduce[\mathcal{F}]{\mu X.\; D \proves E \hastype S}{}
              }
            }
          \end{equation*}
          %
          as required.

        \item[Case]
          $v;(c_i\;\bullet)E \evalto c_i\; v ; E$.
          Therefore $\mathcal{D}$ is of the form

          %
          \begin{equation*}
            \infer{
              \proves v; (c_i\;\bullet) E \hastype S
            }{
              \deduce[\mathcal{D}^\prime]{
                \proves v \hastype T
              }{}
              &
              \deduce[\mathcal{F}]{
                T \proves (c_i\;\bullet) E \hastype S
              }{}
            }
          \end{equation*}

          We invert $\mathcal{F}$ to get
          %
          \begin{equation*}
            \infer{
              [\mu X.\; D/X]T_i \proves (c_i\;\bullet)E \hastype S
            }{
              \deduce[\mathcal{F}^\prime]{\mu X.\; D \proves E \hastype S}{}
            }
          \end{equation*}
          %
          and we discover $T = [\mu X.\; D]T_i$.

          Then we construct a derivation
          %
          \begin{equation*}
            \infer{
              \proves c_i\; v ; E \hastype S
            }{
              \infer{
                \proves c_i\; v \hastype \mu X. D
              }{
                \deduce[\mathcal{D}^\prime]{
                  \proves v \hastype [\mu X.\; D/X]T_i
                }{}
              }
              &
              \deduce[\mathcal{F}^\prime]{
                \mu X.\; D \proves E \hastype S
              }{}
            }
          \end{equation*}
          %
          as required.
      \end{description}

    \end{proof}
\end{enumerate}

\question{Let-polymorphism}

\begin{enumerate}
  \item
    We prove type preservation for the new language using a big-step semantics.

    First, we state and prove the necessary substitution lemma.
    %
    \begin{lem}[Substitution]
      If $\Gamma \proves N \hastypesc S$ and
      $\Gamma, x \hastypesc S, \Gamma^\prime \proves M \sim T$,
      then $\Gamma, \Gamma^\prime \proves [N/x]M \sim T$,
      where $\sim$ is either $\hastype$ or $\hastypesc$.
    \end{lem}

    \begin{proof}
      By lexicographic induction on $M$ and the size of the derivation
      $\mathcal{D} \deduc
      \Gamma, x \hastypesc S, \Gamma^\prime
      \proves M \sim T$.

      \begin{description}
        \item[Case]
          The derivation $\mathcal{D}$ is for a type schema, so we have
          %
          $
          \mathcal{D} \deduc
          \Gamma, x \hastypesc S, \Gamma^\prime \proves M \hastypesc S
          $
          %
          \begin{description}
            \item[Subcase]
              we have $\mathcal{D}$ of the form
              %
              \begin{equation*}
                \infer{
                  \Gamma, x \hastypesc S^\prime, \Gamma^\prime \proves M \hastypesc T
                }{
                  \deduce[\mathcal{D}^\prime]{
                    \Gamma, x \hastypesc S^\prime, \Gamma^\prime \proves M \hastype T
                  }{}
                }
              \end{equation*}

              By the induction hypothesis on $\mathcal{D}^\prime$ we deduce
              $\mathcal{E}^\prime \deduc \Gamma, x \hastypesc S^\prime, \Gamma^\prime \proves [N/x]M \hastype T$.
              
              We construct
              %
              \begin{equation*}
                \infer{
                  \Gamma, x \hastypesc S^\prime, \Gamma^\prime \proves [N/x]M \hastypesc T
                }{
                  \deduce[\mathcal{E}^\prime]{
                    \Gamma, x \hastypesc S^\prime, \Gamma^\prime \proves [N/x]M \hastype T
                  }{}
                }
              \end{equation*}
              %
              as required.

            \item[Subcase]
              we have $\mathcal{D}$ of the form
              %
              \begin{equation*}
                \infer{
                  \Gamma, x \hastypesc S^\prime, \Gamma^\prime \proves M \hastypesc \forall \alpha.\; S
                }{
                  \deduce[\mathcal{D}^\prime]{
                    \Gamma, x \hastypesc S^\prime, \Gamma^\prime \proves M \hastypesc S
                  }{}
                }
              \end{equation*}

              By the induction hypothesis on $\mathcal{D}^\prime$ we deduce
              $\mathcal{E}^\prime \deduc \Gamma, x \hastypesc S^\prime, \Gamma^\prime \proves [N/x]M \hastypesc S$.

              We construct
              %
              \begin{equation*}
                \infer{
                  \Gamma, x \hastypesc S^\prime, \Gamma^\prime \proves [N/x]M \hastypesc \forall \alpha.\; S
                }{
                  \Gamma, x \hastypesc S^\prime, \Gamma^\prime \proves [N/x]M \hastypesc S
                }
              \end{equation*}
              %
              as required.
          \end{description}

        \item[Case]
          The derivation $\mathcal{D}$ is for a type, so we have
          $
          \mathcal{D} \deduc \Gamma, x \hastypesc S, \Gamma^\prime
          \proves M \hastype T
          $
          \begin{description}
            \item[Subcase]
              $M = y$ is a variable other than $x$.

              Then we have
              %
              $\Gamma, x \hastypesc S, \Gamma^\prime \proves y \hastype T$.

              We can strengthen the context, since $x$ is unused, giving us
              %
              $\Gamma, \Gamma^\prime \proves y \hastype T$.

              Finally, by definition of substitution we get
              %
              $\Gamma, \Gamma^\prime \proves [N/x]y \hastype T$,
              %
              as required.

            \item[Subcase]
              $M = x$ is the variable we want to substitute.

              Then we have by assumption
              %
              $\Gamma, x \hastypesc S, \Gamma^\prime \proves x \hastype T$,
              %
              which must be obtained by the variable rule, so by inversion we
              obtain $\mathcal{E} \deduc S \inst T$.

              By a lemma\footnotemark{} we deduce from $\mathcal{E}$ that
              $\Gamma \proves N \hastype T$, and by weakening that
              $\Gamma, \Gamma^\prime \proves N \hastype T$.
              %
              \footnotetext{%
                \emph{If $\Gamma \proves N \hastypesc S$ and $S \inst T$, then
                $\Gamma \proves N \hastype T$.}
                This is easy to prove by induction on the instantiation
                derivation.%
              }

              Finally, by definition of substitution we get
              %
              $\Gamma, \Gamma^\prime \proves [N/x]x \hastype T$.

            \item[Subcase]
              $M = \lambda y. M^\prime$ is an abstraction.

              Without loss of generality, we assume $y \neq x$ and $y$ is not free
              in $N$; this can be achieved by renaming $y$ in the abstraction if
              necessary.

              Then we have
              %
              $\mathcal{D} \deduc \Gamma, x \hastypesc S, \Gamma^\prime
              \proves \lambda y. M^\prime \hastype T$.

              By inverting $\mathcal{D}$ we deduce
              %
              \begin{equation*}
                \infer{
                  \Gamma, x\hastypesc S, \Gamma^\prime
                  \proves \lambda y. M^\prime \hastype T_1 \to T_2
                }{
                  \deduce[\mathcal{D}^\prime]{
                    \Gamma, x\hastypesc S, y\hastypesc T_1, \Gamma^\prime
                    \proves M^\prime \hastype T_2
                  }{}
                }
              \end{equation*}
              %
              and we discover that $T = T_1 \to T_2$.

              By the induction hypothesis on $\mathcal{D}^\prime$ we deduce that
              $\mathcal{E}^\prime \deduc \Gamma, y \hastypesc T_1, \Gamma^\prime
              \proves [N/x]M^\prime \hastype T_2$.

              We construct a derivation $\mathcal{E}$
              %
              \begin{equation*}
                \infer{
                  \Gamma, \Gamma^\prime
                  \proves \lambda y. [N/x]M^\prime \hastype T_1 \to T_2
                }{
                  \deduce[\mathcal{E}^\prime]{
                    \Gamma, y\hastypesc T_1, \Gamma^\prime
                    \proves [N/x]M^\prime \hastype T_2
                  }{}
                }
              \end{equation*}

              Since $y \neq x$ and we assume $y$ is not free in $N$, we hoist the
              substitution over the $\lambda$, so
              $\Gamma, \Gamma^\prime
              \proves [N/x]\lambda y. M^\prime \hastype T_1 \to T_2$,
              as required.

            \item[Subcase]
              $M = M_1 M_2$ is an application.

              This case is uninteresting, since it just applies the induction
              hypothesis twice and repackages the results, so I won't write out the
              full proof.

            \item[Subcase]
              $M = \mathtt{let}\; y = M_1 \;\mathtt{in}\; M_2$
              is a let construct.

              We assume without loss of generality that $y$ is not free in $N$ and
              that $x \neq y$, as we can rename variables.

              Then we have
              $\mathcal{D} \deduc
              \Gamma, x \hastypesc S, \Gamma^\prime
              \proves \mathtt{let}\; y = M_1 \;\mathtt{in}\; M_2
              \hastype T$.

              By inverting $\mathcal{D}$ we deduce that
              %
              \begin{equation*}
                \infer{
                  \Gamma, x \hastypesc S, \Gamma^\prime
                  \proves \mathtt{let}\; y = M_1 \;\mathtt{in}\; M_2
                  \hastype T
                }{
                  \deduce[\mathcal{D}_1]{
                    \Gamma, x \hastypesc S, \Gamma^\prime
                    \proves M_1 \hastypesc S^\prime
                  }{}
                  &
                  \deduce[\mathcal{D}_2]{
                    \Gamma, x \hastypesc S, \Gamma^\prime, y \hastypesc S^\prime
                    \proves M_2 \hastype T
                  }{}
                }
              \end{equation*}

              By the induction hypothesis on $\mathcal{D}_1$ we deduce
              %
              \begin{equation*}
                \deduce[\mathcal{E}_1]{
                  \Gamma, \Gamma^\prime \proves [N/x]M_1 \hastypesc S^\prime
                }{}
              \end{equation*}

              By the induction hypothesis on $\mathcal{D}_2$ we deduce
              %
              \begin{equation*}
                \deduce[\mathcal{E}_2]{
                  \Gamma, \Gamma^\prime, y \hastypesc S^\prime \proves [N/x]M_2
                  \hastype T
                }{}
              \end{equation*}

              We construct a derivation
              %
              \begin{equation*}
                \infer{
                  \Gamma, \Gamma^\prime
                  \proves \mathtt{let}\; y = [N/x]M_1 \;\mathtt{in}\; [N/x]M_2
                  \hastype T
                }{
                  \deduce[\mathcal{E}_1]{
                    \Gamma, \Gamma^\prime \proves [N/x]M_1 \hastypesc S^\prime
                  }{}
                  &
                  \deduce[\mathcal{E}_2]{
                    \Gamma, \Gamma^\prime, y \hastypesc S^\prime \proves [N/x]M_2
                    \hastype T
                  }{}
                }
              \end{equation*}

              Since we assume that $y$ is not free in $N$ and that $y \neq x$, we
              can hoist the substitutions out of the \texttt{let} to get
              $
              \Gamma, \Gamma^\prime
              \proves [N/x] \mathtt{let}\; y = M_1 \;\mathtt{in}\; M_2
              \hastype T
              $
              by the definition of substitution, as required.
          \end{description}
      \end{description}
    \end{proof}

    \begin{prop}[Type preservation]
      If $\Gamma \proves t \sim T$ and $t \evalbig v$,
      then $\Gamma \proves v \sim T$.
    \end{prop}

    \begin{proof}
      By lexicographic induction on the evaluation $t \evalbig v$
      and the typing $\Gamma \proves t \sim T$.

      We need to be able to use the induction hypothesis on the schema typing
      judgment (in the case for \texttt{let}), so we generalize the theorem to
      account for this. We proceed in a similar fashion as for the substitution
      lemma, by first checking whether the typing is for a schema or for a
      type: in the former case we simply apply the induction hypothesis until
      we hit the bottom and switch back to the main induction on the
      evaluation.

      We will not write out the cases for the schema typing since they are
      tedious and essentially the same as for the substitution lemma, and
      instead we will focus on the evaluation derivation, with ordinary typing.

      \begin{description}
        \item[Case]
          $\infer{\lambda x. t \evalbig \lambda x. t}{}$

          Then there is nothing to do, since $t = v$.

        \item[Case]
          the evaluation is of the form
          %
          \begin{equation*}
            \infer{
              t\; s \evalbig v
            }{
              \deduce[\mathcal{E}_1]{t \evalbig \lambda x. t^\prime}{}
              &
              \deduce[\mathcal{E}_2]{s \evalbig v^\prime}{}
              &
              \deduce[\mathcal{E}_3]{[v^\prime/x]t^\prime \evalbig v}{}
            }
          \end{equation*}

          We have $\Gamma \proves t\; s \hastype T$ by assumption.
          By inversion we get
          %
          \begin{equation*}
            \infer{
              \Gamma \proves t\; s \hastype T
            }{
              \deduce[\mathcal{D}_1]{
                \Gamma \proves t \hastype T^\prime \to T
              }{}
              &
              \deduce[\mathcal{D}_2]{
                \Gamma \proves s \hastype T^\prime
              }{}
            }
          \end{equation*}

          By the induction hypothesis on $\mathcal{D}_1$ and $\mathcal{E}_1$
          deduce
          $\mathcal{F}_1 \deduc
          \Gamma \proves \lambda x.t^\prime \hastype T^\prime \to T$.

          By the induction hypothesis on $\mathcal{D}_2$ and $\mathcal{E}_2$
          deduce
          $\mathcal{F}_2 \deduc
          \Gamma \proves v^\prime \hastype T^\prime$.

          By inverting $\mathcal{F}_1$,
          deduce
          $\mathcal{F}_1^\prime \deduc
          \Gamma, x \hastypesc T^\prime \proves t^\prime \hastype T$.

          Construct
          $\mathcal{F}_2^\prime \deduc
          \Gamma \proves v^\prime \hastypesc T^\prime$ by
          applying rule $\mathqd{tpsc-base}$ to $\mathcal{F}_2$.

          By the substitution lemma on $\mathcal{F}_2^\prime$ and
          $\mathcal{F}_1^\prime$, deduce
          $\mathcal{F} \deduc
          \Gamma \proves [v^\prime/x]t^\prime \hastype T$.

          By the induction hypothesis on $\mathcal{F}$ and $\mathcal{E}_3$,
          deduce that $\Gamma \proves v \hastype T$ as required.

        \item[Case]
          the evaluation is of the form
          %
          \begin{equation*}
            \infer{
              \mathtt{let}\; x = t_1 \;\mathtt{in}\; t_2 \evalbig v
            }{
              \deduce[\mathcal{E}_1]{t_1 \evalbig v^\prime}{}
              &
              \deduce[\mathcal{E}_2]{[v^\prime/x]t_2 \evalbig v}{}
            }
          \end{equation*}

          We have
          $\Gamma \proves \mathtt{let}\; x = t_1 \;\mathtt{in}\; t_2 \hastype
          T$ by assumption.
          By inversion, deduce
          %
          \begin{equation*}
            \infer{
              \Gamma \proves
              \mathtt{let}\; x = t_1 \;\mathtt{in}\; t_2 \hastype T
            }{
              \deduce[\mathcal{D}_1]{\Gamma \proves t_1 \hastypesc S}{}
              &
              \deduce[\mathcal{D}_2]{
                \Gamma, x \hastypesc S \proves t_2 \hastype T
              }{}
            }
          \end{equation*}

          By the induction hypothesis on $\mathcal{E}_1$ and $\mathcal{D}_1$,
          deduce $\mathcal{F}^\prime \deduc v^\prime \hastypesc S$.

          By the substitution lemma on $\mathcal{F}^\prime$ and
          $\mathcal{D}_2$, deduce
          $\mathcal{D}^\prime \deduc
          \Gamma \proves [v^\prime/x]t_2 \hastype T$.

          By the induction hypothesis on $\mathcal{E}_2$ and
          $\mathcal{D}^\prime$, deduce
          $\mathcal{F} \deduc
          \Gamma \proves v \hastype T$, as required.
      \end{description}
    \end{proof}

  \item
    \newcommand{\phts}{\proves_{\hastypesc}}
    %
    We state the theorem that asserts the equivalence between these two type
    assignment systems.

    Note that we write $\phts$ for the system involving type schemas
    and simply write $\proves$ for the system involving substitutions.

    \begin{prop}
      $\Gamma \phts t \hastype T$ if and only if $\Gamma^- \proves t \hastype T$.
    \end{prop}
    %
    where $\Gamma^-$ is defined by dropping any schema bindings that involve
    quantifications.
    %
    \begin{align*}
      (\cdot)^- &= \cdot \\
      (\Gamma, x \hastypesc T)^- &= \Gamma^-, x \hastype T \\
      (\Gamma, x \hastypesc \forall \alpha.\; S)^- &= \Gamma^-
    \end{align*}

  \item
    We prove the easy direction of this equivalence.

    \begin{proof}
      Suppose
      $\mathcal{D} \deduc \Gamma \proves [\sigma]t \hastype T$.
      We want to show that $\Gamma, \Gamma^\prime \phts t \hastype T$, where
      $\sigma \hastype \Gamma^\prime$ is a substitution for polymorphic
      variables.

      We proceed by induction on $\mathcal{D}$.

      \begin{description}
        \item[Case]
          $\mathcal{D}$ is of the form
          %
          \begin{equation*}
            \infer{
              \Gamma \proves [\sigma] \mathtt{let}\; x = t_1 \;\mathtt{in}\; t_2
              \hastype T
            }{
              \deduce[\mathcal{D}_1]{\Gamma \proves [\sigma]t_1 \hastype T_1}{}
              &
              \deduce[\mathcal{D}_2]{\Gamma \proves [t_1/x,\;\sigma]t_2 \hastype T}{}
            }
          \end{equation*}

          By the induction hypothesis on $\mathcal{D}_1$, deduce
          $\mathcal{F}_1 \deduc \Gamma, \Gamma^\prime \phts T_1$.

          We quantify over all free type variables in $T_1$ (by applying rule
          $\mathqd{tpsc-base}$ followed by repeatedly applying rule
          $\mathqd{tpsc-all}$) to deduce
          $\mathcal{F}_1^\prime \deduc \Gamma, \Gamma^\prime \phts S$

          By the induction hypothesis on $\mathcal{D}_2$, deduce
          $\mathcal{F}_2 \deduc
          \Gamma, \Gamma^\prime, x \hastypesc S \phts t_2 \hastype T$

          Then we construct the derivation
          %
          \begin{equation*}
            \infer{
              \Gamma, \Gamma^\prime
              \phts \mathtt{let}\; x = t_1 \;\mathtt{in}\; t_2 \hastype T
            }{
              \deduce[\mathcal{F}_1]{\Gamma, \Gamma^\prime \phts t_1 \hastypesc S}{}
              &
              \deduce[\mathcal{F}_2]{
                \Gamma, \Gamma^\prime, x \hastypesc S \phts t_2 \hastype T
              }{}
            }
          \end{equation*}
          %
          as required.
      \end{description}
    \end{proof}

  \item
    We consider the harder direction of the equivalence.

    \begin{proof}
      Suppose $\mathcal{D} \deduc \Gamma \phts t \hastype T$.
      We want to show that $\Gamma \proves t \hastype T$.
      We proceed by induction on $\mathcal{D}$.

      \begin{description}
        \item[Case]
          $\mathcal{D}$ is of the form
          %
          \begin{equation*}
            \infer{
              \Gamma \phts \mathtt{let}\; x = t_1 \;\mathtt{in}\; t_2
            }{
              \deduce[\mathcal{D}_1]{\Gamma \phts t_1 \hastypesc S}{}
              &
              \deduce[\mathcal{D}_2]{
                \Gamma, x \hastypesc S \phts t_2 \hastype T
              }{}
            }
          \end{equation*}

          We would like to proceed by using the substitution lemma to deduce
          $\Gamma \phts [t_1/x]t_2 \hastype T$ followed by the induction
          hypothesis to complete the proof.
          %
          However, this use of the induction hypothesis would not be justified,
          since we can't say anything about the size of the derivation that
          results from the use of the substitution lemma.

          It isn't clear how exactly to proceed in this case.
      \end{description}
    \end{proof}
\end{enumerate}

\end{document}
