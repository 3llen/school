\documentclass[11pt,letterpaper]{article}

\author{Jacob Thomas Errington (260636023)}
\title{Assignment \#3\\Language-based security -- COMP 523}
\date{Thursday, 2 November 2017}

\usepackage[geometry]{jakemath}
\renewcommand{\R}{\mathcal{R}}

\newcommand{\bool}{\mathtt{bool}}
\newcommand{\True}{\mathtt{true}}
\newcommand{\False}{\mathtt{false}}
\newcommand{\ifthenelse}[3]{%
  \mathtt{if}\;#1\;\mathtt{then}\;#2\;\mathtt{else}\;#3%
}

\begin{document}

\maketitle

\section{Weak normalization}

\begin{enumerate}
  \item
    We define the reducibility relation for booleans similarly as for unit,

    \begin{equation*}
      t \in \R_{\bool} \quad\text{if $t$ halts}
    \end{equation*}

  \item
    We extend the backwards closure lemma for booleans.

    \begin{lem}[Backwards closure]
      If $t \evalto t^\prime$ and $t^\prime \in \R_T$, then $t \in \R_T$.
    \end{lem}

    \begin{proof}
      By induction on the type $T$.
      %
      \begin{description}
        \item[Case] $T = \bool$.

          We know that $t^\prime \in \R_\bool$ by assumption.
          Hence there exists a value $v$ such that $t^\prime \evalto^* v$,
          since $t^\prime$ halts.
          By the definition of the multistep evaluation,
          we deduce that $t \evalto^* v$ as well.
          Therefore by definition $t$ halts and $t \in \R_\bool$.
      \end{description}
    \end{proof}

  \item
    We extend the main lemma for booleans.

    \begin{lem}[Main lemma]
      If $\Gamma \proves t \hastype T$
      and $\sigma \in \R_\Gamma$,
      then $[\sigma]t \in \R_T$.
    \end{lem}

    \begin{proof}
      By induction on the typing derivation
      $\mathcal{D} : \Gamma \proves t \hastype T$.
      %
      \begin{description}
        \item[Case] $\mathcal{D}$ is of the form
          \begin{equation*}
            \infer{\Gamma \proves \True \hastype \bool}{}
          \end{equation*}
          %
          Notice that $\True$ is already a value, so it halts by reflexivity
          rule of multistep evaluation. Therefore $\True \in \R_\bool$.
          Finally, substitution does not affect values,
          so $[\sigma]\True \in \R_\bool$ as required.

        \item[Case] for $\False$ follows the same argument.

        \item[Case] $\mathcal{D}$ is of the form
          \begin{equation*}
            \infer{%
              \Gamma \proves \ifthenelse{t}{s_1}{s_2} \hastype S%
            }{%
              \deduce[\mathcal{D}^\prime]{\Gamma \proves t \hastype \bool}{}
              &
              \deduce[\mathcal{D}_1]{\Gamma \proves s_1 \hastype S}{}
              &
              \deduce[\mathcal{D}_2]{\Gamma \proves s_2 \hastype S}{}
            }
          \end{equation*}
          %
          % WTS: [sigma](if t then s1 else s2) in R_S
          %
          By the induction hypothesis applied to $\mathcal{D}^\prime$,
          we deduce that $[\sigma]t \in \R_\bool$.
          By definition of $\R_\bool$, we have that $[\sigma]t$ halts, so
          there exists a value $v$ such that $[\sigma]t \evalto^* v$.
          By (Extended) Type Preservation\footnotemark,
          $\Gamma \proves v \hastype \bool$.
          Now we analyze $v$, for which there are only two cases having the
          correct type.
          %
          \footnotetext{%
            By ``extended'' we mean that we generalize the theorem to work on
            multistep evaluation. This is easy to prove: if you can take one
            step preserving types, then of course you can take $n$ steps
            preserving types.
          }
          %
          \begin{description}
            \item[Subcase] $v = \True$.

              By the induction hypothesis applied to $\mathcal{D}_1$,
              we deduce that $[\sigma]s_1 \in \R_S$.
              We have the evaluation rule
              $\ifthenelse{\True}{[\sigma]s_1}{[\sigma]s_2} \evalto [\sigma]s_1$,
              so by the Backwards Closure Lemma,
              we deduce that
              $\ifthenelse{\True}{[\sigma]s_1}{[\sigma]s_2} \in \R_S$.

            \item[Subcase] $v = \False$ follows the same argument, except that
              we pick the other basic evaluation rule for $\mathtt{if}$.
          \end{description}
          %
          Now we have that
          $\ifthenelse{v}{[\sigma]s_1}{[\sigma]s_2} \in \R_S$.

          We assume the following congruence lemma.
          %
          \begin{quote}
            if $t \evalto^* t^\prime$, then
            $\ifthenelse{t}{s_1}{s_2}
            \evalto^* \ifthenelse{t^\prime}{s_2}{s_2}$.
          \end{quote}

          By this congruence lemma we deduce that
          \begin{equation*}
            \ifthenelse{[\sigma]t}{[\sigma]s_1}{[\sigma]s_2]}
            \evalto^*
            \ifthenelse{v}{[\sigma]s_1}{[\sigma]s_2}
          \end{equation*}

          We assume the following Extended Backwards Closure lemma:
          \begin{quote}
            if $t \evalto^* t^\prime$ and $t^\prime \in \R_T$,
            then $t \in \R_T$.
          \end{quote}

          By the Extended Backwards Closure lemma, we deduce that
          \begin{equation*}
            \ifthenelse{[\sigma]t}{[\sigma]s_1}{[\sigma]s_2} \in \R_S
          \end{equation*}

          Finally, by the definition of substitution, we factor out $\sigma$
          to deduce that
          %
          \begin{equation*}
            [\sigma](\ifthenelse{t}{s_1}{s_2}) \in \R_S
          \end{equation*}
          %
          as required.
      \end{description}
    \end{proof}
\end{enumerate}

\section{Type safety using logical relations}

\newcommand{\safe}{\ensuremath{\;\text{safe}}}

\begin{thm}[Type safety]
  If $\proves t \hastype T$, then $t \safe$.
\end{thm}

\begin{proof}
  Suppose $\proves t \hastype T$.
  Take an arbitrary $s$ and suppose $t \evalto^* s$.
  By (Extended) Type Preservation,
  we deduce that $\proves s \hastype T$.
  By Progress,
  either $s$ is a value
  or there exists $s^\prime$ such that $s \evalto s^\prime$.
\end{proof}

\renewcommand{\S}{\mathcal{S}}
\renewcommand{\V}{\mathcal{V}}

\begin{prop}
  If $t \in \S_T$, then $t \safe$.
\end{prop}

\begin{proof}
  Suppose $t \in \S_T$, i.e. for all $s$, if $t \evalto^* s$ and $s$ is
  irreducible, then $s \in \V_T$.
  We want to show that for all $s$, if $t \evalto^* s$, then either $s$ is a
  value or there exists $s^\prime$ so that $s \evalto s^\prime$.

  Take arbitrary $s$ and suppose $t \evalto^* s$.
  We deduce that if $s$ is irreducible, then $s \in \V_T$.
  We want to show that either $s$ is a value or there exists $s^\prime$ so that
  $s \evalto s^\prime$.
  We analyze whether $s$ is irreducible.
  \begin{description}
    \item[Case] $s$ is irreducible.
      Then we deduce that $s \in \V_T$.
      By definition, the possible elements of $\V_T$ are the booleans $\true$
      and $\false$ and the lambda abstractions.
      By definition, these terms are values, so we're done.
    \item[Case] $s$ is not irreducible.
      Then there exists $s^\prime$ so that $s \evalto s^\prime$, so we're done.
  \end{description}
\end{proof}

Next we define a notion of safe substitutions.
Intuitively, a substitution $\sigma$ is safe for a context $\Gamma$ if it sends
each variable to a safe expression.
We formalize this inductively.
\newcommand{\safefor}{\;\text{safe for}\;}
%
\begin{equation*}
  \infer{\cdot \safefor \cdot}{}
  \quad
  \infer{(\sigma, v/x) \safefor \Gamma, x:T}{
    \sigma \safefor \Gamma
    &
    v \in \V_T
  }
\end{equation*}

\begin{prop}
  If $\Gamma \proves t \hastype T$ and $\sigma \safefor \Gamma$,
  then $[\sigma]t \in \S_T$.
\end{prop}

\begin{proof}
  By induction on $\mathcal{D} : \Gamma \proves t \hastype T$.
  %
  \begin{description}
    \item[Case] $\mathcal{D}$ is of the form
      \begin{equation*}
        \infer{\Gamma \proves \true \hastype T}{}
      \end{equation*}

      By assumption we have $\sigma \safefor \Gamma$.
      We want to show that $[\sigma]\true \in \S_\bool$.

      Take arbitrary $s$ and suppose both that $\true \evalto^* s$ and that
      $s$ is irreducible.
      Note that by value soundness $s = \true$.
      By definition of $\V_\bool$, we have that $s \in \V_\bool$.
      Therefore $\true \in \S_\bool$, as required.

    \item[Case] for $\Gamma \proves \false \hastype \bool$ is the same.

    \item[Case] $\mathcal{D}$ is of the form
      %
      \renewcommand{\t}{\ensuremath{\lambda x \hastype S . t}}
      \begin{equation*}
        \infer{\Gamma \proves \t \hastype S \to T}{
          \deduce[\mathcal{D}^\prime]{
            \Gamma, x \hastype S \proves t \hastype T
          }{}
        }
      \end{equation*}

      By assumption we have $\sigma \safefor \Gamma$.
      We want to show that
      $[\sigma](\t) \in \S_{S \to T}$.

      Take arbitrary $s$ and suppose both that $[\sigma](\t) \evalto^* s$ and
      that $s$ is irreducible.
      Applying a substitution to an abstraction still gives an abstraction, so
      we deduce that $s = [\sigma](\t)$ by value soundness.
      Now we need to show that $s \in \V_{S \to T}$.
      Take arbitrary $v \in \V_S$.
      Define $\sigma^\prime = (\sigma, v/x)$.
      Notice that $\sigma^\prime \safefor \Gamma, x \hastype S$.
      We invoke the induction hypothesis on $\mathcal{D}^\prime$ and
      $\sigma^\prime$ to deduce $[\sigma^\prime]t \in \S_T$.
      We expand $\sigma^\prime$ to get that $[v/x][\sigma]t \in \S_T$.
      We deduce that $(\lambda x \hastype S. [\sigma]t) \in \V_{S \to T}$.
      By definition of $\sigma$ we hoist it out of the $\lambda$ to get
      $[\sigma](\t) \in \V_{S \to T}$.
      Since $[\sigma](\t) = s$, we rewrite as $s \in \V_{S \to T}$,
      as required.
      We deduce that $[\sigma](\t) \in \S_{S \to T}$, as required.

    \item[Case] $\mathcal{D}$ is of the form
      \begin{equation*}
        \infer{\Gamma \proves t s \hastype T}{
          \Gamma \proves t \hastype S \to T
          &
          \Gamma \proves s \hastype S
        }
      \end{equation*}

      By assumption, we have $\sigma \safefor \Gamma$.
      By the induction hypothesis we immediately deduce that
      $[\sigma]t \in \S_{S \to T}$ and $[\sigma]s \in \S_S$.

      Take arbitrary $k$ and suppose $[\sigma]t [\sigma]s \evalto^* k$ and $k$
      is irreducible.
      We want to show that $k \in \V_T$.

      By lemma \ref{lem:midway-reducible-1} we deduce that there exists some
      irreducible $u$ such that
      $[\sigma]t [\sigma]s \evalto^* [\sigma]t u \evalto^* k$.
      It's clear that from this derivation we can extract a derivation that
      $[\sigma]s \evalto^* u$.
      Since $[\sigma]s \in \S_S$, we deduce that $u \in \V_S$.

      By lemma \ref{lem:midway-reducible-2} we deduce that there exists some
      irreducible $v$ such that
      $[\sigma]t u \evalto^* v u \evalto^* k$.
      Again we can extract a derivation that $[\sigma]t \evalto^* v$.
      Since $[\sigma]t \in \S_{S \to T}$, we deduce that $v \in \V_{S \to T}$.

      By a canonical form argument,
      we deduce that $v$ is of the form $\lambda x \hastype S. b$
      Therefore the next step in $v u \evalto^* k$ must be a $\beta$ rule, so
      $v u \evalto [u/x]b \evalto^* k$.

      Note that since $\lambda x \hastype S.b \in \V_{S \to T}$
      and $u \in \V_S$,
      we deduce that $[u/x]b \in \S_T$.
      Since $[u/x]b \evalto^* k$ and $k$ is irreducible, we deduce that
      $k \in \V_T$, as required.
  \end{description}

\end{proof}

\begin{lem}
  \label{lem:midway-reducible-1}
  Suppose $t s \evalto^* u$ and $u$ is irreducible.
  Then there exists some $v$ that is irreducible such that
  $t s \evalto^* t v \evalto^* u$.
\end{lem}

\begin{proof}
  By induction on $\mathcal{D} : t s \evalto^* u$.

  \begin{description}
    \item[Case] $s$ is irreducible.

      Then we're done because $s$ is the irreducible $v$ we were looking
      for and $t s \evalto^* t s \evalto^* u$.

    \item[Subcase] $s^\prime$ is not irreducible.

      Then we analyze $\mathcal{D}$.
      Since $t s$ is not irreducible, the reflexive case is impossible, so we
      have only the single-step case, and the only single-step rule that could
      apply is the second congruence rule for applications.
      Therefore there exists some $s^\prime$ such that $t s \evalto t s^\prime$
      and we have a smaller derivation $\mathcal{D}^\prime : t s^\prime
      \evalto^* u$.

      By the induction hypothesis on $\mathcal{D}^\prime$, we deduce that there
      exists some $v$ that is irreducible such that $t s^\prime \evalto^* t v
      \evalto^* u$.
      Since $t s \evalto t s^\prime \evalto^* t v \evalto^* u$,
      we conclude that $t s \evalto^* t v \evalto^* u$, as required.
  \end{description}
\end{proof}

\begin{lem}
  \label{lem:midway-reducible-2}
  Suppose $t v \evalto^* u$ and $u$ and $v$ are irreducible.
  Then there exists some $f$ that is irreducible such that
  $t v \evalto^* f v \evalto^* u$.
\end{lem}

\begin{proof}
  By induction on $\mathcal{D} : t v \evalto^* u$.

  \begin{description}
    \item[Case] $t$ is irreducible.

      Then we're done because $t$ is the irreducible $f$ we were looking for
      and $t v \evalto^* f v \evalto^* u$.

    \item[Case] $t$ is not irreducible.

      Then since $t v$ is not irreducible, the only applicable case for
      $\mathcal{D}$ is the single-step rule, and the only applicable
      single-step rule is the first congruence rule for applications.
      Therefore ther exists some $t^\prime$ such that $t v \evalto t^\prime v$
      and we have a smaller derivation
      $\mathcal{D}^\prime : t^\prime v \evalto^* u$.

      By the induction hypothesis on $\mathcal{D}^\prime$, we deduce that there
      exists some $f$ that is irreducible such that
      $t^\prime v \evalto^* f v \evalto^* u$.
      We stitch back on the step $t v \evalto t^\prime v$ to deduce that
      $t v \evalto^* f v \evalto u$ as required.
  \end{description}
\end{proof}

\end{document}
