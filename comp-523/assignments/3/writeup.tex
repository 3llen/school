\documentclass[11pt,letterpaper]{article}

\author{Jacob Thomas Errington (260636023)}
\title{Assignment \#3\\Language-based security -- COMP 523}
\date{Thursday, 2 November 2017}

\usepackage[geometry]{jakemath}
\renewcommand{\R}{\mathcal{R}}

\newcommand{\bool}{\mathtt{bool}}
\newcommand{\True}{\mathtt{true}}
\newcommand{\False}{\mathtt{false}}
\newcommand{\ifthenelse}[3]{%
  \mathtt{if}\;#1\;\mathtt{then}\;#2\;\mathtt{else}\;#3%
}

\begin{document}

\maketitle

\section{Weak normalization}

\begin{enumerate}
  \item
    We define the reducibility relation for booleans similarly as for unit,

    \begin{equation*}
      t \in \R_{\bool} \quad\text{if $t$ halts}
    \end{equation*}

  \item
    We extend the backwards closure lemma for booleans.

    \begin{lem}[Backwards closure]
      If $t \evalto t^\prime$ and $t^\prime \in \R_T$, then $t \in \R_T$.
    \end{lem}

    \begin{proof}
      By induction on the type $T$.
      %
      \begin{description}
        \item[Case] $T = \bool$.

          We know that $t^\prime \in \R_\bool$ by assumption.
          Hence there exists a value $v$ such that $t^\prime \evalto^* v$,
          since $t^\prime$ halts.
          By the definition of the multistep evaluation,
          we deduce that $t \evalto^* v$ as well.
          Therefore by definition $t$ halts and $t \in \R_\bool$.
      \end{description}
    \end{proof}

  \item
    We extend the main lemma for booleans.

    \begin{lem}[Main lemma]
      If $\Gamma \proves t \hastype T$
      and $\sigma \in \R_\Gamma$,
      then $[\sigma]t \in \R_T$.
    \end{lem}

    \begin{proof}
      By induction on the typing derivation
      $\mathcal{D} : \Gamma \proves t \hastype T$.
      %
      \begin{description}
        \item[Case] $\mathcal{D}$ is of the form
          \begin{equation*}
            \infer{\Gamma \proves \True \hastype \bool}{}
          \end{equation*}
          %
          Notice that $\True$ is already a value, so it halts by reflexivity
          rule of multistep evaluation. Therefore $\True \in \R_\bool$.
          Finally, substitution does not affect values,
          so $[\sigma]\True \in \R_\bool$ as required.

        \item[Case] for $\False$ follows the same argument.

        \item[Case] $\mathcal{D}$ is of the form
          \begin{equation*}
            \infer{%
              \Gamma \proves \ifthenelse{t}{s_1}{s_2} \hastype S%
            }{%
              \deduce[\mathcal{D}^\prime]{\Gamma \proves t \hastype \bool}{}
              &
              \deduce[\mathcal{D}_1]{\Gamma \proves s_1 \hastype S}{}
              &
              \deduce[\mathcal{D}_2]{\Gamma \proves s_2 \hastype S}{}
            }
          \end{equation*}
          %
          % WTS: [sigma](if t then s1 else s2) in R_S
          %
          By the induction hypothesis applied to $\mathcal{D}^\prime$,
          we deduce that $[\sigma]t \in \R_\bool$.
          By definition of $\R_\bool$, we have that $[\sigma]t$ halts, so
          there exists a value $v$ such that $[\sigma]t \evalto^* v$.
          By type preservation, $\Gamma \proves v \hastype \bool$.
          Now we analyze $v$, for which there are only two cases having the
          correct type.
          %
          \begin{description}
            \item[Subcase] $v = \True$.

              By the induction hypothesis applied to $\mathcal{D}_1$,
              we deduce that $[\sigma]s_1 \in \R_S$.
              We have the evaluation rule
              $\ifthenelse{\True}{[\sigma]s_1}{[\sigma]s_2} \evalto [\sigma]s_1$,
              so by the Backwards Closure Lemma,
              we deduce that
              $\ifthenelse{\True}{[\sigma]s_1}{[\sigma]s_2} \in \R_S$.

            \item[Subcase] $v = \False$ follows the same argument, except that
              we pick the other basic evaluation rule for $\mathtt{if}$.
          \end{description}
          %
          Now we have that
          $\ifthenelse{v}{[\sigma]s_1}{[\sigma]s_2} \in \R_S$.

          We assume the following congruence lemma.
          %
          \begin{quote}
            if $t \evalto^* t^\prime$, then
            $\ifthenelse{t}{s_1}{s_2}
            \evalto^* \ifthenelse{t^\prime}{s_2}{s_2}$.
          \end{quote}

          By this congruence lemma we deduce that
          \begin{equation*}
            \ifthenelse{[\sigma]t}{[\sigma]s_1}{[\sigma]s_2]}
            \evalto^*
            \ifthenelse{v}{[\sigma]s_1}{[\sigma]s_2}
          \end{equation*}

          We assume the following Extended Backwards Closure lemma:
          \begin{quote}
            if $t \evalto^* t^\prime$ and $t^\prime \in \R_T$,
            then $t \in \R_T$.
          \end{quote}

          By the Extended Backwards Closure lemma, we deduce that
          \begin{equation*}
            \ifthenelse{[\sigma]t}{[\sigma]s_1}{[\sigma]s_2} \in \R_S
          \end{equation*}

          Finally, by the definition of substitution, we factor out $\sigma$
          to deduce that
          %
          \begin{equation*}
            [\sigma](\ifthenelse{t}{s_1}{s_2}) \in \R_S
          \end{equation*}
          %
          as required.
      \end{description}
    \end{proof}
\end{enumerate}

\end{document}
