\documentclass[11pt,letterpaper]{article}

\usepackage[geoemtry]{jakemath}

\author{Jacob Thomas Errington (260636023)}
\title{Assignment \#1\\Language-based security -- COMP 523}
\date{20 September 2017}

\begin{document}

\maketitle

\section{Order of evaluation}

The proposed new evaluation rules for E-PRED-SUCC and E-ISZERO-SUCC affect the
order of evaluation. In particular, rather than require numerical values (i.e.
that the inside be evaluated first), the new rules allow unevaluated terms to
be manipulated. Second, the new rule for E-PRED-SUCC in fact changes the order
of the \texttt{pred} and \texttt{succ} operations.

\begin{description}
    \newcommand{\Succ}{\operatorname{\mathtt{succ}}}
    \newcommand{\Pred}{\operatorname{\mathtt{pred}}}
    \item[E-PRED-SUCC.]
        Consider the term \texttt{succ (pred 0)}.
        Using the new, non-strict E-PRED-SUCC rule, we can cancel the
        \texttt{succ pred}, leaving us with \texttt{0}.
        However, we can \emph{also} (and this is where we have nondeterminism)
        evaluate inside the \texttt{succ} to reduce \texttt{pred 0} to
        \texttt{0}, followed by reducing \texttt{succ 0} to \texttt{1}.

        Not only is evaluation nondeterministic, but different evaluation
        orders lead to different results! In other words, we also lose
        \emph{confluence}.

        (I think that the rule $\Pred (\Succ t) \to t$  would give a
        nondeterministic but confluent semantics, since we're certain that
        $\Succ t$ will always evaluate to a successor, so taking the
        predecessor will always work out.)

    \item[E-ISZERO-SUCC.]
        Consider a term like \texttt{iszero (succ (pred 0))}.
        There are two different ways to evaluate this term using the lazy
        variant of E-ISZERO-SUCC. We could go inside using the E-ISZERO
        congruence rule, or we could immediately determine that the inner term
        is nonzero since it has a \texttt{succ} constructor.

        In the presence of the broken E-PRED-SUCC rule, these different
        evaluation strategies will lead to different results. But let's forget
        about the broken E-PRED-SUCC rule and concentrate on whether
        E-ISZERO-SUCC is broken. In this case, it's clear from the rules that
        a leading \texttt{succ} constructor can never be eliminated. Hence,
        once we determine that $\iszero t$ is such that $t$ is of
        the form $\Succ t^\prime$, then it doesn't matter what $t^\prime$ is:
        the leading \texttt{succ} isn't going anywhere.

        Overall, what this means is that if we use the traditional E-PRED-SUCC
        rule, (or possibly as discussed above the lazy variant that doesn't
        reverse the order of the $\Pred$ and $\Succ$ constructors,) then the
        lazy E-ISZERO-SUCC leads to the same results as the strict variant. In
        other words, the system is still \emph{confluent}. However, it would be
        possible to chain the E-ISZERO and E-SUCC congruence rules to evaluate
        inside the $\Succ$. Therefore the system will be nondeterministic.
\end{description}

\section{Values evaluate to themselves}

\begin{prop}
    In the small-step semantics, all values evaluate to themselves,
    i.e. if $v \in V$ and $v \to^* v^\prime$, then $v = v^\prime$.
\end{prop}

\end{document}
