\documentclass[11pt,letterpaper]{article}

\usepackage[geometry]{jakemath}

\author{Jacob Thomas Errington (260636023)}
\title{Assignment \#1\\Language-based security -- COMP 523}
\date{20 September 2017}

\begin{document}

\maketitle

\section{Order of evaluation}

The proposed new evaluation rules for E-PRED-SUCC and E-ISZERO-SUCC affect the
order of evaluation. In particular, rather than require numerical values (i.e.
that the inside be evaluated first), the new rules allow unevaluated terms to
be manipulated. Second, the new rule for E-PRED-SUCC in fact changes the order
of the \texttt{pred} and \texttt{succ} operations.

\begin{description}
    \newcommand{\Succ}{\operatorname{\mathtt{succ}}}
    \newcommand{\Pred}{\operatorname{\mathtt{pred}}}
    \newcommand{\iszero}{\operatorname{\mathtt{iszero}}}
    \item[E-PRED-SUCC.]
        Consider the term \texttt{succ (pred 0)}.
        Using the new, non-strict E-PRED-SUCC rule, we can cancel the
        \texttt{succ pred}, leaving us with \texttt{0}.
        However, we can \emph{also} (and this is where we have nondeterminism)
        evaluate inside the \texttt{succ} to reduce \texttt{pred 0} to
        \texttt{0}, followed by reducing \texttt{succ 0} to \texttt{1}.

        Not only is evaluation nondeterministic, but different evaluation
        orders lead to different results! In other words, we also lose
        \emph{confluence}.

        (I think that the rule $\Pred (\Succ t) \evalto t$  would give a
        nondeterministic but confluent semantics, since we're certain that
        $\Succ t$ will always evaluate to a successor, so taking the
        predecessor will always work out.)

    \item[E-ISZERO-SUCC.]
        Consider a term like \texttt{iszero (succ (pred 0))}.
        There are two different ways to evaluate this term using the lazy
        variant of E-ISZERO-SUCC. We could go inside using the E-ISZERO
        congruence rule, or we could immediately determine that the inner term
        is nonzero since it has a \texttt{succ} constructor.

        In the presence of the broken E-PRED-SUCC rule, these different
        evaluation strategies will lead to different results. But let's forget
        about the broken E-PRED-SUCC rule and concentrate on whether
        E-ISZERO-SUCC is broken. In this case, it's clear from the rules that
        a leading \texttt{succ} constructor can never be eliminated. Hence,
        once we determine that $\iszero t$ is such that $t$ is of
        the form $\Succ t^\prime$, then it doesn't matter what $t^\prime$ is:
        the leading \texttt{succ} isn't going anywhere.

        Overall, what this means is that if we use the traditional E-PRED-SUCC
        rule, (or possibly as discussed above the lazy variant that doesn't
        reverse the order of the $\Pred$ and $\Succ$ constructors,) then the
        lazy E-ISZERO-SUCC leads to the same results as the strict variant. In
        other words, the system is still \emph{confluent}. However, it would be
        possible to chain the E-ISZERO and E-SUCC congruence rules to evaluate
        inside the $\Succ$. Therefore the system will be nondeterministic.
\end{description}

\section{Values evaluate to themselves}

\begin{prop}
    In the small-step semantics, all values evaluate to themselves,
    i.e. if $v \in V$ and $v \evalto^* v^\prime$, then $v = v^\prime$.
\end{prop}

\begin{proof}
    By structural induction on the multistep evaluation $v \evalto^* v^\prime$.

    \begin{description}
        \item[Case] the reflexitivity rule. Then,
            \begin{equation*}
                \infer[\text{refl}]{v \evalto^* v}{}
            \end{equation*}
            for a value $v$,
            so $v = v^\prime$ trivially.

        \item[Case] the transitivity rule. Then,
            \begin{equation*}
                \infer[\text{trans}]{v \evalto^* v^\prime}{
                    v \evalto^* w
                    &
                    w \evalto^* v^\prime
                }
            \end{equation*}
            for values $v$, $w$, and $v^\prime$.

            By the induction hypothesis,
            we deduce that $v = w$ and $w = v^\prime$.
            By the transitivity of equality, we conclude that $v = v^\prime$.

        \item[Case] the singleton rule. Then,
            \begin{equation*}
                \infer[\text{sing}]{v \evalto^* v^\prime}{
                    v \evalto v^\prime
                }
            \end{equation*}
            for values $v$ and $v^\prime$.

            We claim that the premise of this rule is impossible. We prove this
            by structural induction on the value $v$.

            If $v = \mathtt{true}$, then we reach a contradiction, as no
            evaluation rules apply. The same can be said for
            $v = \mathtt{false}$, and $v = \mathtt{0}$.
            One inductive case needs to be checked as well: E-SUCC.
            The induction hypothesis tells us that premise of the E-SUCC rule
            is impossible, so we can deduce that it's conclusion is impossible
            as well.

            Hence, the singleton rule is an impossible case.
    \end{description}

    Intuitively, values step to themselves in the multistep semantics because
    we are only allowed to use reflexivity and to chain reflexivity statements
    together with transitivity.
\end{proof}

\section{An alternative formulation of the multistep rules}

Consider the alternative formulation of the multistep semantics.
%
\newcommand{\altevalto}{\Longrightarrow}

\begin{equation*}
    \infer[\text{m-refl}]{t \altevalto^* t}{}
    \quad
    \infer[\text{m-step}]{t \altevalto^* t^\prime}{
        t \evalto s
        &
        s \altevalto^* t^\prime
    }
\end{equation*}

\begin{prop}
    The step formulation is equivalent to the reflexive-transitive
    closure formulation, i.e.
    $t \evalto^* t^\prime$ if and only if $t \altevalto^* t^\prime$.
\end{prop}

Before engaging in the proof of this proposition, we require the following
transitivity lemma for the step formulation.

\begin{lem}[Transitivity]
    Suppose $\mathcal{D}_1 : t \altevalto^* s$
    and $\mathcal{D}_2 : s \altevalto^* t^\prime$.
    Then, there exists $\mathcal{E}$
    such that $\mathcal{E} : t \altevalto^* t^\prime$.
\end{lem}

\begin{proof}
    By induction on the structure of the derivation $\mathcal{D}_1$.

    \begin{description}
        \item[Case] reflexivity.

            Then we have that $\mathcal{D}_1$ is of the form
            %
            \begin{equation*}
                \infer[\text{m-refl}]{t \altevalto^* t}{}
            \end{equation*}

            Then we discover that $s = t$,
            so by substitution we have that
            %
            \begin{equation*}
                \deduce[\mathcal{D}_2]{t \altevalto^* t^\prime}{}
            \end{equation*}
            %
            Thus we simply take $\mathcal{E} = \mathcal{D}_2$.

        \item[Case] step.

            Then we have that $\mathcal{D}_1$ is of the form
            %
            \begin{equation*}
                \infer[\text{m-step}]{
                    t \altevalto^* s
                }{
                    \deduce[\mathcal{F}]{t \evalto t_1}{}
                    &
                    \deduce[\mathcal{D}_1^\prime]{t_1 \altevalto^* s}{}
                }
            \end{equation*}

            We apply the induction hypothesis on
            $\mathcal{D}_1^\prime : t_1 \altevalto^* s$
            and
            $\mathcal{D}_2 : s \altevalto^* t^\prime$
            to obtain
            $\mathcal{E}^\prime : t_1 \altevalto^* t^\prime$.

            Then we construct $\mathcal{E}$ as follows.
            %
            \begin{equation*}
                \infer[\text{m-step}]{
                    t \altevalto^* s
                }{
                    \deduce[\mathcal{F}]{t \evalto t_1}{}
                    &
                    \deduce[\mathcal{E}^\prime]{t_1 \altevalto^* t^\prime}{}
                }
            \end{equation*}
    \end{description}
\end{proof}

\begin{proof}[Proof of proposition.]
    Suppose $t \evalto^* t^\prime$. Now proceed by induction on the evaluation
    derivation.

    \begin{description}
        \item[Case] reflexivity.
            Then we have
            \begin{equation*}
                \infer[\text{refl}]{t \evalto^* t}{}
            \end{equation*}
            so we discover that $t = t^\prime$.

            We construct a derivation for $t \altevalto^* t^\prime$ using the
            \texttt{m-refl} rule:
            %
            \begin{equation*}
                \infer[\text{m-refl}]{t \altevalto^* t}{}
            \end{equation*}

        \item[Case] singleton.
            Then we have
            \begin{equation*}
                \infer[\text{sing}]{t \evalto^* t^\prime}{
                    \deduce[\mathcal{D}]{t \evalto t^\prime}{}
                }
            \end{equation*}

            We construct a derivation for $t \altevalto^* t^\prime$.
            %
            \begin{equation*}
                \infer[\text{m-step}]{t \evalto^* t^\prime}{
                    \deduce[\mathcal{D}]{t \evalto t^\prime}{}
                    &
                    \infer[\text{m-refl}]{t^\prime \altevalto^* t^\prime}{}
                }
            \end{equation*}

        \item[Case] transitivity.
            Then we have
            \begin{equation*}
                \infer[\text{trans}]{t \evalto^* t^\prime}{
                    \deduce[\mathcal{D}_1]{t \evalto^* s}{}
                    &
                    \deduce[\mathcal{D}_2]{s \evalto^* t^\prime}{}
                }
            \end{equation*}

            By the induction hypothesis, we obtain the following derivations.
            \begin{align*}
                \mathcal{E}_1 &: t \altevalto^* s \\
                \mathcal{E}_2 &: s \altevalto^* t^\prime
            \end{align*}

            Using the transitivity lemma for the step formulation, we obtain
            $\mathcal{E} : t \altevalto^* t^\prime$, as required.
    \end{description}

    Next, suppose that $t \altevalto^* t^\prime$.
    Now we proceed by induction on the step formuation evaluation derivation.

    \begin{description}
        \item[Case] reflexivity.

            Then we have
            %
            \begin{equation*}
                \infer[\text{m-refl}]{t \altevalto^* t}{}
            \end{equation*}
            %
            so we discover that $t^\prime = t$.

            We construct the derivation $\mathcal{E}$ as follows.
            %
            \begin{equation*}
                \infer[\text{refl}]{t \evalto^* t}{}
            \end{equation*}

        \item[Case] step.

            Then we have
            %
            \begin{equation*}
                \infer[\text{m-step}]{
                    t \altevalto^* t^\prime
                }{
                    \deduce[\mathcal{F}]{t \evalto s}{}
                    &
                    \deduce[\mathcal{D}^\prime]{s \altevalto^* t^\prime}{}
                }
            \end{equation*}

            We apply the induction hypothesis to
            $\mathcal{D}^\prime : s \altevalto^* t^\prime$
            to obtain
            $\mathcal{E}^\prime : s \evalto^* t^\prime$.

            We construct the derivation $\mathcal{E}$ as follows.
            %
            \begin{equation*}
                \infer[\text{trans}]{
                    t \evalto^* t^\prime
                }{
                    \infer[\text{sing}]{t \evalto^* s}{
                        \deduce[\mathcal{F}]{t \evalto s}{}
                    }
                    &
                    \deduce[\mathcal{E}^\prime]{s \evalto^* t^\prime}{}
                }
            \end{equation*}
    \end{description}

    This completes the proof that the step formulation is equivalent to the
    reflexive-transitive closure formulation.
\end{proof}

\section{Less than or equals}
\newcommand{\Succ}{\operatorname{\mathtt{succ}}}
\newcommand{\Leq}[2]{\operatorname{\mathtt{leq}} #1 \; #2}

We extend the arithmetic language with a built-in comparison operator.

\begin{enumerate}
    \item
        Here are evaluation rules for the new operator.

        \begin{align*}
            \infer[\text{E-LEQ-1}]{
                \Leq{t}{s}
                \evalto
                \Leq{t^\prime}{s}
            }{
                t \evalto t^\prime
            }
            &\quad
            \infer[\text{E-LEQ-2}]{
                \Leq{v}{s}
                \evalto
                \Leq{v}{s^\prime}
            }{
                s \evalto s^\prime
            }
            \\
            \infer[\text{E-LEQ-Z}]{
                \Leq{0}{v}
                \evalto
                \mathtt{true}
            }{}
            &\quad
            \infer[\text{E-LEQ-SUCC}]{
                \Leq{(\Succ v_1)}{(\Succ v_2)}
                \evalto
                \Leq{v_1}{v_2}
            }{}
            \\
            \infer[\text{E-LEQ-SUCC-Z}]{
                \Leq{(\Succ v)}{0}
                \evalto
                \mathtt{false}
            }{}
            &\quad
        \end{align*}
\end{enumerate}

The rules are based on the following intuitive idea of how evaluation should
proceed.
\begin{enumerate}
    \item Evaluate the first operand.
    \item Evaluate the second operand.
    \item Peel off successors until a zero is reached.
    \item If the first operand is zero, then the comparison is true.
    \item Else, the second operand is zero, so the comparison is false.
\end{enumerate}

\begin{prop}
    The given rules for the comparison operator are deterministic,
    i.e. if $\mathcal{D}_1 : s \evalto s_1$
    and $\mathcal{D}_2 : s \evalto s_2$,
    then $s_1 = s_2$.
\end{prop}

\begin{proof}
    By structural induction on $\mathcal{D}_1 : s \evalto s_1$.

    \begin{description}
        \item[Case] E-LEQ-Z.

            Then we have $s = \Leq{0}{v}$
            for a value $v$
            and $s_1 = \mathtt{true}$.

            Let's examine which rules could construct $\mathcal{D}_2$.
            %
            \begin{itemize}
                \item
                    Since $0$ is a value, E-LEQ-1 could not apply.
                \item
                    Since $v$ is a value, E-LEQ-2 could not apply.
                \item
                    Since $0$ is distinct from $\Succ u$ for any $u$,
                    E-LEQ-SUCC or E-LEQ-SUCC-Z could not apply.
            \end{itemize}

            So the only applicable rule would be E-LEQ-Z, thus showing that
            $s_2 = \mathtt{true} = s_1$.

        \item[Case] E-LEQ-SUCC.

            Then we have $s = \Leq{(\Succ v_1)}{(\Succ v_2)}$
            and $s_1 = \Leq{v_1}{v_2}$.

            Let's examine which rules could construct $\mathcal{D}_2$.
            %
            \begin{itemize}
                \item
                    Since $0$ is distinct from $\Succ u$ for any $u$,
                    E-LEQ-Z or E-LEQ-SUCC-Z could not apply.
                \item
                    Since $\Succ v_1$ is a value, E-LEQ-1 could not apply.
                \item
                    Since $\Succ v_2$ is a value, E-LEQ-2 could not apply.
            \end{itemize}

            So the only applicable rule would be E-LEQ-SUCC, thus showing that
            $s_2 = \Leq{v_1}{v_2} = s_1$.

        \item[Case] E-LEQ-SUCC-Z.

            Then we have $s = \Leq{(\Succ v)}{0}$
            and $s_1 = \mathtt{false}$.

            I'll spare a full listing.
            The only rule that admits $\Leq{(\Succ v)}{0}$ on the left-hand
            side in the conclusion is E-LEQ-SUCC-Z; every other rule would
            require either one of the operands to be an unevaluated term or
            different constructors for the operands.

            Since the same rule had to be used in $\mathcal{D}_1$ and in
            $\mathcal{D}_2$, we can conclude that $s_1 = s_2$.

        \item[Case] E-LEQ-2.

            Then we have $s = \Leq{v}{t}$
            and $s_1 = \Leq{v}{t_1}$.

            The only rule that admits $\Leq{v}{t}$ on the left hand side of the
            conclusion is E-LEQ-1; the other rules require that both operands
            be fully evaluated or that neither be fully evaluated.

            Since the rule E-LEQ-2 had to be used in $\mathcal{D}_1$ and in
            $\mathcal{D}_2$, we obtain smaller derivations
            $\mathcal{D}_1^\prime : t \evalto t_1$
            and $\mathcal{D}_2^\prime : t \evalto t_2$.
            Applying the induction hypothesis, we deduce that
            $t_1 = t_2$.
            By a substitution, we conclude that $\Leq{v}{t_1} = \Leq{v}{t_2}$.

        \item[Case] E-LEQ-1.

            This case is the same as E-LEQ-2, with the only difference being
            that we look at the first operand instead of the second.

            No rule but E-LEQ-1 could have been used for $\mathcal{D}_2$ for
            the same reason: E-LEQ-1 is the only rule permitting unevaluated
            terms in both operands.

            We apply the induction hypothesis in the same way and use a
            substitution to conclude that
            $s_1 = \Leq{t_1}{t} = \Leq{t_2}{t} = s_2$.
    \end{description}
\end{proof}

Here is a typing rule for $\Leq{t}{t^\prime}$.

\begin{equation*}
    \infer[\text{t-leq}]{\Leq{t}{t^\prime} \hastype \mathtt{bool}}{
        t \hastype \mathtt{nat}
        &
        t^\prime \hastype \mathtt{nat}
    }
\end{equation*}

\begin{prop}
    Type preservation holds in the language extended by the comparison
    operator,
    i.e.
    if $t \hastype T$ and $t \evalto t^\prime$, then $t^\prime \hastype T$.
\end{prop}

\begin{proof}
    By structural induction on $\mathcal{D} : t \evalto t^\prime$.

    \begin{description}
        \item[Case] E-LEQ-Z.

            Then we have
            %
            \begin{equation*}
                \infer[\text{E-LEQ-Z}]{
                    \Leq{0}{v}
                    \evalto
                    \mathtt{true}
                }{}
            \end{equation*}
            %
            so $t = \Leq{0}{v}$ and $t^\prime = \mathtt{true}$.

            By the typing rule ``t-leq'' we deduce that
            $\Leq{0}{v} \hastype \mathtt{bool}$.
            By the trivial typing rule for $\mathtt{true}$, we deduce that
            $\mathtt{true} \hastype \mathtt{bool}$ as required.

        \item[Case] E-LEQ-SUCC-Z.

            Then we have
            %
            \begin{equation*}
                \infer[\text{E-LEQ-SUCC-Z}]{
                    \Leq{(\Succ v)}{0}
                    \evalto
                    \mathtt{false}
                }{}
            \end{equation*}

            The argument is the same as in the previous case.

        \item[Case] E-LEQ-SUCC.

            Then we have
            %
            \begin{equation*}
                \infer[\text{E-LEQ-SUCC}]{
                    \Leq{(\Succ v_1)}{(\Succ v_2)}
                    \evalto
                    \Leq{v_1}{v_2}
                }{}
            \end{equation*}

            By the typing rule t-leq, we deduce that both sides have type
            $\mathtt{bool}$, so the types are preserved.

        \item[Case] E-LEQ-1.

            Then we have
            %
            \begin{equation*}
                \infer[\text{E-LEQ-1}]{
                    \Leq{u}{s}
                    \evalto
                    \Leq{u^\prime}{s}
                }{
                    \deduce[\mathcal{D}^\prime]{u \evalto u^\prime}{}
                }
            \end{equation*}
            %
            so $t = \Leq{u}{s}$ and $t^\prime = \Leq{u^\prime}{s}$.

            By the typing rule t-leq, we deduce that $u \hastype \mathtt{nat}$
            and $s \hastype \mathtt{nat}$.
            By the induction hypothesis applied to $u \hastype \mathtt{nat}$
            and $\mathcal{D}^\prime$,
            we deduce that $u^\prime \hastype \mathtt{nat}$.
            By the typing rule t-leq, we conclude that
            $\Leq{u^\prime}{s} \hastype \mathtt{bool}$, thus showing that the
            type is preserved.

        \item[Case] E-LEQ-2.

            Then we have
            %
            \begin{equation*}
                \infer[\text{E-LEQ-2}]{
                    \Leq{v}{s}
                    \evalto
                    \Leq{v}{s^\prime}
                }{
                    s \evalto s^\prime
                }
            \end{equation*}
            %
            so $t = \Leq{v}{s}$ and $t^\prime = \Leq{v}{s^\prime}$.

            By the typing rule t-leq, we deduce that $v \hastype \mathtt{nat}$
            and $s \hastype \mathtt{nat}$.
            By the inductive hypothesis applied to $s \hastype \mathtt{nat}$
            and the smaller derivation $\mathcal{D}^\prime$, we deduce that
            $s^\prime \hastype \mathtt{nat}$.
            By the typing rule t-leq, we conclude that
            $\Leq{v}{s^\prime} \hastype \mathtt{bool}$, as required.

            Thus this rule preserves typing.
    \end{description}

    Therefore, evaluation in the extended language preserves typing.
\end{proof}

\end{document}
